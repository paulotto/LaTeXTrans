\section{Preliminary \& Related Work}

% \subsection{Preliminaries}
The following section will introduce the preliminaries of multi-graph matching, while also reviewing the most relevant existing literature on the subject. Subsequently, domain generalization and test-time adaptation will be discussed.
% We will then review the theory of multi-matching through the lens of the notion of universe points.
\subsection{Multi-Graph Matching}
We consider $m \in \mathbb{N}$ different graphs $\mathcal{G}_1, \mathcal{G}_2, \cdots, \mathcal{G}_m$, $i \in [m]$, $[m] := \{1,\cdots,m\}$. For each $\mathcal{G}_i=(\mathcal{V}_i, \mathcal{A}_i)$, $\mathcal{V}_i\in \mathbb{R}^{n_i \times h}$ represents the $h$-dimensional feature of $n_i$ nodes, and adjacency matrix $\mathcal{A}_i\in \mathbb{R}^{n_i \times n_i}$ encodes the connectivity between nodes, represented as the set of edges. For any two graphs $\mathcal{G}_i$ and $\mathcal{G}_j$, the set of $n_i \times n_j$ partial permutation matrices $\mathbb{P}_{n_i n_j}$ is defined as 
\begin{equation}
    \mathbb{P}_{n_i n_j} := \{\bold{X}\in \{0,1\}^{n_i\times n_j}: \bold{X}\bold{1}_{n_j} \leq \bold{1}_{n_i}, \bold{X}^{\mathsf{T}}\bold{1}_{n_i} \leq \bold{1}_{n_j}\},
\end{equation}
where $\bold{1}_{n_i}$ denotes a $n_i$-dimensional column vector whose elements are all ones. The assignment matrix $\bold{X}_{ij} \in \mathbb{P}_{n_i n_j}$ between a pair of graphs $(\mathcal{G}_i, \mathcal{G}_j )$ denotes a meaningful correspondence that encodes the matching. 

When considering the synchronisation matching~\cite{pachauri2013solving,maset2017practical} of multiple graphs $(m>2)$, relying on local matches between graph pairs can readily result in erroneous correspondences that contradict each other globally~\cite{yan2013joint}. Consequently, the multi-graph matching has been addressed in terms of simultaneously solving under the constraints of cycle-consistency~\cite{yan2015multi,yan2014graduated}.

\noindent \textit{\textbf{Definition 1 Cycle-consistency.} The matching among $\mathcal{G}_1, \mathcal{G}_2, \cdots, \mathcal{G}_m$ is cycle-consistent (partial transitivity), if }
\begin{equation}
\label{cycle_consist}
    \bold{X}_{ik}\bold{X}_{kj} \leq \bold{X}_{ij}, \qquad \forall{i,j,k\in [m]}.
\end{equation}

In contrast to full matching, i.e. in Eq. (\ref{cycle_consist}) the inequalities become equalities and $n_i = n_j$, partial matching requires only that the pairwise matching combinations in the cycle form a subset of the identity matching~\cite{bernard2019synchronisation}. 
An iterative refinement strategy~\cite{nguyen2011optimization} was employed to enhance pairwise matchings by ensuring global mapping consistency. In~\cite{pachauri2013solving}, the authors proposed a method to achieve cycle-consistency based on spectral decomposition of the assignment matrix. In addition, methods such as convex programming~\cite{chen2014near}, low-rank matrix recovery~\cite{zhou2015multi}, and spectral decomposition~\cite{maset2017practical} have been proposed to address the synchronization of partial matchings.

Instead of explicitly modeling the cubic number of non-convex quadratic constraints, a more efficient approach to enforcing cycle consistency is the use of the \textit{universe of nodes}~\cite{tron2017fast}. In a universe comprising nodes of size $d \in \mathbb{N}$, the matching between graphs can be decomposed by matching each graph $\mathcal{G}_i$ to the space of universe (refer to Fig.~\ref{fig:moti}). We denote the \textit{universe matchings} as follows:
\begin{equation}
    \mathbb{U}_{n_id} := \{ \bold{U} \in \{0,1\}^{n_i\times d}: \bold{U}\bold{1}_d=\bold{1}_{n_i}, \bold{U}^{\mathsf{T}}\bold{1}_{n_i} \leq \bold{1}_d \}.
\end{equation}
% where $\bold{U} = [U_1, \cdots, U_m]$ is the compact form of $U_i$.
 \noindent \textit{\textbf{Lemma 1 Cycle-consistency, universe matching.} \label{lemma1} The pairwise (partial) matching matrices $\{ \bold{X}_{ij}\}_{i,j=1}^{m} $ is cycle-consistent iff there exists a collection of universe matching $\{ \bold{U}_i \in \mathbb{U}_{n_id} \}_{i=1}^{m}$, such that for each $\bold{X}_{ij}$, it holds that $\bold{X}_{ij} = \bold{U}_i \bold{U}_{j}^{\mathsf{T}}$. }

The proof will be presented in the appendix. By projecting each node of a graph to the universe of nodes and identifying the universe matchings $\mathbb{U}_{n_id}$, it is ensured that the cycle-consistency constraint will be satisfied throughout the multi-graph matching process. In~\cite{wang2020graduated,wang2023unsupervised}, the authors proposed a gradual assignment procedure to achieve soft matching and clustering through the universe matching. In~\cite{bernard2019hippi}, the universe of nodes is leveraged to incorporate geometric consistency, ensuring both point scaling and convergence. \cite{nurlanov2023universe} utilized the universe matching to facilitate effective partial multi-graph matching.

\noindent \textit{\textbf{Definition 2 Multi-Matching Koopmans-Beckmann's Quadratic Assignment Problem (KB-QAP)~\cite{koopmans1957assignment}.} Multi-graph matching is formulated with KB-QAP, by summing KB-QAP objectives among all pairs of graphs:
\begin{align}
\label{eq:mgqap}
    \underset {\bold{X}_{ij}} {\max} \underset {i,j\in [m]}\sum ( & \lambda \cdot \text{tr}(\bold{X}_{ij}^{\mathsf{T}}\mathcal{A}_i\bold{X}_{ij}\mathcal{A}_j) + \text{tr}(\bold{X}_{ij}^{\mathsf{T}} \bold{M}_{ij})), \\ \nonumber
    s.t.  \quad & \bold{X}_{ij}\in \{ 0,1\}^{n_i\times n_j}, \bold{X}_{ij} \in \mathbb{P}_{n_i n_j},  \\ \nonumber
    & \bold{X}_{ik} \bold{X}_{kj} \leq \bold{X}_{ij}, \quad \forall i,j,k\in [m].
\end{align} }

\noindent In Eq. (\ref{eq:mgqap}), $\lambda$ is a scaling factor for edge-to-edge similarity, and $\bold{M}_{ij}$ is the node-to-node similarity between $\mathcal{G}_i, \mathcal{G}_j$.
% The first term in the objective function measures edge-to-edge similarity to ensure structural consistency, scaled by $\lambda$, while the second term captures node-to-node similarity $\bold{M}_{ij}$ between $\mathcal{G}_i, \mathcal{G}_j$. This objective is subject to the binary constraint, partial permutation matrix constraint, and cycle constraint.
% }
% The first term in the objective function quantifies edge-to-edge similarity for structural consistency, controlled by a scaling factor $\lambda$, while the second term captures node-to-node similarity $\bold{M}_{ij}$ between $\mathcal{G}_i, \mathcal{G}_j$. This objective is subject to the binary constraint, partial permutation matrix Constraint, and cycle-constraint.

\subsection{Domain Generalization}

Domain generalization addresses a challenging scenario where one or more different but related domains are provided, with the objective of training a model that can generalize effectively to an unseen test domain~\cite{wang2022generalizing,zhou2022domain,yoon2023domain}. Medical image analysis encounters significant challenges due to factors such as variability in image appearance, the complexity and high dimensionality of the data, difficulties in data acquisition, and issues related to data organization, labeling, safety, and privacy. Previous research has proposed DG algorithms across multiple levels, including data-level~\cite{yu2023san,zhang2020generalizing}, feature-level~\cite{bi2023mi,li2018domain}, model-level~\cite{gu2023cddsa,peng2019moment,zuo2021attention}, and analysis-level~\cite{liu2022single,qiao2020learning}. Yu \textit{et al.}~\cite{yu2023san} proposed a U-Net z-score nomalization network for the stroke lesion segmentation. Liu \textit{et al.}~\cite{liu2022single} employed dictionary learning for prostate and fundus segmentation by creating a shape dictionary composed of template masks. In~\cite{bi2023mi}, the authors utilized mutual information to differentiate between domain-invariant features and domain-specific ones in ultrasound image segmentation. Gu \textit{et al.}~\cite{gu2023cddsa} proposed a domain-style contrastive learning approach that disentangles an image into invariant representations and style codes for DG. 
% Our experiments indicate that DG methods still show model bias toward existing training datasets in complex real-world medical scenarios. To tackle this issue, we propose a multi-graph matching TTA that integrates geometric priors from medical images to enhance model performance.

\subsection{Test-Time Adaptation}

Test-time adaptation is an emerging paradigm that allows a pre-trained model to adapt to unlabeled data during the testing phase, before making predictions~\cite{liang2024comprehensive}.
Several TTA methods have been proposed recently, utilizing techniques such as self-supervised learning~\cite{chen2023improved,kundu2022concurrent,zhang2022memo}, batch normalization calibration~\cite{zhang2023domainadaptor,chen2024each,zou2022learning,schneider2020improving,nado2020evaluating}, and input data adaptation~\cite{shu2022test,karani2021test} to achieve better test-time performance. Typically, VPTTA~\cite{chen2024each} is a method that freezes the pre-trained model and generates low-frequency prompts for each image during inference in medical image segmentation. In~\cite{karani2021test}, the authors proposed designing two CNN-based sub-networks along with an image normalization network. During test-time training, the image normalization network was adapted for each image. \cite{zhang2023domainadaptor} proposed a dynamic mixture coefficient and a statistical transformation operation to adaptively merge the training and testing statistics of the normalization layers. Additionally, the authors design an entropy minimization loss to address the issue of domain shifts. 
% In contrast to the previously mentioned TTA methods, we are the first to integrate organ shapes and morphological priors from medical images by incorporating geometric consistency within the framework of multi-graph matching. This approach not only ensures cycle-consistency across multiple graphs but also enhances performance in medical segmentation. Our innovative methodology effectively tackles the challenges posed by domain shifts in medical imaging, improving the model's adaptability and robustness in diverse clinical environments.

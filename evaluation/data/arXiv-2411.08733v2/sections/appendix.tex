\newpage

\section{More Implementation Details}
\label{sec:impl_details}

\subsection{Hyper-parameters for \ours}
\begin{table}[H]
\begin{center}
\begin{tabular}{ c c c c } 
    \toprule
    \textbf{Experiment} & \bm{$W$} & \bm{$M$} & \bm{$D$} \\
    \midrule
    ICL optimization & 1 & 1 & 5 \\
    System Prompt optimization & 2 & 3 & 20 \\
    \bottomrule
\end{tabular}
\caption{All the hyper-parameters used by \ours during ICL optimization and system prompt optimization.}
\end{center}
\label{tab:hyper_params_beam}
\end{table}

\subsection{Baselines}

\noindent \textbf{Monte Carlo Search}: Monte Carlo search performs directionless 1-step sampling multiple times. The sampling method was kept the same as \ours; we sampled 120 prompts in this method to keep the cost the same as \ours and ensure a fair comparison.

\noindent \textbf{Greedy Search}: Greedy search is the special case of beam search with beam width $W$ fixed as 1, the sampling method, number of action samples per state $M$ was kept the same as \ours but still as the beam width has decreased in this method the overall cost is lower.

\noindent \textbf{Static Rewarding}: In this method, we keep the search algorithm the same as \ours. Instead of choosing dynamic aspects, we always provide a fixed set of aspects to the optimizer and evaluator. The fixed set of aspects was chosen as helpfulness, clarity, factuality, depth, engagement, and safety i.e. the evaluation aspects. This allowed the static rewarding method to perform the best on evaluation metrics and establish a strong baseline. Note that we keep the number of in-context learning examples as 2 while evaluating this baseline.

\subsection{Seed Samples}
Out of the 180 samples in the sampled dataset, $47.8 \%$ of samples comes from \texttt{AlpacaEval}, $28.9 \%$ from LIMA, and the rest from \texttt{HH-RLHF-redteam}. We ensure a fair evaluation by only sampling examples that are not present in the evaluation dataset.

\subsection{Base ICL Examples}
\label{sec:i_base}
Examples in $\mathcal{I}_{base}$ are classified into two groups: ``unethical'', which teaches the model to handle malicious queries, and ``informative'', which teaches the model to present relevant information in an acceptable format. $\mathcal{I}_{base}$, contains an equal number of ``unethical'' queries and ``informative'' queries. 

\subsection{Cost Analysis of \ours}
\label{sec:i_cost}
\noindent \textbf{System Prompt Optimization}. 
Our optimization process leverages a beam search strategy, with the number of sampled prompts being determined by the parameters $W$ (beam width), $M$ (number of action samples per state), and $D$ (beam depth). Specifically, these parameters result in:

    \begin{enumerate}
        \item  $W \times M \times D$ API calls to the optimizer LLM $\mathcal{O}$ for prompt sampling.
        \item $D$ API calls to LLM for reward selection of seed samples.
        \item $W \times M \times D$ calls to base LLM $\mathcal{B}$ for response generation corresponding to each of the sampled prompts.
        \item $W \times M \times D$ API calls to the evaluator LLM $\mathcal{E}$ for sampled prompt evaluation using seed samples.
    \end{enumerate}

Thus, the overall cost ($C_{\text{system}}$), including both API calls and base LLM inferences, for system prompt optimization can be expressed as:

\begin{align*}
    C_{\text{system}} = & \underbrace{W \times M \times D}_{\text{prompt sampling}}
    + \underbrace{D}_{\text{reward selection}} + \\
    & \underbrace{W \times M \times D}_{\text{response generation}}
    + \underbrace{W \times M \times D}_{\text{prompt evaluation}}
\end{align*}
    
Notably, the reward selection cost is incurred only once, as these results are cached and reused across all models. Moreover, the system prompt optimization is also a one-time process for each model; once optimized, the prompts can be reused without incurring additional costs. This approach ensures that the incurred cost is limited and does not scale with the number of subsequent uses.

\noindent \textbf{ICL Optimization}.
Similar to System prompt optimization we can also use beam search for ICL optimization. The cost for optimizing one ICL example is as follows:

    \begin{enumerate}
        \item A single API call to LLM for reward selection of the example.
        \item $W \times M \times D$ API calls to the evaluator LLM to evaluate the ICL example. (amounting to 5 given the hyperparameters)
        \item $W \times M \times D$ API calls to the optimizer LLM, for optimizing the ICL example.
    \end{enumerate}

Thus, the total cost ($C_{\text{ICL}}$) for ICL optimization can be expressed as:

\begin{align*}
    C_{\text{ICL}} = \quad & (\underbrace{1}_{\text{reward selection}} + \underbrace{W \times M \times D}_{\text{evaluation}} + \\
    & \underbrace{W \times M \times D}_{\text{eptimization}} ) \times N
\end{align*}

where $N$ denotes the number of examples we want to optimize.

ICL examples are model-agnostic and can be reused across different models, thus making the optimization cost a one-time expense per example.

\newpage

\section{Categorized Performance}
\label{sec:cat_perf}


\subsection{Mistral 7b}
\begin{figure}[h]
\centering
\begin{subfigure}[b]{.5\textwidth}
  \centering
  \includegraphics[width=0.95\linewidth]{images/mistral_1.png}
  \label{fig:cat_mistral_1}
\end{subfigure}%

\vspace{1em}

\begin{subfigure}[b]{.5\textwidth}
  \centering
  \includegraphics[width=0.95\linewidth]{images/mistral_2.png}
  \label{fig:cat_mistral_2}
\end{subfigure}
\caption{Categorized performance of Mistral 7b across various domains. Using \ours we see a strong improvement in performance across all domains. Notably, we can see that domains like Humanities, Reasoning, STEM improves significantly. This highlights the fact that base models can benefit a great deal from \ours.  }
\label{fig:categorized_performance_mistral}
\end{figure}

\newpage
\subsection{Llama 2 70b}
\begin{figure}[h]
\centering
\begin{subfigure}[b]{.5\textwidth}
  \centering
  \includegraphics[width=0.95\linewidth]{images/llama_1.png}
  \label{fig:cat_llama_1}
\end{subfigure}%

\vspace{1em}

\begin{subfigure}[b]{.5\textwidth}
  \centering
  \includegraphics[width=0.95\linewidth]{images/llama_2.png}
  \label{fig:cat_llama_2}
\end{subfigure}
\caption{Categorized performance of Llama 2 70$b^q$ across various domains. Using \ours we see an improvement in performance across all domains barring math where we see a small drop. The performance using \ours strongly improves domains such as Info-seek, Coding, and Finance.  }
\label{fig:categorized_performance_llama}
\end{figure}

\newpage
\subsection{\texttt{gpt-3.5-turbo}}

\begin{figure}[h]
\centering
\begin{subfigure}[b]{.5\textwidth}
  \centering
  \includegraphics[width=0.95\linewidth]{images/gpt_1.png}
  \label{fig:cat_gpt_1}
\end{subfigure}%

\vspace{1em}

\begin{subfigure}[b]{.5\textwidth}
  \centering
  \includegraphics[width=0.95\linewidth]{images/gpt_2.png}
  \label{fig:cat_gpt_2}
\end{subfigure}
\caption{Categorized performance of \texttt{gpt-3.5-turbo} across various domains. The results for \texttt{gpt-3.5-turbo} are promising because using \ours, the performance has improved across all domains. \\
Note: \ours method has been applied to RLHF-tuned \texttt{gpt-3.5-turbo} as we don't have access to the base model.
}
\label{fig:categorized_performance_gpt}
\end{figure}


\newpage
\section{Optimization Algorithms}
\label{sec:opti_algo}


\subsection{ICL optimization}

\begin{algorithm}[h]
\caption{ICL Optimization}\label{alg:icl_opti}


\KwIn{$\mathcal{I}_{base}$, $N$, $\mathcal{O}$, $\mathcal{E}$, $\mathcal{R}$, $D$, $W$, $M$,  $\mathcal{T}$}
\KwOut{$\mathcal{I}^{*}$}

\SetKwBlock{Definitions}{Definitions}{}
\Definitions{

    $\mathcal{I}_{base}$: base ICL examples\;
    $N$: number of ICL examples\;
    $\mathcal{O}$: optimizer\;
    $\mathcal{E}$: evaluator\;
    $\mathcal{R}$: reward function\;
    $D$: beam depth\;
    $W$: beam width\;
    $M$: number of action samples per state\;
    
    $\mathcal{T}: \mathcal{S} \times \mathcal{A} \rightarrow \mathcal{S}$: transition function
   
}

\For{i $= 1$ to $N$}
{
        $(q_i, b_i)$ $ = \mathcal{I}_{\text{base}}[i]$\;
        $s_0 = b_i$ \tcp*[r]{Initialize state}
        Initialize beam with $s_0$\;
        \For{t $= 1$ to $D$}
        {   
            next\_beam = []\;
            \For{j $= 1$ to min(len(beam), $W$)}
            {
                $s_{{t-1}_{j}}$ = beam[j]\;
                $r_{{t-1}_j} = \mathcal{R}(s_{{t-1}_{j}} \mid \mathbb{R}_{q_i})$\;

                \SetKwBlock{SampleMtimes}{Repeat (sample) $M$ times:}{end}
                \SampleMtimes{
                    $a_{{t-1}_j} = \mathcal{E}(s_{{t-1}_j} \mid \mathbb{R}_{q_i})$\;
                    $s_{t_j} = \mathcal{T}(s_{{t-1}_j}, a_{{t-1}_j})$\;
                    Add $s_{t_j}$ to \textit{next\_beam}\;
                }
            }
            beam = top $W$ states from next\_beam\;
        }
        $s^{*}_{\mathcal{D}}$ = final state of the top beam\;
        $\mathcal{I}^*[i] = (q_i, s^{*}_{\mathcal{D}})$\;
}

\Return $\mathcal{I}^*$
\end{algorithm}


\newpage
\subsection{System Prompt Optimization}
\begin{algorithm}[h]
\caption{System Prompt Optimization}\label{alg:prompt_opti}


\KwIn{$\mathcal{I}^*$, $\mathcal{B}$, $\mathcal{O}$, $\mathcal{E}$, $\mathcal{R}$, 
$\mathcal{X}$.
$\mathcal{P}$, $D$, $W$, $M$, $\mathcal{T}$}
\KwOut{$\mathcal{P}^{*}$}

\SetKwBlock{Definitions}{Definitions}{}
\Definitions{
    
    $\mathcal{I}^*$: optimized ICL examples\;
    $\mathcal{B}$: base LLM\;
    $\mathcal{O}$: optimizer model\;
    $\mathcal{E}$: evaluator model\;
    $\mathcal{R}$: reward function\;
    $\mathcal{X}$: seed dataset\;
    $\mathcal{P}$: initial system prompt\;
    $D$: beam depth\;
    $W$: beam width\;
    $M$: number of action samples per state\;
    
    $\mathcal{T}: \mathcal{S} \times \mathcal{A} \rightarrow \mathcal{S}$: transition function
   
}



$s_0 = \mathcal{P}$ \tcp*[r]{Initialize state}
Initialize beam with $s_0$\;
\For{t $= 1$ to $D$}
{   
    $x_{t-1} = \mathcal{X}$[$t-1$]\;
    $\mathcal{I}_K^*$ = $K$ examples most similar to $x_{t-1}$ from $\mathcal{I}^*$\tcp*[r]{example selection}
    next\_beam = []\;
    \For{j $= 1$ to min(len(beam), $W$)}
    {
        $s_{{t-1}_{j}}$ = beam[j]\;
        $r_{{t-1}_j} = \mathcal{R}(\mathcal{B}(x_{t-1} \mid s_{{t-1}_{j}}, \mathcal{I}_K^*) \mid \mathbb{R}_{x_{t-1}})$\;
        
        \SetKwBlock{SampleMtimes}{Repeat (sample) $M$ times:}{end}
        \SampleMtimes{
            $a_{{t-1}_j} = \mathcal{E}(\mathcal{B}(x_{t-1} \mid s_{{t-1}_{j}}, \mathcal{I}_K^*) \mid \mathbb{R}_{x_{t-1}})$\;
            $s_{t_j} = \mathcal{T}(s_{{t-1}_j}, a_{{t-1}_j})$\;
            Add $s_{t_j}$ to \textit{next\_beam}\;
        }
    }
    beam = top $W$ states from next\_beam\;
}
$s^{*}_{\mathcal{D}}$ = final state of top beam\;
$\mathcal{P}^* = s^{*}_{\mathcal{D}}$\;


\Return $\mathcal{P}^*$
\end{algorithm}



\newpage
\section{Optimized Prompt Case Study}
\label{sec:prompt_case_study}


\begin{table}[h]
% \vspace{-2mm}

% \vspace{-1mm}
\definecolor{Gray}{gray}{0.90}
\newcolumntype{a}{>{\columncolor{Gray}}c}
\centering
\resizebox{1\linewidth}{!}{%
\begin{tabular}{@{}lp{10cm}@{}}
\toprule
\textbf{Model} & \textbf{Optimized Prompt} \\
\midrule
Mistral 7b & \ctext[RGB]{255,225,255}{As a helpful and ethical assistant, your mission is to provide responses that are not only accurate and safe but also deeply engaging, empathetic, and rich in content. Your role is to thoroughly understand the context of each query, offering insights that demonstrate a comprehensive grasp of the subject matter} \ctext[RGB]{255,230,200}{while being mindful of ethical considerations. Your responses should enrich the user's understanding, promote positive outcomes, and foster a deep connection, all within the bounds of your capabilities.} \ctext[RGB]{255,230,230}{It's crucial to directly address the user's query, providing concise yet comprehensive information,}\ctext[RGB]{233,252,232}{and to be transparent about your limitations.}\ctext[RGB]{255,230,230}{Enhance the user experience by making your responses as engaging, creative, and human-like as possible.}
\ctext[RGB]{233,252,232}{- You do not have access to the internet or real-time data, and you are unable to take physical actions. Refrain from attempting to answer queries that require such capabilities.}
\ctext[RGB]{255,230,200}{- Avoid engaging with queries that could promote illegal activities, harm to others, or unethical behavior. Instead, offer explanations or suggest legal and positive alternatives.}
\ctext[RGB]{255,230,230}{- Strive for creativity by using vivid language, incorporating storytelling elements, and providing relatable examples that resonate with the user.
- Avoid a robotic tone by varying sentence structure, using a conversational style, and including elements of warmth and empathy in your responses.
- Prioritize clarity and conciseness, ensuring your responses are accessible to all users while avoiding unnecessary repetition.}
\ctext[RGB]{255,225,255}{- Encourage critical thinking by presenting multiple viewpoints or considerations, inviting users to explore the topic further.}
\ctext[RGB]{233,252,232}{- Be transparent about the speculative nature of certain responses and your limitations, suggesting areas for further inquiry or related topics that might offer additional insights.} 
 \\ \midrule
\texttt{gpt-3.5-turbo} & \ctext[RGB]{255,225,255}{As a helpful and ethical assistant, your primary goal is to provide responses that are accurate, engaging, clear, and emotionally resonant across a wide range of queries. Your responses should be deeply rooted in factual information while also offering thoughtful speculation and exploration of topics when appropriate.} \ctext[RGB]{230, 230, 255}{It's essential to delve into authorial intent, historical contexts, and cultural significance to add depth and foster critical thinking.}\ctext[RGB]{230,246,255}{Strive to make complex topics understandable and emotionally engaging, communicating in a human-like and relatable manner. Organize your responses to enhance readability and emotional connection, avoiding overly technical jargon.} \ctext[RGB]{233,252,232}{When faced with limitations or requests for harmful information, prioritize safety, legality, and ethical considerations.
Always acknowledge the limitations of your knowledge, especially when speculating about historical 'what-ifs', future predictions, or interpreting emotions. Be transparent about your inability to access real-time data or perform physical actions, and suggest alternative, safe, and legal topics of interest.}
\ctext[RGB]{255,225,255}{Aim for a balance between detailed, informative content and a conversational, engaging tone. Incorporate storytelling elements, examples, analogies, and direct questions to make information relatable.} \ctext[RGB]{230,246,255}{Avoid overwhelming the user with excessive information; structure your responses to be clear, well-organized, and mindful of the user's cognitive load.}
 \\

 \bottomrule
\end{tabular}
}
\label{tab:opti_prompt_case_study} 
\caption{
Comparison of the optimized prompts by \ours for Mistral 7b and \texttt{gpt-3.5-turbo}. \ours customizes the prompt to identify and fix alignment weaknesses specific to any model. (The semantics for color labels can be found below.)
}
% \vspace{-1mm}
\end{table}

We highlight different aspects of the optimized prompts with colors, including \ctext[RGB]{233,252,232}{Limitations such as no access to real-time data}, \ctext[RGB]{255,230,230}{Guidance to avoid repetition tailored for a small model like Mistral 7b}, \ctext[RGB]{230,246,255}{Guidance to avoid jargon tailored for a large model like \texttt{gpt-3.5-turbo}}, \ctext[RGB]{255,230,200}{Ethical guidance}, \ctext[RGB]{255,225,255}{General guidelines for an AI assistant}, \ctext[RGB]{230, 230, 255}{Tips to enhance engagement of responses}.



\newpage
\section{Meta Prompts}
\label{sec:meta_prompts}


\subsection{Rewarding Prompt}

In this section, we present the prompt used to compute the overall reward. The reward prompt uses components like eval$\_$dict and reward selection prompt. We first use the reward selection prompt as shown in section \ref{sec:reward_selection_prompt} to select the appropriate rewards, then an eval$\_$dict with the format as shown in section \ref{sec:eval_dict} is created for the selected rewards. Finally, with the list of rewards and eval$\_$dict, we use the reward prompt as shown below to compute dynamic rewards.

\begin{lstlisting}[breaklines=true,breakatwhitespace=true]

Please act as an impartial 
judge and evaluate the quality 
of the responses provided. 
You will rate the quality 
of the output based on 
several selected aspects.

## Query: 
[QUERY]

## Output:
[OUTPUT]

## Evaluate
### Aspects 

Below is a list of 
aspects for evaluating 
the quality of the response:
[ASPECT_LIST]

These aspects are selected 
for the following reasons:
[ASPECT_REASON]

### Format 

Given the query, please rate the quality of the output by scoring it from 1 to 5 individually on **each aspect**. 
- 1: strongly disagree 
- 2: disagree 
- 3: neutral
- 4: agree
- 5: strongly agree

Now, please output your scores and a short rationale below in a JSON format by filling in the placeholders in []:
```
[EVAL_DICT]
```
\end{lstlisting}

\subsubsection{Eval Dict}
\label{sec:eval_dict}
\begin{lstlisting}[breaklines=true,breakatwhitespace=true]

{"Helpfulness": {
        "rationale": "[your thoughts on the helpfulness of the response]",
        "score": "[your helpfulness score]"
    },
    "Clarity": {
        "rationale": "[your thoughts on the clarity of the response]",
        "score": "[your clarity score]"
    },
    "Factuality": {
        "rationale": "[your thoughts on the factuality of the response]",
        "score": "[your factuality score]"
    },
    "Depth": {
        "rationale": "[your thoughts on the depth of the response]",
        "score": "[your depth score]"
    },
    ...... for all chosen rewards
}

\end{lstlisting}

\subsubsection{Reward selection Prompt}
\label{sec:reward_selection_prompt}
\begin{lstlisting}[breaklines=true,breakatwhitespace=true]
Please act as an impartial judge and select the most relevant aspects for providing a high-quality response to the given query. Choose at least 2 and at most 5 aspects from the list below, or propose new aspects if you believe they are important for crafting the best possible response.

## Aspects 
- Helpfulness: The response should directly address the user's query and provide a relevant and practical solution or guidance.
- Clarity: The response should be well-structured and articulate, with ideas presented in a clear, understandable, and coherent manner.
- Factuality: Information provided must be accurate, truthful, and based on reliable sources, acknowledging any uncertainties where applicable.
- Depth: The response should offer an appropriate level of detail and thoroughness, providing a comprehensive understanding of the topic.
- Engagement: The conversation should be engaging, maintaining the user's interest with a natural, conversational tone and possibly interactive elements.
- Conciseness: Information should be conveyed efficiently, avoiding unnecessary complexity or verbosity while maintaining completeness.
- Safety: Responses must adhere to ethical guidelines, promoting positive interactions and avoiding harmful, inappropriate, or sensitive content.
- Compliance: The response should be in line with the instructions provided in the query, ensuring user expectations are met unless there are ethical or safety concerns.
- Limitations: The response should recognize and acknowledge the AI system's limitations, such as lacking up-to-date information, inability to perform searches or physical actions, or any other relevant constraints if applicable.
- Critical-Thinking: The response should question and analyze the information and assumptions presented in the user's query critically, rather than accepting them at face value.
- Creativity: Responses should demonstrate originality and innovation, offering unique perspectives or solutions where appropriate.
- Interactivity: Where applicable, the AI should employ interactive elements like questions, prompts, or actionable suggestions to engage users actively in the conversation.
- Empathy: The AI should aim to recognize and appropriately respond to the user's emotional state and context, fostering a supportive and understanding interaction.
- Sensitivity: Responses should be culturally aware and sensitive, avoiding assumptions and generalizations while respecting diversity.

## Query: 
[QUERY]

## Aspect Selection
Given the query, please analyze its content, intent, and potential challenges in providing a suitable response. Consider the following:

1. What is the main topic or subject of the query?
2. What is the user's intent or goal in asking this question?
3. Are there any potential ambiguities, uncertainties, or missing/wrong information in the query?
4. What type of information or response format would best satisfy the user's needs?
5. Are there any potential challenges or limitations in providing a comprehensive response?

Based on your analysis, select the most relevant aspects for providing a high-quality response. Provide your reasoning for choosing these aspects.

Output your analysis and aspect selection in the following JSON format:
```
{
    "query_analysis": {
        "main_topic": "[main topic or subject of the query]",
        "user_intent": "[user's intent or goal]",
        "ambiguities": "[potential ambiguities, uncertainties, or missing information]",
        "response_format": "[type of information or response format needed]",
        "challenges": "[potential challenges or limitations in providing a response]"
    },
    "aspects_selection": {
        "reasoning": "[your rationale for selecting the aspects based on the query analysis]",
        "selected_aspects": ["aspect1", "aspect2", ...]
    }
}
```
Note: The "selected_aspects" array should contain at least 2 and at most 5 aspects.
\end{lstlisting}






\subsection{State Transition Prompt}

This section describes the prompt used to leverage an LLM as a transition function. Note that in the prompt, we supply `[CURRENT$\_$SYSTEM$\_$PROMPT]', i.e. the current state and the alignment feedback `[OUTPUT$\_$EVALUATION] to generate the next state.

\begin{lstlisting}[breaklines=true,breakatwhitespace=true]
I am designing a system prompt for a language model to generate responses to user queries. The goal is to optimize the quality of the responses across multiple aspects.

The current system prompt is:
[CURRENT_SYSTEM_PROMPT]

When using this prompt to answer the query below:
[QUERY]

The model generates the following output:
[OUTPUT]

Below are the evaluations of the output on multiple aspects:
[OUTPUT_EVALUATION]

There are a list of former system prompts including the current one, and each of them is improved from the previous one:
[FORMER_SYSTEM_PROMPTS]

Based on all the information above, you need to design a new system prompt following the general guidelines below:
1. Make sure the new system prompt is better than the current one.
2. Feel free to modify existing prompts, integrate freshly new instructions, or conceive a completely new one.
3. An evaluation score of 5 in an aspect indicates the best quality, while a score of 1 indicates the worst quality.
4. Try to make the system prompt balance out the quality across all aspects.
5. The prompt MUST be a general one suited for all kinds of queries, NOT specific to the current query.

While designing the system prompt make sure to structure it in a way that it abides to the instructions below:
1. Write some general instructions/statements to the model about what it is supposed to do and it's capabilities in the start.
2. Mention some limitations like no access to internet/real-time data, unable to take physical actions, avoiding answering malicious questions, etc. using bullet points. 
3. Try to list the model capabilities in the bullet points i.e mention that it is better to refuse to answer things it is not capable of answering than giving an unrelated response.
4. Try to generate a prompt in a structure as follows:

    General Instructions about being a helpful, ethical assistant that helps the model to perform better in all the aspects of evaluation provided.
    - Bullet Points containing important and specific instructions to keep in mind.

5. Try to make some bullet points giving instructions/tips to the model on how to make the responses more engaging and human-like, like some pitfalls to avoid sounding robot-like.
6. Try to make some specific tips from the outputs and their evaluation you see above, you can list things to follow or to avoid to make the response better suited as per the evaluation remarks.
7. Try to make the bullent points of the prompt you design to be informative while being succinct.
8. General Instructions you give at the beginning can be detailed or long  and should try to cover as many aspects/issues as possible.
9. When adding bullet points to the system prompt, do NOT add more than 2 bullet points at once.
10. When deleting bullet points, do not remove bullet points which are relevant to overall goal but irrelevant to current query, instead modify/merge those.
11. Do NOT make more than 8 bullet points, if necessary add/modify/merge bullet points.

Please output your new system prompt in the format below by filling in the placeholders in [] in the following JSON format:
```
{
    "analysis": "[carefully examine the evaluation scores and the current system prompt to identify the areas of improvement]",
    "thought": "[your thoughts about how you can improve the current system prompt]",
    "new_system_prompt": "[your new system prompt]"
}
```
\end{lstlisting}
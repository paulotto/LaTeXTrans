%%
%% This is file `sample-manuscript.tex',
%% generated with the docstrip utility.
%%
%% The original source files were:
%%
%% samples.dtx  (with options: `manuscript')
%% 
%% IMPORTANT NOTICE:
%% 
%% For the copyright see the source file.
%% 
%% Any modified versions of this file must be renamed
%% with new filenames distinct from sample-manuscript.tex.
%% 
%% For distribution of the original source see the terms
%% for copying and modification in the file samples.dtx.
%% 
%% This generated file may be distributed as long as the
%% original source files, as listed above, are part of the 
%% same distribution. (The sources need not necessarily be
%% in the same archive or directory.)
%%
%% Commands for TeXCount
%TC:macro \cite [option:text,text]
%TC:macro \citep [option:text,text]
%TC:macro \citet [option:text,text]
%TC:envir table 0 1
%TC:envir table* 0 1
%TC:envir tabular [ignore] word
%TC:envir displaymath 0 word
%TC:envir math 0 word
%TC:envir comment 0 0
%%
%%
%% The first command in your LaTeX source must be the \documentclass command.
% \documentclass[manuscript,screen]{acmart}
\documentclass[manuscript,screen]{acmart}

%%
%% \BibTeX command to typeset BibTeX logo in the docs
\AtBeginDocument{%
  \providecommand\BibTeX{{%
    \normalfont B\kern-0.5em{\scshape i\kern-0.25em b}\kern-0.8em\TeX}}}

%\usepackage{times} % 会使得标题的字体改变
\usepackage{soul}
%\usepackage{url}
%\usepackage[utf8]{inputenc}
\usepackage{graphicx}
\usepackage{booktabs}
\usepackage{float}
\usepackage{overpic}
\usepackage{float}  %设置图片浮动位置的宏包
\usepackage{subfloat}
\usepackage{stfloats} % for pic
% \usepackage{subfigure}  %插入多图时用子图显示的宏包
\usepackage[skip=0.5\baselineskip]{caption}
\usepackage{bm}
\usepackage{amsmath}
%\usepackage{amsthm}
\usepackage{booktabs}
%\usepackage[switch]{lineno}
\usepackage{multirow}
\usepackage{multicol}
\usepackage{enumitem}
\usepackage{subcaption}
\usepackage{threeparttable}
\usepackage{xspace}
%\usepackage{amssymb}
\usepackage{color}
\usepackage[normalem]{ulem}
%\usepackage{algorithm}
%\usepackage{algorithmicx}
\usepackage{algorithmicx,algorithm}
\usepackage{algpseudocode}
\usepackage{algorithm,algpseudocode}
\usepackage{color, xspace}
\newcommand{\eg}[0]{\emph{e.g.,}\xspace}
\newcommand{\zxy}[1]{{\color{blue} [#1 – Xy]}}
\newcommand{\zc}[0]{{\color{blue} [cite]}\xspace}
\newcommand{\yc}[1]{{\color{purple}[yc: #1]}}
\newcommand{\xp}[1]{{\color{blue}[xp: #1]}}
\newcommand{\py}[1]{{\color{red}[py: #1]}}
\newcommand{\xdr}[1]{{\color{green}[xdr: #1]}}
\newcommand{\wl}[1]{{\color{orange}[wl: #1]}}
\newcommand{\edi}[1]{{\color{blue}{#1}}}
% for taxonomy figure drawing
\usepackage{color}
\usepackage{tikz}
\usepackage{xcolor}
\usepackage{xspace}
\usepackage{booktabs}
\usepackage{multirow}
\usepackage{makecell}
\usepackage[edges]{forest}

%% Rights management information.  This information is sent to you
%% when you complete the rights form.  These commands have SAMPLE
%% values in them; it is your responsibility as an author to replace
%% the commands and values with those provided to you when you
%% complete the rights form.
\setcopyright{acmlicensed}
\copyrightyear{2018}
\acmYear{2018}
\acmDOI{XXXXXXX.XXXXXXX}

%% These commands are for a PROCEEDINGS abstract or paper.
\acmConference[Conference acronym 'XX]{Make sure to enter the correct
  conference title from your rights confirmation emai}{June 03--05,
  2018}{Woodstock, NY}
\acmISBN{978-1-4503-XXXX-X/18/06}


%%
%% Submission ID.
%% Use this when submitting an article to a sponsored event. You'll
%% receive a unique submission ID from the organizers
%% of the event, and this ID should be used as the parameter to this command.
%%\acmSubmissionID{123-A56-BU3}

%%
%% For managing citations, it is recommended to use bibliography
%% files in BibTeX format.
%%
%% You can then either use BibTeX with the ACM-Reference-Format style,
%% or BibLaTeX with the acmnumeric or acmauthoryear sytles, that include
%% support for advanced citation of software artefact from the
%% biblatex-software package, also separately available on CTAN.
%%
%% Look at the sample-*-biblatex.tex files for templates showcasing
%% the biblatex styles.
%%

%%
%% The majority of ACM publications use numbered citations and
%% references.  The command \citestyle{authoryear} switches to the
%% "author year" style.
%%
%% If you are preparing content for an event
%% sponsored by ACM SIGGRAPH, you must use the "author year" style of
%% citations and references.
%% Uncommenting
%% the next command will enable that style.
%%\citestyle{acmauthoryear}

%%
%% end of the preamble, start of the body of the document source.
\begin{document}

%%
%% The "title" command has an optional parameter,
%% allowing the author to define a "short title" to be used in page headers.
\title{A Survey of Personalization: From RAG to Agent}

%%
%% The "author" command and its associated commands are used to define
%% the authors and their affiliations.
%% Of note is the shared affiliation of the first two authors, and the
%% "authornote" and "authornotemark" commands
%% used to denote shared contribution to the research.

\author{Xiaopeng Li$^{*}$}
\affiliation{%
  \institution{City University of Hong Kong}
  \country{Hong Kong}
}
\email{xiaopli2-c@my.cityu.edu.hk}

\author{Pengyue Jia$^{*}$}
\affiliation{%
  \institution{City University of Hong Kong}
  \country{Hong Kong}
}
\email{jia.pengyue@my.cityu.edu.hk}

\author{Derong Xu}
\affiliation{%
  \institution{City University of Hong Kong}
  \country{Hong Kong}
}
\affiliation{%
  \institution{University of Science and Technology of China}
  \country{China}
}
\email{derongxu@mail.ustc.edu.cn}

\author{Yi Wen}
\affiliation{%
  \institution{City University of Hong Kong}
  \country{Hong Kong}
}
\email{yiwen23-c@my.cityu.edu.hk}

\author{Yingyi Zhang}
\affiliation{%
  \institution{City University of Hong Kong}
  \country{Hong Kong}
}
\affiliation{%
  \institution{Dalian University of Technology}
  \country{China}
}
\email{yingyizhang@mail.dlut.edu.cn}

\author{Wenlin Zhang}
\affiliation{%
  \institution{City University of Hong Kong}
  \country{Hong Kong}
}
\email{wl.z@my.cityu.edu.hk}

\author{Wanyu Wang}
\affiliation{%
  \institution{City University of Hong Kong}
  \country{Hong Kong}
}
\email{wanyuwang4-c@my.cityu.edu.hk}

\author{Yichao Wang}
\affiliation{%
  \institution{Noah's Ark Lab, Huawei}
  \country{China}
}
\email{wangyichao5@huawei.com}

\author{Zhaocheng Du}
\affiliation{%
  \institution{Noah's Ark Lab, Huawei}
  \country{China}
}
\email{zhaochengdu@huawei.com}

\author{Xiangyang Li}
\affiliation{%
  \institution{Noah's Ark Lab, Huawei}
  \country{China}
}
\email{lixiangyang34@huawei.com}

\author{Yong Liu}
\affiliation{%
  \institution{Noah's Ark Lab, Huawei}
  \country{Singapore}
}
\email{liu.yong6@huawei.com}

\author{Huifeng Guo}
\affiliation{%
  \institution{Noah's Ark Lab, Huawei}
  \country{China}
}
\email{huifeng.guo@huawei.com}

\author{Ruiming Tang$^\dagger$}
\affiliation{%
  \institution{Noah's Ark Lab, Huawei}
  \country{China}
}
\email{tangruiming@huawei.com}

\author{Xiangyu Zhao$^\dagger$}
\affiliation{%
  \institution{City University of Hong Kong}
  \country{Hong Kong}
}
\email{xianzhao@cityu.edu.hk}



% \author{Jianghao Lin$^{*}$}
% \affiliation{%
% \institution{Shanghai Jiao Tong University}
% \country{China}}
% \email{chiangel@sjtu.edu.cn}

% \author{Xinyi Dai$^{*}$}
% \affiliation{%
% \institution{Noah's Ark Lab, Huawei}
% \country{China}}
% \email{daixinyi3@huawei.com}

% \author{Yunjia Xi}
% \affiliation{%
% \institution{Shanghai Jiao Tong University}
% \country{China}}
% \email{xiyunjia@sjtu.edu.cn}

% \author{Weiwen Liu}
% \author{Bo Chen}
% \affiliation{%
% 	\institution{Noah's Ark Lab, Huawei}
% 	\country{China}}
% \email{liuweiwen8@huawei.com}
% \email{chenbo116@huawei.com}

% \author{Hao Zhang}
% \author{Yong Liu}
% \affiliation{%
% 	\institution{Noah's Ark Lab, Huawei}
% 	\country{Singapore}}
% \email{zhang.hao3@huawei.com}
% \email{liu.yong6@huawei.com}

% \author{Chuhan Wu}
% \author{Xiangyang Li}
% \affiliation{%
% 	\institution{Noah's Ark Lab, Huawei}
% 	\country{China}}
% \email{wuchuhan1@huawei.com}
% \email{lixiangyang34@huawei.com}

% \author{Chenxu Zhu}
% \author{Huifeng Guo}
% \affiliation{%
% 	\institution{Noah's Ark Lab, Huawei}
% 	\country{China}}
% \email{zhuchenxu1@huawei.com}
% \email{huifeng.guo@huawei.com}

% \author{Yong Yu}
% \affiliation{%
%   \institution{Shanghai Jiao Tong University}
%   \country{China}}
% \email{yyu@sjtu.edu.cn}

% \author{Ruiming Tang$^\dagger$}
% \affiliation{%
% \institution{Noah's Ark Lab, Huawei}
% \country{China}}
% \email{tangruiming@huawei.com}

% \author{Weinan Zhang$^\dagger$}
% \affiliation{%
%   \institution{Shanghai Jiao Tong University}
%   \country{China}}
% \email{wnzhang@sjtu.edu.cn}

%%
%% By default, the full list of authors will be used in the page
%% headers. Often, this list is too long, and will overlap
%% other information printed in the page headers. This command allows
%% the author to define a more concise list
%% of authors' names for this purpose.
\renewcommand{\shortauthors}{X. Li and P. Jia, et al.}

%%
%% The abstract is a short summary of the work to be presented in the
%% article.
\begin{abstract}
Personalization has become an essential capability in modern AI systems, enabling customized interactions that align with individual user preferences, contexts, and goals. Recent research has increasingly concentrated on Retrieval-Augmented Generation (RAG) frameworks and their evolution into more advanced agent-based architectures within personalized settings to enhance user satisfaction. Building on this foundation, this survey systematically examines personalization across the three core stages of RAG: pre-retrieval, retrieval, and generation. Beyond RAG, we further extend its capabilities into the realm of Personalized LLM-based Agents, which enhance traditional RAG systems with agentic functionalities, including user understanding, personalized planning and execution, and dynamic generation. 
For both personalization in RAG and agent-based personalization, we provide formal definitions, conduct a comprehensive review of recent literature, and summarize key datasets and evaluation metrics. Additionally, we discuss fundamental challenges, limitations, and promising research directions in this evolving field. Relevant papers and resources are continuously updated at the Github Repo\footnote{\url{https://github.com/Applied-Machine-Learning-Lab/Awesome-Personalized-RAG-Agent}}.
\let\thefootnote\relax\footnotetext{* Equal contribution.}
\let\thefootnote\relax\footnotetext{$\dagger$ Corresponding authors.}
\end{abstract}

%%
%% The code below is generated by the tool at http://dl.acm.org/ccs.cfm.
%% Please copy and paste the code instead of the example below.
%%
\begin{CCSXML}
<ccs2012>
   <concept>
       <concept_id>10002951.10003260.10003261.10003271</concept_id>
       <concept_desc>Information systems~Personalization</concept_desc>
       <concept_significance>500</concept_significance>
       </concept>
 </ccs2012>
\end{CCSXML}

\ccsdesc[500]{Information systems~Personalization}

%%
%% Keywords. The author(s) should pick words that accurately describe
%% the work being presented. Separate the keywords with commas.
\keywords{{Large Language Model, Retrieval-Augmented Generation, Agent, Personalization}}

\received{20 February 2007}
\received[revised]{12 March 2009}
\received[accepted]{5 June 2009}

%%
%% This command processes the author and affiliation and title
%% information and builds the first part of the formatted document.
\maketitle


% \vspace{-15pt}
\begin{figure}[h]
    \centering
    \includegraphics[width=0.85\linewidth]{fig/score.png}
    \caption{AIME 2024 scores of \textbf{VAPO} on the Qwen2.5-32B base model, demonstrates significant superiority over the previous state-of-the-art (SOTA) method DAPO, achieving this with notably fewer training steps. The x-axis denotes the gradient update steps.
    }
    \label{fig:front}
\end{figure}

\section{Introduction}

% Start by inserting a figure: Figure 1 compares VAPO, DAPO, and PPO. 

% To frame the discussion, current approaches primarily rely on algorithms lacking a value model, which often suffer from low efficiency. Highlight the advantages of PPO-based improvements—such as enhanced stability and sample efficiency 

% introducing the key modifications proposed in this work.
%\subsection{Hello World}
Reasoning models~\cite{gemini-thinking,qwq,k1.5} such as OpenAI O1 \citep{o1} and DeepSeek R1 \citep{deepseekai2025deepseekr1incentivizingreasoningcapability} have significantly advanced artificial intelligence by exhibiting remarkable performance in complex tasks such as mathematical reasoning, which demand step-by-step analysis and problem-solving through long chain-of-thought (CoT)~\cite{cot_2022} at test time.
Reinforcement learning (RL) plays a pivotal role in the success of these models~\cite{k1.5,areal,dapo,shao2024deepseekmath,hu2025reinforce++,rloo,ahmadian2024basicsrevisitingreinforcestyle,sutton1998rl}. It gradually enhances the model's performance by continuously exploring reasoning paths toward correct answers on verifiable problems, achieving unprecedented reasoning capabilities.


In the Large Language Models (LLM)~\cite{gpt3,gpt4,dsv3,grok,team2023gemini,claude35sonnet,chowdhery2023palm} RL training, value-model-free methods like GRPO~\cite{shao2024deepseekmath} and DAPO~\cite{dapo} have demonstrated remarkable effectiveness. These approaches eliminate the computational overhead of learning a value model, instead computing advantage solely based on the final reward of the entire trajectory. The trajectory-level advantage is then directly assigned as the token-level advantage for each position in the sequence. When training a reliable value model is particularly challenging, value-model-free methods deliver an accurate and stable baseline for advantage calculation by averaging the rewards across multiple trajectories within a group. This group-based reward aggregation mitigates the need for explicit value estimation, which often suffers from instability in complex tasks. Consequently, value-model-free methods have gained significant traction in addressing difficult problems such as long-CoT reasoning, with substantial research efforts focused on optimizing their frameworks.

Despite the notable success achieved by the value-model-free methods, we argue that value-model-based approaches possess a higher performance ceiling if the challenges in training value models can be addressed. First, value models enable more precise credit assignment by accurately tracing the impact of each action on subsequent returns, facilitating finer-grained optimization~\cite{ppo}. This is particularly critical for complex reasoning tasks, where subtle errors in individual steps often lead to catastrophic failures, and it remains challenging for model optimizing under value-model-free frameworks~\cite{vc-ppo}. Secondly, in contrast to the advantage estimates derived from Monte Carlo methods in value-model-free approaches, value models can provide lower-variance value estimates for each token, thereby enhancing training stability. Furthermore, a well-trained value model exhibits inherent generalization capabilities, enabling more efficient utilization of samples encountered during online exploration. This significantly elevates the optimization ceiling of reinforcement learning algorithms. Consequently, despite the formidable challenges in training value models for complex problems, the potential benefits of overcoming these difficulties are substantial.

However, training a perfect value model in Long COT tasks presents significant challenges.
First, learning a low-bias value model is non-trivial given the long trajectory and the instability of learning value in a bootstrapped way. Second, handling both short and long responses simultaneously is also challenging, as they might exhibit very distinct preferences towards the bias-variance trade-off during optimization. Last but not least, the sparsity of the reward signal from verifiers is further exacerbated by the long CoT pattern, which intrinsically requires better mechanisms to balance exploration and exploitation. To address the aforementioned challenges and fully unleash the potential of value-model-based methods in reasoning tasks, we present \textbf{V}alue \textbf{A}ugmented proximal \textbf{P}olicy \textbf{O}ptimization (\textbf{VAPO}), a value-model-based RL training framework. VAPO draws inspiration from prior research works such as VC-PPO~\cite{vc-ppo} and DAPO~\cite{dapo}, and further extends their concepts.

We summarize our key contributions as follows:
\begin{enumerate}
\item We introduce VAPO, the first value-model-based RL training framework to outperform value-model-free methods on long COT tasks significantly. VAPO not only demonstrates remarkable superiority in terms of performance but also showcases enhanced training efficiency, streamlining the learning process and underscoring its potential as a new benchmark in the field.
\item We propose Length-adaptive GAE, which adaptively adjusts the $\lambda$ parameter in GAE computation based on response lengths. By doing so, it effectively caters to the distinct bias-variance trade-off requirements associated with responses of highly variable lengths. As a result, it optimizes the accuracy and stability of the advantage estimation process, particularly in scenarios where the length of the data sequences varies widely.
\item We systematically integrate techniques from prior work, such as Clip-Higher and Token-level Loss from DAPO \citep{dapo}, Value-Pretraining and Decoupled-GAE from VC-PPO \citep{vc-ppo}, self-imitation learning from SIL \citep{SIL}, and Group-Sampling from GRPO \citep{shao2024deepseekmath}. Additionally, we further validate their necessity through ablation studies.
\iffalse
    \item In the context of CoT reasoning, value model bias hinders the effective training of value-model-based models. To address this, we propose a \textbf{Length-Adaptive GAE} method and adopt the decoupling Generalized Advantage Estimation (GAE) and value warm-up from VC-PPO~\cite{vc-ppo} to facilitate stable value learning. Experimental results validate our hypothesis, showing that without these techniques, the model fails to achieve satisfactory performance. 
    \item  Long-time training negatively impacts the model's exploration ability. To counter this, we follow DAPO~\cite{dapo} using the CLIP, encouraging more extensive exploration in later stages of training. Experimental results also show that this adjustment leads to significant performance improvements.
\fi
\end{enumerate}
 \textbf{VAPO} is an effective reinforcement learning system that brings together these improvements. These enhancements work together smoothly, leading to a combined result that’s better than the sum of the individual parts. 
 We conduct experiments using the Qwen2.5-32B pre-trained model, ensuring no SFT data is introduced in any of the experiments, to maintain comparability with related works (DAPO and DeepSeek-R1-Zero-Qwen-32B).
 The performance of \textbf{VAPO} improves from vanilla PPO a score of 5 to 60, surpassing the previous SOTA value-model-free methods DAPO~\cite{dapo} by 10 points. More importantly, \textbf{VAPO} is highly stable — we don't observe any crashes during training, and the results across multiple runs are consistently similar.


\iffalse
In this paper, we present \textbf{V}alue \textbf{A}ugmented proximal \textbf{P}olicy \textbf{O}ptimization (\textbf{VAPO}), a value-model-based RL training framework. \textbf{VAPO} addresses the challenges of value model training in long-CoT tasks while maximizing the benefits that value models offer to RL algorithms. 
The main differences between reasoning models and traditional RLHF methods lie in two key aspects: (a) The introduction of long CoT significantly increases the sequence length, often tens of times longer than that of traditional models. (b) Since the reward are given by a verifier instead of a model, reward hacking is not a concern, allowing for much longer training, typically tens of times more steps compared to conventional approaches.
The long sequences introduce bias into the reward model, particularly due to reward discounting. A more effective reward learning approach is thus needed.
And  the longer training  leads to a collapse in exploration during later stages, necessitating enhanced exploration mechanisms.
To address this issue, we draw inspiration from prior work~\cite{vc-ppo,dapo} and propose our system approach as follows.
\fi
%We aim to explore the limits of VAPO and also conduct a systematic study of a few methods proposed in prior work.
%Concurrently, VAPO integrates established strategies from prior literature to enhance the efficiency of exploration and exploitation during RL training. 
%The specific improvement strategies are as follows:
% During the RL training phase, GRPO is widely employed and achieves favorable results in balancing model training effectiveness and training resource consumption. However, due to the absence of a value model, GRPO is unable to perform fine-grained token-level policy optimization. This limitation not only restricts the model's exploration capabilities but also reduces its training efficiency. To address this, DAPO introduces a higher clip strategy and dynamic sampling to mitigate issues of insufficient exploration and low training efficiency. Nevertheless, these approaches cannot fundamentally resolve the core problems inherent in GRPO. In contrast, a more promising alternative is to adopt PPO, as exemplified in OpenAI's o series models. However, applying the PPO algorithm to long COT generation scenarios is non-trivial, primarily facing two key challenges: 1) How to efficiently train a value model to enable accurate predictions on long text sequences; 2) How to balance exploration and exploitation to improve optimization efficiency, which is particularly crucial in scenarios with high resource consumption for long COT models.

% In this paper, we propose Value Augmented proximal Policy Optimization (VAPO), a novel RL framework designed to address the aforementioned challenges of PPO in long CoT tasks. VAPO enhances the standard PPO framework through seven key technical innovations:  

%\begin{enumerate}
%\item \textbf{Value Model Pretraining}, a dedicated pretraining stage for the value model to rectify estimation biases inherent in value initialization.  
%\item \textbf{Decoupled Generalized Advantage Estimation (Decoupled-GAE)}, which reduces advantage estimation variance while maintaining unbiased optimization objectives for the value model.  
%\item \textbf{Adaptive GAE}, which dynamically adjusts the $\lambda$ parameter in GAE based on sequence length during policy optimization, mitigating reward signal decay in long sequences.  
%\item \textbf{Clip-Higher Strategy}, which expands the policy optimization scope to include more tokens with lower sampling probability to enhance model exploration.  
%\item \textbf{Token-Level Policy Gradient Loss}, which ensures balanced gradient contributions from each token in long sequences, strengthening the suppression of erroneous reasoning patterns.  
%\item \textbf{Positive Example LM Loss}, which improves the exploitation efficiency of high-quality responses.
%\item \textbf{Rebalance Generation Budget}, which reduces the number of prompts per generation run while increasing the number of responses generated per prompt. Additionally, incorporating more detailed contrastive signals enhances the granularity of comparisons.
%\end{enumerate} 


%Through the aforementioned enhancements, VAPO dramatically enhances training stability, evidenced by the steady evolution of generated sequence lengths and consistent improvement in test set metrics throughout the training process. Furthermore, VAPO demonstrated significantly superior training efficiency compared to DAPO. Under the same learning rate, VAPO achieved superior performance using only 70\% of the training steps required by DAPO. On the AIME24, VAPO achieves 53.3, significantly outperforming DAPO (46.7\footnote{DAPO \cite{dapo} achieved 50.0 on a new training set, while we conduct comparisons on the old training set}) and the original PPO baseline (???). And our ablation studies demonstrate that all of the six techniques contribute to the improvement of model performance.
% \input{sections/introduction_new}
\section{Introduction}
Large Language Models (LLMs) have revolutionized AI-driven applications by enabling natural language understanding and generation at an unprecedented scale. However, these models often suffer from issues such as outdated responses and hallucinations, which severely hinder the accuracy of information generation. 
Retrieval-Augmented Generation (RAG) has emerged as a promising framework that integrates retrieved information from external corpora, such as external APIs~\cite{google,bing}, scientific repositories~\cite{arxiv,pubmed} or domain-specific databases~\cite{amazon_dataset, espn_dataset}, ensuring more knowledge-grounded and up-to-date outputs. 

Its versatility has led to significant applications across various domains, including question answering~\cite{siriwardhana2023improving}, enterprise search~\cite{bulfamante2023generative} and healthcare~\cite{wu2024medical}, etc. Among these applications, one particularly notable area is in agent workflows, where RAG enhances autonomous systems by providing context-aware, dynamically retrieved, and reliable knowledge. This is because each stage of the RAG process closely mirrors key aspects of an agent’s workflow, as shown in Figure~\ref{fig:structure}. For instance, the query rewriting phase in RAG, which involves semantic understanding and parsing, aligns with the semantic comprehension stage in agent workflows. Likewise, RAG’s retrieval phase, which focuses on extracting the most relevant documents, corresponds to the planning and execution phases of an agent, where decisions are made based on retrieved knowledge. Finally, the generation phase in RAG parallels an agent’s execution stage, where actions are performed based on the given task. This structural alignment suggests that the architecture of RAG is fundamentally converging with agent workflows, solidifying its position as a key facilitator of intelligent and autonomous systems.

Although the structural alignment between RAG and agent workflows highlights their deepening convergence, a critical next step in enhancing these intelligent systems lies in personalization. Personalization is a key driver toward achieving more adaptive and context-aware AI, which is fundamental for the progression toward Artificial General Intelligence (AGI). It plays an essential role in applications such as personalized reasoning~\cite{xu2021transformer, henze2004reasoning}, adaptive decision-making~\cite{lu2011budgeted}, user-specific content generation~\cite{xu2025personalized, shaker2010towards}, and interactive AI systems~\cite{ma2021one, qian2021learning}. However, existing research lacks a comprehensive comparative analysis of personalized RAG and agentic approaches. Current surveys primarily focus on general RAG methodologies~\cite{gao2023retrieval, fan2024survey} or agent-related literature~\cite{li2024personal, wang2024survey, zhang2024survey}, without systematically exploring their implications for personalization. While recent works such as~\cite{zhang2024personalization, liu2025survey} discuss personalization, they predominantly address personalized generation within LLMs or specific downstream tasks, overlooking how personalization can be effectively integrated into RAG and agent workflows.

Motivated by the above issues, this survey aims to provide a comprehensive review of the integration of personalization into RAG and agentic RAG frameworks to enhance user experiences and optimize satisfaction. The key contributions of this work can be summarized as follows:
\begin{itemize}[leftmargin=*] 
\item We provide an extensive exploration of the existing literature on how personalization is integrated into various stages of RAG (pre-retrieval, retrieval, and generation) and agentic RAG (understanding, planning, execution, and generation).
\item We summarize the key datasets, benchmarks, and evaluation metrics used in existing research for each subtask to facilitate future studies in the respective domains. 
\item We also highlight the limitations of current research and suggest future directions for personalized RAG, emphasizing potential advancements to address existing challenges.
\end{itemize}


The outline of this survey is as follows: we introduce what is personalization (Sec.~\ref{sec:what}) and explain how personalization is adopted into RAG pipeline (Sec.~\ref{sec:how}). Then, we present a literature review on where to integrate personalization within different stages of RAG and agentic RAG workflows~(Sec.~\ref{sec:where}) and discuss the key datasets and evaluation metrics used in existing research~(Sec.\ref{sec:evaluation&dataset}). Lastly, we present a discussion on the limitations of current research and future directions~(Sec.~\ref{sec:futuredirection}).

\begin{figure}[t]
    \centering
    \includegraphics[width = 0.6\linewidth]{figures/structure.pdf}
    \caption{Correlation between personalization and RAG with agent flow.}
    \label{fig:structure}
\end{figure}

\begin{table*}[t]
% \setlength\tabcolsep{3pt}  %可以控制列间距
% \renewcommand{\arraystretch}{1} %可以控制行间距
\caption{Overview of Personalized RAG and Agent.}
\centering
\resizebox{\textwidth}{!}{
\begin{tabular}{c|c|c|c} 
\toprule
\textbf{Field}                              & \textbf{Sub-field}                                                                                       & \textbf{Subsub-field}                                                                          & \textbf{Papers}                                                                                                                                                                                                                                                                                                                                                                                                                                                                                                                                                                                                                                                                                                                                                                                                                                            \\ 
\midrule
\multirow{7}{*}{\textbf{Pre-retrieval}}     & \multirow{4}{*}{\begin{tabular}[c]{@{}c@{}}Query\\Rewriting\end{tabular}}                       & \begin{tabular}[c]{@{}c@{}}Learning to\\Personalized Query Rewrite\end{tabular}          & CLE-QR \cite{li2022query}, CGF \cite{hao2022cgf}, PEARL \cite{mysore2023pearl}                                                                                                                                                                                                                                                                                                                                                                                                                                                                                                                                                                                                                                                                                                                 \\ 
\cmidrule{3-4}
                                   &                                                                                                 & \begin{tabular}[c]{@{}c@{}}LLM to\\Personalized Query Rewrite\end{tabular} & Least-to-Most Prompting \cite{zhou2022least}, ERAGent \cite{shi2024eragent}, CoPS \cite{zhou2024cognitive}, Agent4Ranking \cite{li2023agent4ranking}, FIG \cite{chen2023graph}, BASES \cite{ren2024bases}                                                                                                                                                                                                                                                                                                                                                                                                                                                                                                                                  \\ 
\cmidrule{2-4}
                                   & \multirow{3}{*}{\begin{tabular}[c]{@{}c@{}}Query\\Expansion\end{tabular}}                       & \begin{tabular}[c]{@{}c@{}}Tagging-based query\\expansion\end{tabular}                & Gossple~\cite{bertier2009toward}, ~\citet{biancalana2009social}, SoQuES~\cite{bouadjenek2011personalized}, ~\citet{zhou2012improving}                                                                                                                                                                                                                                                                                                                                                                                                                                                                                                                                                                         \\ 
\cmidrule{3-4}
                                   &                                                                                                 & Else                                                                                  & ~\citet{lin2006personalized}, ~\citet{bender2008exploiting}, Axiomatic PQEC~\cite{mulhem2016axiomatic}, WE-LM~\cite{wu2017personalized}, PSQE~\cite{bouadjenek2019personalized}, PQEWC~\cite{bassani2023personalized}                                                                                                                                                                                                                                                                                                                                                                                                                       \\ 
\cmidrule{2-4}
                                   & \multicolumn{2}{c|}{Others}                                                                                                                                                             & Bobo~\cite{gao2010utilizing}, ~\citet{kannadasan2019personalized}, PSQE~\cite{baumann2024psqe}                                                                                                                                                                                                                                                                                                                                                                                                                                                                                                                                                                                                                                                 \\ 
\midrule
\multirow{9}{*}{\textbf{Retrieval}}         & \multicolumn{2}{c|}{Indexing}                                                                                                                                                           & PEARL~\cite{mysore2023pearl}, KG-Retriever~\cite{chen2024kg}, EMG-RAG~\cite{wang2024crafting}, PGraphRAG~\cite{au2025personalized}                                                                                                                                                                                                                                                                                                                                                                                                                                                                                                                                                                                                                                                                                                                                                                                                     \\ 
\cmidrule{2-4}
                                   & \multirow{7}{*}{Retrieval}                                                                       & \begin{tabular}[c]{@{}c@{}}Dense\\Retrieval\end{tabular}                               & \begin{tabular}[c]{@{}c@{}}MeMemo \cite{wang2024mememo}, RECAP \cite{liu2023recap}, LAPDOG \cite{huang2024learning}, \citet{gu2021partner}, PersonaLM \cite{mathur2023personalm}, UIA \cite{zeng2023personalized}, XPERT \cite{vemuri2023personalized}, DPSR \cite{zhang2020towards}, \\RTM \cite{bi2021learning}, Pearl \cite{mysore2023pearl}, MemPrompt \cite{madaan2022memory}, ERRA \cite{cheng2023explainable}, MALP \cite{zhang2023llm}, USER-LLM \cite{ning2024user}, PER-PCS \cite{tan2024personalized}\end{tabular}                                                      \\ 
\cmidrule{3-4}
                                   &                                                                                                 & \begin{tabular}[c]{@{}c@{}}Sparse\\Retrieval\end{tabular}                              & OPPU \cite{tan2024democratizing}, PAG \cite{richardson2023integrating}, \citet{au2025personalized}, UniMS-RAG \cite{wang2024unims}, \citet{deng2022toward},                                                                                                                                                                                                                                                                                                                                                                                                                                                                                                                                                                                                  \\ 
\cmidrule{3-4}
                                   &                                                                                                 & \begin{tabular}[c]{@{}c@{}}Prompt-based\\Retrieval\end{tabular}                        & LAPS \cite{joko2024doing}, UniMP \cite{wei2024towards}, \citet{shen2024heart}                                                                                                                                                                                                                                                                                                                                                                                                                                                                                                                                                                                                                                                                                                                  \\ 
\cmidrule{3-4}
                                   &                                                                                                 & Others                                                                                & \citet{salemi2024optimization}, PersonalTM \cite{lian2023personaltm}, \citet{zhang2024personalized}                                                                                                                                                                                                                                                                                                                                                                                                                                                                                                                                                                                                                                                                                            \\ 
\cmidrule{2-4}
                                   & \multicolumn{2}{c|}{Post-retrieval}                                                                                                                                                      & PersonaRAG~\cite{zerhoudi2024personarag}, \citet{pavliukevich2024improving}, UniMS-RAG~\cite{wang2024unims}, \citet{salemi2024learning}, \citet{zhang2025rehearse}, AutoCompressors~\cite{chevalier2023adapting}, FIT-RAG~\cite{mao2024fit}                                                                                                                                                                                                                                                                                                                                                                                                                                                                                                                                                                                                                                                                                                                                                                \\ 
\midrule
\multirow{11}{*}{\textbf{Generation}}        & \multirow{6}{*}{\begin{tabular}[c]{@{}c@{}}Generation from\\Explicit Preferences\end{tabular}}  & \begin{tabular}[c]{@{}c@{}}Direct\\Prompting\end{tabular}                             & P$^2$~\cite{jiang2023evaluating}, Character Profiling~\cite{yuan2024evaluating}  OpinionQA~\cite{santurkar2023whose}, ~\citet{kang2023llms}, ~\citet{liu2023chatgpt}, Cue-CoT~\cite{wang2023cue}, TICL~\cite{cho2025tuning}                                                                                                                                                                                                                                                                                                                                                                           \\ 
\cmidrule{3-4}
                                   &                                                                                                 & \begin{tabular}[c]{@{}c@{}}Profile-Augmented\\Prompting\end{tabular}                  & GPG~\cite{zhang2024guided}, ~\citet{richardson2023integrating}, ONCE~\cite{liu2024once}, LLMTreeRec~\cite{zhang2025llmtreerec}, KAR~\cite{xi2024towards}, Matryoshka~\cite{li2024matryoshka}                                                                                                                                                                                                                                                                                                                                                                                                                                                \\ 
\cmidrule{3-4}
                                   &                                                                                                 & \begin{tabular}[c]{@{}c@{}}Personalized-Prompt\\Prompting\end{tabular}                & \citet{li2024learning}, RecGPT~\cite{zhang2024recgpt}, PEPLER-D~\cite{li2023personalized}, GRAPA~\cite{qu2024graph}, SGPT~\cite{deng2024unlocking}, PFCL~\cite{yu2024personalized}                                                                                                                                                                                                                                                                                                                                                                                                                                                                          \\ 
\cmidrule{2-4}
                                   & \multirow{4}{*}{\begin{tabular}[c]{@{}c@{}}Generation from \\Implicit Preferences\end{tabular}} & \begin{tabular}[c]{@{}c@{}}Fine-tuning-Based\\Methods\end{tabular}                    & \begin{tabular}[c]{@{}c@{}}PLoRA~\cite{zhang2024personalized}, LM-P~\cite{wozniak2024personalized}, MiLP~\cite{zhang2024personalized}, OPPU~\cite{tan2025democratizing}, PER-PCS~\cite{tan2024personalized}, Review-LLM~\cite{peng2024reviewllm},\\UserIdentifier~\cite{mireshghallah2021useridentifier}, UserAdapter~\cite{zhong2021useradapter}, HYDRA~\cite{zhuang2406hydra}, PocketLLM~\cite{peng2024pocketllm}, CoGenesis~\cite{zhang2024cogenesis}\end{tabular}  \\ 
\cmidrule{3-4}
                                   &                                                                                                 & \begin{tabular}[c]{@{}c@{}}Reinforcement\\Learning-Based\\Methods\end{tabular}        & P-RLHF~\cite{li2024personalized}, P-SOUPS~\cite{jang2023personalized}, PAD~\cite{chen2024pad}, REST-PG~\cite{salemi2025reasoning}, \citet{salemi2024optimization}, RewriterSlRl~\cite{li2024learning},\citet{kulkarni2024reinforcement}                                                                                                                                                                                                                                                                                                                                                                                                    \\ 
\midrule
\multirow{13}{*}{\textbf{From RAG to Agent}} & \multirow{6}{*}{\begin{tabular}[c]{@{}c@{}}Personalized\\Understanding\end{tabular}}            & \begin{tabular}[c]{@{}c@{}}In user-profile\\understanding\end{tabular}                & \citet{xu2024penetrative}, \citet{abbasian2023conversational},                                                                                                                                                                                                                                                                                                                                                                                                                                                                                                                                                                                                                                                                                                                                                  \\ 
\cmidrule{3-4}
                                   &                                                                                                 & \begin{tabular}[c]{@{}c@{}}In agent’s role\\understanding\end{tabular}                & RoleLLM~\cite{wang2023rolellm}, Character-LLM~\cite{shao2023character}, \citet{wang2023incharacter},                                                                                                                                                                                                                                                                                                                                                                                                                                                                                                                                                                                                                                                           \\ 
\cmidrule{3-4}
                                   &                                                                                                 & \begin{tabular}[c]{@{}c@{}}In agent’s user-role\\joint understanding\end{tabular}     & SocialBench \cite{chen2024socialbench}, \citet{dai2024mmrole}, \citet{ran2024capturing}, \citet{wang2023enabling}, \citet{tu2024charactereval}, Neeko \cite{yu2024neeko}                                                                                                                                                                                                                                                                                                                                                                                                                                                                                                                                                                    \\ 
\cmidrule{2-4}
                                   & \multirow{2}{*}{\begin{tabular}[c]{@{}c@{}}Personalized Planning\\and Execution\end{tabular}}   & \begin{tabular}[c]{@{}c@{}}Memory\\Management\end{tabular}                            & EMG-RAG \cite{wang2024crafting}, \citet{park2023generative}, \citet{abbasian2023conversational}, RecAgent \cite{wang2023user}, TravelPlanner+ \cite{singh2024personal}, PersonalWAB \cite{cai2025large}, VOYAGER \cite{wangvoyager}, MemoeryLLM \cite{wangmemoryllm}                                                                                                                                                                                                                                                                                                                                                                                                                                                                                                                                                                                      \\ 
\cmidrule{3-4}
                                   &                                                                                                 & Tool and API Calling                                                                  & VOYAGER \cite{wangvoyager}, \citet{zhangbootstrap}, PUMA \cite{cai2025large}, \citet{wang2023enabling}, PenetrativeAI \cite{xu2024penetrative}, \citet{huang2022language}, \cite{park2023generative}, MetaGPT \cite{hong2023metagpt}, OKR-Agent \cite{zheng2023agents}                                                                                                                                                                                                                                                                                                                                                                                                                                                                                                                                                                                                                                                                        \\ 
\cmidrule{2-4}
                                   & \multirow{4}{*}{\begin{tabular}[c]{@{}c@{}}Personalized\\Generation\end{tabular}}               & \begin{tabular}[c]{@{}c@{}}Alignment with \\User Fact\end{tabular}                   & Character-LLM \cite{shao2023character}, \citet{wang2024investigating}, \citet{dai2024mmrole}                                                                                                                                                                                                                                                                                                                                                                                                                                                                                                                                                                                                                                                                                                   \\ 
\cmidrule{3-4}
                                   &                                                                                                 & \begin{tabular}[c]{@{}c@{}}Alignment with User\\Preferences\end{tabular}              & \citet{wang2023rolellm}, \citet{ran2024capturing}, \citet{wang2023incharacter}, \citet{chen2024socialbench}                                                                                                                                                                                                                                                                                                                                                                                                                                                                                                                                                                                                                                                                   \\
\bottomrule
\end{tabular}}
\end{table*}

\section{What is Personalization} \label{sec:what}
Personalization in current research refers to the tailoring of model predictions or generated content to align with an individual's preferences. In the context of RAG and agents, personalization involves incorporating user-specific information at various stages of the RAG pipeline or within agents. User personalization can be categorized into the following types:

\begin{itemize}[leftmargin=*] 
\item Explicit User Profile: Explicitly presented user information, including biographical details, attributes (\eg age, location, gender, education), and social connections (\eg social networks).
\item User Historical Interactions: Behavioral data, including browsing history, clicks, and purchases, which help infer user interests and preferences to improve personalization. 
\item User Historical Content: Implicit personalization derived from user-generated content, such as chat history, emails, reviews, and social media interactions. 
\item Persona-Based User Simulation: The use of LLM-based agents to simulate and generate personalized interactions.
\end{itemize}

Integrating this personalized information at various stages of the RAG and agent workflows enables dynamic alignment with human preferences, thereby making responses more user-centric and adaptive.
\section{How to Adopt Personalization} \label{sec:how}
We define the process of introducing personalization within the RAG pipeline as follows:
% \vspace{-1mm}
\begin{equation}\label{equ:definition}
g = \mathcal{G} \left( \mathcal{R}\left(\mathcal{Q}\left(q,p\right),\mathcal{C},p\right),\text{prompt},p,\theta \right)
\end{equation}
% \vspace{-1mm}
where $p$ denotes personalized information, and the process unfolds in three steps. In the \textbf{pre-retrieval phase}, query processing ($\mathcal{Q}$) refines the query $q$ using personalized information, such as through query rewriting or expansion. During the \textbf{retrieval phase}, the retriever ($\mathcal{R}$) leverages $p$ to fetch relevant documents from the corpus ($\mathcal{C}$). Finally, in the \textbf{generation phase}, the retrieved information, combined with $p$ and structured using the given prompt, id fed into the generator ($\mathcal{G}$) with parameter $\theta$ to produce the final response $g$. It is evident that personalized information directly influences multiple stages of the RAG pipeline. In this survey, we consider the agent system as a specialized application of the RAG framework, where personalization is incorporated in a manner similar to the RAG framework.
\section{Where to Adopt Personalization} \label{sec:where}

\subsection{Pre-retrieval} \label{sec:Pre-retrieval}

\subsubsection{\textbf{Definition}} Pre-retrieval is a crucial step in information retrieval systems, where the original user query is enhanced or modified before the retrieval process to improve the relevance and quality of the search results, as shown in Figure~\ref{fig:pre_rag}. This process often incorporates additional contextual or personalized information to better align the query with the user's intent. The process can be formalized as follows:
\begin{equation}
    q^{*} = \mathcal{Q}\left(q,p\right)
\end{equation}
where $p$ and $q$ denote the personalized information and original query, and $q^{*}$ is the optimized query after query reformulation.

\begin{figure}[t]
    \centering
    \includegraphics[width=1\linewidth]{figures/pre_rag.pdf}
    \caption{Overview of the personalized pre-retrieval stage.}
    \label{fig:pre_rag}
\end{figure}

\subsubsection{\textbf{Query Rewriting}}
Query rewriting in RAG at the pre-retrieval stage refers to the process of reformulating user queries to enhance retrieval effectiveness by improving relevance, disambiguating intent, or incorporating contextual information before retrieving documents from an external knowledge source.
The literature on personalized query rewriting can be broadly classified into two primary categories: (1) Direct Personalized Query Rewriting and (2) Auxiliary Personalized Query Rewriting.

\paragraph{\textbf{\textit{{(1). Direct Personalized Query Rewriting.}}}}

The first category focuses on personalized query rewriting by using direct models. 
For example, \citet{cho2021personalized} 
presents a personalized search-based query rewrite system for conversational AI that addresses user-specific semantic and phonetic errors. 
\citet{nguyen2025rl} apply reinforcement learning techniques to improve query rewriting in online e-commerce systems, leveraging distilled LLMs for personalized performance. 
CLE-QR~\cite{li2022query} explores query rewriting in Taobao's search engine to enhance user satisfaction through customized query adaptation. 
CGF~\cite{hao2022cgf} introduces a constrained generation framework that allows for more flexible and personalized query rewriting in conversational AI. 
\citet{li2024learning} investigate learning methods to rewrite prompts for personalized text generation, improving the relevance and engagement of AI-generated content. 
Additionally, PEARL~\cite{mysore2023pearl} discusses personalizing large language model-based writing assistants through the integration of generation-calibrated retrievers, enhancing AI-generated content.

\paragraph{\textbf{\textit{{(2). Auxiliary Personalized Query Rewriting.}}}} The second category emphasizes personalized query rewriting by using auxiliary mechanisms, such as retrieval, reasoning strategies, and external memory. 
\citet{zhou2022least} propose a least-to-most prompting strategy that aids in complex reasoning within LLMs, which can be adapted for personalized text generation.
ERAGent~\cite{shi2024eragent} enhance retrieval-augmented LLMs to improve personalization, efficiency, and accuracy, indirectly supporting personalized query rewriting for content generation. 
CoPS~\cite{zhou2024cognitive} integrate LLMs with memory mechanisms to create more personalized search experiences, which also influences content generation through better query understanding.
Further,  Agent4Ranking~\cite{li2023agent4ranking} employs multi-agent LLMs to perform semantic robust ranking, including personalized query rewriting to improve search rankings. 
FIG~\cite{chen2023graph} combine graph-based methods with LLMs to query rewrite, improving personalized content generation and conversational interactions. 
Lastly, BASES~\cite{ren2024bases} employ LLM-based agents to simulate large-scale web search user interactions, contributing to the development of personalized query rewriting strategies for content generation.


\subsubsection{\textbf{Query Expansion}}
Query expansion enhances retrieval systems by expanding a user’s original query with additional terms, synonyms, or refined structure to better capture intent. 
% \yc{the description here should be distinguished from that of query rewriting.} 
This improves the relevance and scope of retrieved documents. Recent advancements in LLMs have reinvigorated this field~\cite{wang2023query2doc,jagerman2023query,jia2024mill}, leveraging their comprehension and generation abilities to expand queries using encoded knowledge or external retrieval, with notable success. Personalized query expansion, a subset, incorporates user-specific data to tailor results, boosting performance and customizing the search experience.


\paragraph{\textbf{\textit{{(1). Tagging-based Query Expansion.}}}}

By 2009, studies began incorporating tagging information to enhance personalized query expansion. For instance, Gossple~\cite{bertier2009toward} introduced the TagMap and TagRank algorithms, which dynamically selected tags from personalized networks constructed using the cosine similarity of user-item tag distances, improving recall performance. Similarly,~\citet{biancalana2009social} recorded user queries and visited URLs, leveraging social bookmarking to extract relevant tags and build a personalized three-dimensional co-occurrence matrix. Based on this, multiple semantically categorized expanded queries were generated to better reflect user interests. Further advancements include SoQuES~\cite{bouadjenek2011personalized}, which integrated tag semantic similarity with social proximity, and a graph-based approach~\cite{zhou2012improving} that utilized Tag-Topic models and pseudo-relevance feedback for term weighting, tailoring the expansion process to individual user preferences.


\paragraph{\textbf{\textit{{(2). Else.}}}}
Apart from tagging-based techniques, early research on Personalized Query Expansion primarily focused on modeling user personalization based on search history~\cite{lin2006personalized}, social networks, or preferences derived from friendship networks~\cite{bender2008exploiting}. The Axiomatic PQEC framework~\cite{mulhem2016axiomatic} formalized expansion rules using both local (user behavior-driven) and social (network-driven) strategies. In 2017, WE-LM~\cite{wu2017personalized} advanced this paradigm by modeling multi-relational networks with word embeddings across tag-word relationships, refining associations through affinity graphs. Later, PSQE~\cite{bouadjenek2019personalized} further improved tagging-based methods using utf-iuf user profiling, integrating a tag similarity graph with user profiles in the online phase to compute expansion terms relevant to user interests in real-time, achieving dynamic personalized expansion. In addition, PQEWC~\cite{bassani2023personalized} leveraged clustering and contextual word embeddings to optimize query expansions dynamically.

\subsubsection{\textbf{Others}}

% \yc{Directly Besides is ok, rm subsection labels.}
Besides query rewriting and query expansion, other personalized query-related research focuses on areas like query disambiguation and query auto-completion~\cite{song2024survey}. Bobo~\cite{gao2010utilizing} allows users to input contextual terms reflecting their domain knowledge. In 2019, a method~\cite{kannadasan2019personalized} applied fastText embeddings from recent queries to rank candidates. In addition, PSQE~\cite{baumann2024psqe} employed synthetic user profiles from Wikipedia and word2vec embeddings for query disambiguation.

\subsubsection{\textbf{Discussion}}
While both query rewriting and query expansion aim to align user input with system understanding to enhance retrieval quality, their roles in personalization differ in fundamental ways.Understanding the distinct operational characteristics and application scenarios of each technique is essential for designing effective personalized retrieval systems. The key takeaways are listed as follows: 
\begin{itemize}[leftmargin=*] 
    \item Query rewriting is most beneficial when the original query is \textbf{ambiguous}, underspecified, or misaligned with retrieval intents, particularly in conversational or multi-turn settings. 
    \item Query expansion is most effective when the original query is \textbf{relevant} but incomplete — i.e., when it needs to be semantically broadened to cover additional relevant concepts.
\end{itemize}


\subsection{Retrieval} \label{sec:Retrieval}
\begin{figure}[t]
    \centering
    \includegraphics[width=\linewidth]{figures/retrieval.pdf}
    \caption{Overview of the personalized retrieval stage.}
    \label{fig:retrieval}
\end{figure}
\subsubsection{\textbf{Definition}}

The retrieval process involves finding the most relevant documents $D^*$ from a corpus $\mathcal{C}$ based on a query $q^*$, as shown in Figure~\ref{fig:retrieval}. To incorporate personalization, additional user-specific information $p$ is integrated into the retrieval function  $\mathcal{R}$. This allows the retrieval process to tailor the selected documents to align with individual user preferences or contexts, thereby enhancing the relevance and personalization of the generated outputs.
\begin{equation}
D^* =  \mathcal{R}\left(q^*,\mathcal{C},p\right)
\end{equation}

In the retrieval process, personalization can primarily be introduced by focusing on three steps: indexing, retrieval, and post-retrieval. These steps ensure efficient and accurate retrieval of relevant documents or knowledge, while tailoring the process to individual user preferences. Below, we provide a detailed explanation of each step.

\subsubsection{\textbf{Indexing}}
Indexing organizes knowledge base data into a structured format to facilitate efficient retrieval. Within the RAG pipeline, documents are either chunked or entirely encoded into representations before being integrated into searchable systems~\cite{douze2024faiss, annoy}. Conventional encoding methods employ either sparse encoding techniques (\eg TF-IDF~\cite{rajaraman2011mining}, BM25~\cite{robertson2009probabilistic}) or dense encoding approaches leveraging pre-trained models, such as BERT~\cite{koroteev2021bert}, Siamese Encoders~\cite{reimers2019sentence}, or LLM-based encoders~\cite{li2023towards, xiao2024c}.

To introduce personalization at the indexing stage, PEARL~\cite{mysore2023pearl} generates user embeddings by encoding personal history data with models like DeBERTa. These embeddings are subsequently clustered to create personalized shared indices. Other approaches integrate knowledge graphs into indexing to enhance retrieval performance. For example, KG-Retriever~\cite{chen2024kg} employs a Hierarchical Index Graph, consisting of a knowledge graph layer and a collaborative document layer, to improve RAG retrieval. EMG-RAG~\cite{wang2024crafting} incorporates personalized memory within an editable knowledge graph, enabling dynamic retrieval. Similarly, PGraphRAG~\cite{au2025personalized} leverages user-centric knowledge graphs to enhance personalization in retrieval tasks.


\subsubsection{\textbf{Retrieval}}
The Retrieval step matches a user query with the indexed knowledge base to fetch relevant candidates. It can be broadly categorized into four different types: (1) Dense Retrieval,  (2) Sparse Retrieval, (3) Prompt-based Retrieval, and  (4) Others.

\paragraph{\textbf{\textit{{(1). Dense Retrieval.}}}}
Dense retrieval methods often use vector embeddings and similarity metrics (\eg cosine similarity) and achieve personalization by encoding user preferences, context, or interactions into query or document embeddings, enabling tailored results through similarity-based matching. For instance, MeMemo \cite{wang2024mememo} retrieves personalized information by matching user-specific embeddings with document vectors, focusing on private, on-device text generation. Similarly, RECAP \cite{liu2023recap} and LAPDOG \cite{huang2024learning} enhance personalized dialogue generation by encoding queries and user profiles as dense vectors and retrieving top-N results, ensuring user-specific context drives the responses. In chatbots, \citet{gu2021partner} integrates conversational context and user profiles to align retrieved responses with user personas. PersonaLM \cite{mathur2023personalm} employs group-wise contrastive learning, training its retrieval model to align user queries with domain-specific text fragments, thereby improving personalization. UIA \cite{zeng2023personalized} employs dual encoders to retrieve documents tailored to user preferences. XPERT \cite{vemuri2023personalized} incorporates temporal events and user interactions into embeddings, enabling large-scale retrieval across millions of items.

Dense retrieval also enhances specific applications like e-commerce, medical assistance, and language models. DPSR \cite{zhang2020towards} and RTM \cite{bi2021learning} encode user queries and product information to personalize product searches dynamically. Pearl \cite{mysore2023pearl} and MemPrompt \cite{madaan2022memory} retrieve personalized content by leveraging historical user data and memory-assisted mechanisms. ERRA \cite{cheng2023explainable} uses review embeddings as dense queries for recommendations. In medical assistance, MALP \cite{zhang2023llm} and USER-LLM \cite{ning2024user} integrate short- and long-term user interactions into embeddings for contextualized, personalized responses. Finally, PER-PCS \cite{tan2024personalized} retrieves relevant information using individual user histories, enhancing the personalization capabilities of large language models.  

\paragraph{\textbf{\textit{{(2). Sparse Retrieval.}}}}
Sparse retrieval methods often rely on term-based matching (\eg BM25) and apply personalization by assigning higher weights to terms or keywords that are more relevant to the user.  OPPU \cite{tan2024democratizing} uses the BM25 algorithm to select the k most relevant records from the user's historical data for the current query. Similarly, PAG \cite{richardson2023integrating} incorporates user input and profiles to enhance summarization and retrieval, aligning sparse representations with personalization objectives for large language models. \citet{au2025personalized} uses BM25 search algorithms to find entries related to the target user or neighboring users through the graph structure. UniMS-RAG \cite{wang2024unims} combines sparse and dense retrieval by leveraging multi-source knowledge, such as dialogue context and user images, to refine personalized responses in dialogue systems. Lastly, \citet{deng2022toward} apply sparse retrieval to support fact-based queries, considering user queries and preferences to enhance answer generation for e-commerce applications. 

\paragraph{\textbf{\textit{{(3). Prompt-based Retrieval.}}}}
Prompt-based retrieval leverages prompts to guide retrieval from the model or external sources and introduces personalization by crafting user-specific prompts that guide the retrieval process. These prompts may include explicit user preferences, historical interactions, or detailed instructions that reflect the user’s unique requirements. By embedding this personalized context directly into the prompt, the retrieval process can dynamically adjust to capture and return results that are most relevant to the user. LAPS \cite{joko2024doing} focuses on multi-session conversational search by storing user preferences and dialogue context, then using prompts to retrieve relevant information tailored to the user's biases and categories of interest. UniMP \cite{wei2024towards} employs user interaction histories as input to prompt-based retrieval, enabling personalized recommendations for multi-modal tasks, such as vision-language applications, by aligning prompts with user behavioral data. In contrast, \citet{shen2024heart} explores the use of LLMs to extract empathy and narrative styles from user-provided stories, but this work primarily focuses on style extraction and does not explicitly involve a retrieval component. 

\paragraph{\textbf{\textit{{(4). Others.}}}}
Reinforcement learning-based retrieval personalizes the process by optimizing retrieval policies based on user feedback, learning user preferences over time to adjust strategies. \citet{salemi2024optimization} combines models like BM25, RbR, and dense retrieval, refining them with reinforcement learning (RL) and knowledge distillation (KD) to adapt to user profiles for personalized outputs. Parameter-based retrieval leverages pre-trained model parameters to implicitly store and retrieve user-specific information, allowing direct retrieval from the model without traditional indices. PersonalTM \cite{lian2023personaltm} generates document identifiers (Document IDs) using a Transformer model, encoding query, history, and document relationships into its parameters for personalization. Similarly, \citet{zhang2024personalized} uses parameterized representations to integrate user queries and histories, tailoring responses to individual preferences.

\subsubsection{\textbf{Post-retrieval}}
Current Post-Retrieval methods primarily focus on refining retrieved documents or responses to improve relevance and coherence, current methodologies could be categorized into three parts (1) Re-ranking, (2) Summarization, and (3) Compression.

\paragraph{\textbf{\textit{{(1). Re-ranking.}}}}
Re-ranking enhances personalized content generation by prioritizing more relevant documents at the top. PersonaRAG~\cite{zerhoudi2024personarag} extends RAG by integrating user-centric agents, such as the Live Session Agent and the Document Ranking Agent, to refine document ranking and improve overall performance. \citet{pavliukevich2024improving} propose a cross-encoder BERT model for re-ranking external knowledge within a personalized context. UniMS-RAG~\cite{wang2024unims} introduces a scoring mechanism that evaluates retrieved documents and outputs by optimizing the retriever. Besides, it includes an evidence attention mask, enabling re-ranking during inference and applying it to personalized datasets. ~\citet{salemi2024learning} present an iterative approach to optimizing ranking results based on the expectation-maximization algorithm, with performance validated in personalized scenarios.


\paragraph{\textbf{\textit{{(2). Summarization.}}}}
Summarization refers to the process of summarizing retrieved information to enhance performance. For instance,~\citet{zhang2025rehearse} introduced a role-playing agent system to summarize retrieved history in order to improve the final Personalized Opinion Summarization process.

\paragraph{\textbf{\textit{{(3). Compression.}}}}
Compression involves condensing embeddings or retrieved content to enhance efficiency and effectiveness. Approaches like AutoCompressor~\cite{chevalier2023adapting} compress contextual embeddings into shorter semantic representations, and FIT-RAG~\cite{mao2024fit} introduces a self-knowledge recognizer along with a sub-document-level token reduction mechanism to minimize tokens within RAG pipeline. However, few studies have specifically explored personalized fields, highlighting a promising direction for future research.

\subsubsection{\textbf{Discussion}}
Indexing, retrieval, and post-retrieval methods each play a critical role in ensuring efficient and personalized information processing, with specific applications and trade-offs. Indexing focuses on organizing knowledge bases for efficient retrieval, using techniques such as sparse encoding methods like TF-IDF and BM25, which are efficient but limited in understanding semantics, and dense encoding methods like BERT and DeBERTa, which provide better semantic understanding but require significant computational resources. These methods are widely used in tasks like question answering and personalized recommendation systems. Retrieval involves matching user queries with relevant documents and can be categorized into dense retrieval, which provides high semantic understanding and personalization but is computationally expensive; sparse retrieval, which is efficient and interpretable but less capable of handling semantics; prompt-based retrieval, which is highly flexible and adaptable to user needs but requires careful engineering of prompts; and advanced methods like reinforcement learning-based approaches, which dynamically adapt to user feedback but are complex to implement. This step is essential in applications like personalized dialogue systems, search engines, and e-commerce. Post-retrieval methods refine retrieved results to enhance relevance and coherence through re-ranking, which improves personalization and prioritizes relevant content but increases computational overhead; summarization, which simplifies complex information for better user understanding but risks losing critical details; and compression, which reduces computational costs by condensing information but remains underexplored in personalized contexts. Together, these methods provide a comprehensive pipeline for delivering efficient, relevant, and personalized outputs, balancing their strengths in semantic understanding, relevance, and flexibility with challenges related to computational costs and implementation complexity.
\subsection{Generation}  \label{sec:Generation}
\subsubsection{\textbf{Definition}}
Personalized generation incorporates user-specific retrieved documents $D^*$, task-specific prompt $prompt$, and user preference information $p$ via the generator $\mathcal{G}$ parameterized by $\theta$ to produce tailored content $g^*$ aligned with individual preference, where the flow is shown in Figure~\ref{fig:generation}. The generation process can be formulated as
\begin{equation}
g^* = \mathcal{G} \left(D^*,\text{prompt},p,\theta \right) .
\end{equation}
Personalized generation can be achieved by incorporating explicit and implicit preferences. Explicit preference-driven methodologies utilize direct input signals (\eg $D^*$, $\text{prompt}$, and $p$), to tailor outputs to specific user preferences. Conversely, implicit preference-encoded approaches embed personalized information within the parameters $\theta$ of the generator model, during training, thereby facilitating preference alignment without the necessity for explicit runtime inputs.

\begin{figure}[t]
    \centering
    \includegraphics[width=\linewidth]{figures/generation.pdf}
    \caption{Overview of the personalized generation stage.}
    \label{fig:generation}
\end{figure}

\subsubsection{\textbf{Generation from Explicit Preferences}}
Integrating explicit preferences into LLMs facilitates personalized content generation. Explicit preference information encompasses user demographic information (\eg age, occupation, gender, location), user behavior sequences (reflecting historical behavioral patterns), and user historical output texts (capturing writing style and tone preferences). The injection of explicit preferences for personalized generation can be categorized into three types: (1) Direct-integrated Prompting, (2) Summary-augmented Prompting, and (3) Adaptive Prompting.

\paragraph{\textbf{\textit{{(1). Direct-integrated Prompting.}}}}
Integrating user explicit preferences into language models through prompting enables the prediction of users' intent and behavioral patterns, facilitating personalized content generation. For instance, P$^2$~\cite{jiang2023evaluating}, Character Profiling~\cite{yuan2024evaluating}, and OpinionQA~\cite{santurkar2023whose} integrate personalized data into LLMs through prompting for role-playing task, thereby aligning the model's responses with specified user profiles. ~\citet{kang2023llms} and ~\citet{liu2023chatgpt} integrate interaction histories into LLMs via prompting to predict user rating for candidate items. Cue-CoT~\cite{wang2023cue} employs chain-of-thought reasoning to infer user needs from contextual cues, enabling personalized responses to in-depth dialogue questions. Additionally, TICL~\cite{cho2025tuning} proposes a trial-and-error framework that critiques initial LLM-generated responses, derives explanations and integrates these negative examples into prompts to improve personalization alignment.

\paragraph{\textbf{\textit{{(2). Summary-augmented Prompting.}}}}
Direct integration of personalized information via prompting struggles with ambiguous intent signals: Lengthy interaction histories introduce noise that obscures critical behavioral patterns~\cite{liu2024lost}, while sparse behavioral data lacks sufficient context for LLMs to derive meaningful user preferences. To address these issues, recent approaches focus on summarizing user personalized intents and integrating them into prompts. For instance, GPG~\cite{zhang2024guided} extracts key user habits and preferences from personal contexts, enabling fine-grained personalization. Similarly, LLMs are employed to generate task-specific summaries of user preferences, enhancing retrieval-augmented personalized generation capabilities~\cite{richardson2023integrating}. In recommendation systems, ONCE~\cite{liu2024once}, LLMTreeRec~\cite{zhang2025llmtreerec}, and KAR~\cite{xi2024towards} leverage historical user-item interactions to summarize user preferences. Furthermore, Matryoshka~\cite{li2024matryoshka} generates user preference summaries by dynamically retrieving and synthesizing historical data.

\paragraph{\textbf{\textit{{(3). Adaptive Prompting.}}}}
Manually designing personalized prompts demands both expert knowledge and significant labor, motivating the development of automated methods for personalized prompt generation. For example, \citet{li2024learning} trains a personalized prompt rewriter via supervised and reinforcement learning. RecGPT~\cite{zhang2024recgpt} and PEPLER-D~\cite{li2023personalized} leverage prompt tuning to generate personalized prompts, enhancing sequential and explainable recommendations, respectively. GRAPA~\cite{qu2024graph} integrates semantic and collaborative signals from user-item interaction graphs with graph neural networks to generate context-aware personalized prompts. SGPT~\cite{deng2024unlocking} employs prompt tuning
to jointly model common and group-specific patterns, bridging generalized and personalized federated learning paradigms. Furthermore, PFCL~\cite{yu2024personalized} achieves multi-granularity human preference modeling: coarse-grained prompts distill shared knowledge, while fine-grained prompts adapt to individual user characteristics.

\subsubsection{\textbf{Generation from Implicit Preferences}}
Unlike explicit preference modeling, which captures user preferences through textual input, implicit preference-based methods incorporate personalization through internal parameters. This personalization is achieved either through Parameter-Efficient Fine-tuning (PEFT) techniques, such as LoRA~\cite{hu2022lora}, or reinforcement learning-based approaches for preference alignment~\cite{li2024learning,chen2024pad}. Based on these strategies, we classify existing methods into two categories: (1) Fine-tuning-Based Methods and (2) Reinforcement Learning-Based Methods.

\paragraph{\textbf{\textit{{(1). Fine-tuning Based Methods.}}}}
For fine-tuning methods, LoRA is the most widely adopted since it is resource-efficient and enables rapid adaptation without compromising model performance. PLoRA~\cite{zhang2024personalized} introduces a personalized knowledge integration framework that combines task-specific LoRA with user-specific knowledge. Similarly, LM-P~\cite{wozniak2024personalized} personalizes information via LoRA by incorporating User ID as a personalization factor. MiLP~\cite{zhang2024personalized} employs Bayesian optimization to determine the optimal personalization injection configuration, including LoRA settings, to effectively capture and utilize user-specific information. OPPU~\cite{tan2025democratizing} and PER-PCS~\cite{tan2024personalized} follow a similar approach, leveraging user history data for fine-tuning LoRA-based personalization. However, PER-PCS differs by incorporating a gating module that selects the appropriate LoRA, enabling fine-grained personalization. Additionally, Review-LLM~\cite{peng2024reviewllm} integrates LoRA for supervised fine-tuning in the task of personalized review generation.

Beyond LoRA-based approaches, alternative pipelines have been proposed for personalized generation. UserIdentifier~\cite{mireshghallah2021useridentifier} introduces a user-specific identifier, significantly reducing training costs while enhancing personalized demonstration. UserAdapter~\cite{zhong2021useradapter} proposes user-independent prefix embeddings, leveraging prefix tuning for personalization. Meanwhile, HYDRA~\cite{zhuang2406hydra} achieves implicit personalization by training user-specific headers. Recently, researchers have also explored fine-tuning personalized model on edge devices~\cite{peng2024pocketllm} and collaborative learning between small and large language models to enable more personalized generation~\cite{zhang2024cogenesis}.

\paragraph{\textbf{\textit{{(2). Reinforcement Learning Based Methods.}}}}
Apart from fine-tuning based methods, recent research has explored reinforcement learning based techniques to personalize text generation by aligning outputs with user preferences. P-RLHF~\cite{li2024personalized} has been proposed to jointly learn a user-specific and reward model to enable text generation that aligns with a user's styles or criteria. P-SOUPS~\cite{jang2023personalized} models multiple user preferences as a Multi-Objective Reinforcement Learning (MORL) problem, decomposing preferences into multiple dimensions, each trained independently. PAD~\cite{chen2024pad} aligns text generation with human preferences during inference by utilizing token-level personalized rewards to guide the decoding process. REST-PG~\cite{salemi2025reasoning} introduces a framework that trains large language models to reason over personal data during response generation. This approach first generates reasoning paths to enhance the LLM's reasoning ability and then employs Expectation-Maximization Reinforced Self-Training to iteratively refine the model based on its high-reward outputs. Additionally,~\citet{salemi2024optimization} incorporate reinforcement learning into the RAG pipeline to improve retrieval accuracy, thereby enhancing the personalization of generated content. Other applications include RewriterSlRl~\cite{li2024learning}, which has been introduced to generate text via RL-based personalized prompt rewriting using API-based LLMs, and ~\citet{kulkarni2024reinforcement}, who explore the use of reinforcement learning to optimize RAG for improving the relevance and coherence of chatbot responses in specialized domains, ultimately enhancing user satisfaction and engagement.

\subsubsection{\textbf{Discussion}}
Personalized generation can be adopted via both explicit and implicit preference injection, yet they exhibit distinct characteristics that make them suitable for different scenarios.
In explicit preference-based generation, personalization is clearly defined through user profile descriptions, contextual information, and similar inputs, which are incorporated into generators via prompts. A key advantage of this approach is explainability, as the personalized information is explicitly provided and easily traceable. 
Despite leveraging provided preferences and internal knowledge, explicit preference injection's personalization is constrained by model capabilities and irrelevant information interference.
In contrast, implicit preference-based generation internalizes personalized information into the generator's parameters through scene-specific personalized data, thereby adapting the model for more fine-grained personalization. However, these methods typically incur substantial training and computational costs, as they require fine-tuning the generator’s internal parameters. Therefore, selecting between these approaches should be guided by the specific application scenario and resource constraints.
\subsection{From RAG to Agent}
% 

\begin{figure*}[t]
    \centering
    \includegraphics[width=\linewidth]{figures/PRAG_agent.pdf}
    \caption{Overview of transition from personalized RAG to personalized agent.}
    \label{fig:Agent}
\end{figure*}


\subsubsection{\textbf{Definition}}
A personalized LLM-based agent is a system designed to dynamically incorporate user context, memory, and external tools or APIs to support highly personalized and goal-oriented interactions \cite{xi2025rise,huang2024understanding,chen2024persona}, and solve problems in a goal-oriented manner \cite{li2024personal,singh2025agentic}.
From the previously introduced stages of RAG, we observe that the evolution of personalized RAG reveals a structural convergence with agent architectures. We analyze them from three key perspectives: 
\begin{itemize}[leftmargin=*]
\item \textbf{Personalized Understanding}: This phase within the agent parallels the query understanding and rewriting process of RAG as outlined in Section \ref{sec:Pre-retrieval}. However, it extends beyond static semantic parsing by incorporating dynamic user profiling \cite{wang2023rolellm} and role modeling \cite{shao2023character}. This integration enables the agent to dynamically align interactions with implicit user preferences, facilitating personalized responses and task-specific adaptations \cite{ran2024capturing}.
%
\item \textbf{Personalized Planning and Execution}: This phase in agents mirrors RAG's retrieval process in Section \ref{sec:Retrieval} yet it advances beyond static document retrieval by incorporating real-time memory management \cite{park2023generative} and sophisticated tool and API calling \cite{wangvoyager}. This approach ensures the dynamic alignment of external knowledge with personalized constraints, such as integrating medical history in healthcare agents \cite{abbasian2023conversational}, to deliver context-aware and user-specific outcomes. %\yc{can use the structure of this paragraph to refine the first and third perspectives.}
%
\item \textbf{Personalized Generation}: This phase in agents mirrors RAG's generative process in Section \ref{sec:Generation} but transcends static template-based generation by integrating user preference and fact alignment. Agents dynamically enforce user preferences and ensure fact consistency through role-specific mechanisms (\eg social adaptability in conversational agents \cite{abbasian2023conversational}), enabling outputs to evolve in harmony with personalized and situational constraints rather than relying solely on predefined generative frameworks.

\end{itemize}
 In general we frame agent architectures as ``\textbf{personalized RAG++}'', where persistent memory \cite{wang2024crafting} replaces static indexes, and tool APIs \cite{cai2025large} serve as dynamic knowledge connectors, enabling complicated, human-aligned interactions beyond one-shot retrieval, as shown in Figure \ref{fig:Agent}. This progression highlights that as RAG systems incorporate deeper personalization—requiring user-state tracking, adaptive tool usage, and context-aware generation, they inherently adopt agent-like capabilities.

\subsubsection{\textbf{Personalized Understanding}}
Personalized understanding refers to an agent’s ability to accurately interpret user inputs by integrating user intent recognition and contextual analysis. This process ensures interactions that are both meaningful and contextually appropriate. The rationale behind this classification lies in its capacity to address three core aspects of understanding: recognizing user intent, analyzing context, and leveraging user profiles. Each of these aspects plays a distinct role in improving the agent's performance. %\yc{emphasize the rationale of these three classification, what are the different impacts on query/user understanding}

\paragraph{\textbf{\textit{{(1). User-profile Understanding.}}}}
In user-profile understanding, an agent's personalized ability primarily depends on its capacity to accurately model and understand the user's preferences, context, and intentions. \citet{xu2024penetrative} proposes a framework in which LLMs are designed to understand the physical world, thereby facilitating a deeper connection between the agent and its environment, which is essential for accurate task execution. \citet{abbasian2023conversational} further expands this understanding by emphasizing the importance of personalization in health agents, where the user’s profile directly influences the behavior and decisions of the agent. This user understanding is foundational to ensuring that the AI agent performs tasks in a way that aligns with individual user needs.

\paragraph{\textbf{\textit{{(2). Role Understanding.}}}}
In agent's role understanding, the role of the agent within these environments is also crucial. Recent studies focus on enhancing role-playing capabilities within LLMs. \citet{wang2023rolellm} introduce RoleLLM, a benchmark that aims to elicit and refine the role-playing abilities of LLMs, demonstrating how role understanding influences agent performance in conversational tasks. Similarly, \citet{shao2023character} present Character-LLM, a trainable agent framework for role-playing, which tailors its responses based on predefined roles. \citet{wang2023incharacter} introduce a method for evaluating personality fidelity in role-playing agents through psychological interviews, aiming to enhance the realism and consistency of AI-driven characters. This role understanding allows for more contextually appropriate interactions, increasing the relevance and utility of AI agents across various applications. %\yc{what is the relation with user-profile, and user/query understanding}

\paragraph{\textbf{\textit{{(3). User-role Joint Understanding.}}}}
In agent's user-role joint understanding, the intersection of user and role understanding is explored through frameworks that evaluate and enhance the social and personality aspects of LLMs. SocialBench \citet{chen2024socialbench} provides a sociality evaluation framework for role-playing agents. \citet{dai2024mmrole}, and \cite{ran2024capturing} extend this by incorporating multi-modal data and personality-indicative information, respectively, which allows agents to better adapt to both user and role understanding in dynamic environments. Furthermore, \citet{wang2023enabling} offers a perspective on how role and environment understanding can improve user experience. \citet{tu2024charactereval} contribute by providing a benchmark specifically for evaluating role-playing agents in the Chinese context, adding a cultural dimension to role understanding. Finally, Neeko \cite{yu2024neeko} further advances role-based interactions.


\subsubsection{\textbf{Personalized Planning and Execution}}
Personalized planning and execution refer to the process of designing and implementing strategies or actions that are specifically tailored to an individual’s unique context, and goals \cite{huang2022language,zhangbootstrap,park2023generative,singh2024personal}. It requires agents to dynamically integrate long-term memory, real-time reasoning, and external tool utilization \cite{hong2023metagpt,zheng2023agents,hongru2023large}, as demonstrated in healthcare decision support \cite{abbasian2023conversational} and travel planning scenarios \cite{cai2025large}. We analyze two fundamental components that enable this personalization in the following.

\paragraph{\textbf{\textit{{(1). Memory Management.}}}}
Effective memory systems allow agents to integrate users' historical preferences, behavioral patterns, and contextual habits, enhancing their ability to make planning and tailor interactions to user-specific needs \cite{wangvoyager,cai2025large,wangmemoryllm}. The EMG-RAG framework \cite{wang2024crafting} combines editable memory graphs with retrieval-augmented generation to maintain dynamic user profiles, while \citet{park2023generative} implements memory streams and periodic reflection mechanisms to simulate human-like behavior. In healthcare applications, \citet{abbasian2023conversational} integrates multimodal user data through specialized memory modules to optimize treatment recommendations. For recommendation systems, RecAgent \cite{wang2023user} employs hierarchical memory structures to model user interaction patterns across multiple domains. Recent advances like TravelPlanner+ \cite{singh2024personal} demonstrate how memory-augmented LLMs achieve higher relevance in personalized itinerary generation compared to generic planners. %\yc{current description seems to have no relation with planning}

\paragraph{\textbf{\textit{{(2). Tool and API Calling.}}}}
The integration of external tools expands agents' capabilities beyond pure linguistic reasoning, enabling agents to interact with users and perform personalized tasks \cite{cai2025large,zhangbootstrap,xu2024penetrative,wangvoyager,wang2023enabling}. For instance, VOYAGER \cite{wangvoyager} establishes a paradigm for lifelong skill acquisition through automatic API curriculum learning and skill library construction. In robotics, \citet{zhangbootstrap} develops a bootstrapping framework where LLMs guide robots in tool-mediated skill discovery, enabling a high success rate in novel object manipulation tasks. The PUMA framework \cite{cai2025large} demonstrates how personalized web agents can achieve performance gains in e-commerce tasks through adaptive API orchestration. For mobile interaction, \citet{wang2023enabling} implements few-shot tool learning to handle diverse UI operations with minimal training data. These approaches highlight the importance of tool grounding mechanisms \cite{huang2022language} that translate linguistic plans into executable API sequences while maintaining personalization constraints.

This synthesis highlights that modern agent systems achieve enhanced personalization through two primary strategies: 1) Memory-augmented architectures, which leverage editable memory graphs \cite{wang2024crafting}, reflection mechanisms \cite{park2023generative}, and hierarchical memory structures \cite{wang2023user} to dynamically adapt to user preferences across various domains; and 2) Tool and API integration, which expand agent capabilities by balancing generalization  with specialization. Future work may explore improving the contextual relevance and adaptability of memory systems while optimizing real-time tool interaction for seamless task execution.



\subsubsection{\textbf{Personalized Generation}}
Based on the foundation of personalized planning and execution mechanisms, which enable agents to adapt strategies to user-specific contexts \cite{huang2022language,zhangbootstrap}, the next critical concern lies in personalized generation. This capability ensures that generated outputs not only align with factual correctness but also resonate with users' unique preferences, personality traits, and situational needs. Personalized generation bridges the gap between adaptive reasoning and human-aligned outcomes, allowing agents to produce contextually relevant and emotionally appropriate responses.

\paragraph{\textbf{\textit{{(1). Alignment with User Fact.}}}}
Alignment with User Fact emphasizes the accuracy, consistency, and factual grounding of personalized responses, ensuring they remain trustworthy across diverse user interactions. This is particularly challenging in personalized agents, where maintaining character authenticity while avoiding hallucinations requires balancing creativity with factual adherence.
Recent advances address these challenges through improved training frameworks and evaluation metrics. For instance, Character-LLM \cite{shao2023character} integrates memory-augmented architectures to reduce hallucinations while preserving character-specific traits. \citet{wang2024investigating} investigate quantization effects on personality consistency in edge-deployed agents and stabilize outputs under computational constraints. \citet{dai2024mmrole} ensures multimodal consistency (text-image) in role-playing. These works highlight the importance of architectural innovations and rigorous evaluation in achieving reliability.

\paragraph{\textbf{\textit{{(2). Alignment with User Preferences.}}}}
Alignment with user preferences ensures that generated outputs reflect individualized personalities, values, and interaction styles. This requires agents to dynamically interpret implicit user cues and adapt responses accordingly.
\citet{wang2023rolellm} benchmarks role-specific alignment. \citet{ran2024capturing} improves personality fidelity via psychological scale datasets. \citet{wang2023incharacter} quantifies alignment via psychological interviews. \citet{chen2024socialbench} evaluates social adaptability in conversations.

\subsubsection{\textbf{Discussion}}
The architectural evolution from RAG to personalized agents introduces significant advancements in human-AI interaction but also surfaces critical challenges that warrant further investigation.

\textbf{Personalized Understanding}, while enabling interpretation of user intent and context, faces limitations in real-time adaptability and generalization. Current approaches like RoleLLM \cite{wang2023rolellm} and Character-LLM \cite{shao2023character} demonstrate robust role-specific comprehension but struggle with dynamic user state tracking, particularly when handling evolving preferences or multi-session interactions. Furthermore, cultural specificity in benchmarks like CharacterEval \cite{tu2024charactereval} reveals gaps in global applicability, as agents trained on region-specific data often fail to generalize across diverse sociocultural contexts. Future work could explore hybrid architectures that combine continuous learning mechanisms with privacy-preserving federated learning to address these adaptability constraints while maintaining user trust.

\textbf{Personalized Planning and Execution}, achieves remarkable task specialization through memory management and tool integration, yet suffers from scalability issues in complex environments. While frameworks like EMG-RAG \cite{wang2024crafting} and VOYAGER \cite{wangvoyager} effectively manage user-specific constraints, their reliance on predefined API taxonomies limits emergent tool discovery in novel scenarios. The "cold-start" problem persists in domains requiring rapid skill acquisition, as seen in healthcare applications \cite{abbasian2023conversational}, where delayed API responses can compromise decision-making efficacy. A promising direction involves developing meta-reasoning architectures that dynamically prioritize memory recall versus tool invocation based on situational urgency and confidence thresholds.

\textbf{Personalized Generation} balances factual accuracy with preference alignment but risks over-fitting, where excessive finetuning to user profiles may reinforce cognitive biases. Techniques address surface-level alignment but lack mechanisms for ethical boundary detection. For instance, agents might inadvertently propagate harmful stereotypes when mirroring user preferences without critical oversight. Future systems could integrate value-aligned reinforcement learning with human-in-the-loop validation to preserve authenticity while preventing detrimental customization.


\section{Evaluation and Dataset}\label{sec:evaluation&dataset}
\begin{table*}[t]
% \setlength\tabcolsep{3pt}  %可以控制列间距
% \renewcommand{\arraystretch}{1} %可以控制行间距
\caption{Datasets and metrics for personalized RAG and Agent.}
\label{tab:datasets}
\centering
\resizebox{\textwidth}{!}{
\begin{tabular}{c|c|c|c} 
\toprule
\textbf{Field}                          & \begin{tabular}[c]{@{}c@{}}\textbf{Metrics}\\\textbf{Category}\end{tabular}      & \textbf{Metrics}                                                                                                                                                & \textbf{Datasets}                                                                                                                                                                                                                                                                                                                                                                                                                                                                                                                                                                                                                                                                                                                                                                                                                                                                                                                                                                                                                                                                                                                                     \\ 
\midrule
\multirow{8}{*}{\textbf{Pre-retrieval}} & \begin{tabular}[c]{@{}c@{}}Textual\\Quality\end{tabular}       & BLEU, ROUGE, EM                                                                                                                                        & Avocado Research Email Collection \cite{oard2015avocado,li2024learning}, Amazon review\cite{ni2019justifying,li2024learning}, Reddit comments\cite{reddit2015,li2024learning}, Amazon ESCI dataset\cite{reddy2022shopping,nguyen2025rl}, PIP                                                                                                                                                                                                                                                                                                                                                                                                                                                                                                                                                                                                                                                                                                                                                                                                                             \\ 
\cmidrule{2-4}
                               & \begin{tabular}[c]{@{}c@{}}Information\\Retrieval\end{tabular} & \begin{tabular}[c]{@{}c@{}}MAP, MRR, NDCG, \\Precision, Recall, RBP\end{tabular}                                                                       & \begin{tabular}[c]{@{}c@{}}AOL\cite{AOL,zhou2024cognitive}, WARRIORS\cite{ren2024bases}, Personalized Results Re-Ranking benchmark~\cite{bassani2022multi}, del.icio.us~\cite{bender2008exploiting, bouadjenek2011personalized,wu2017personalized, zhou2012improving},\\~Flickr~\cite{bender2008exploiting,schifanella2010folks}, CiteULike~\cite{bertier2009toward,bouadjenek2019personalized}, LRDP~\cite{biancalana2009social}, Delicious~\cite{wetzker2008analyzing}, Bibsonomy~\cite{mulhem2016axiomatic},\\~Wikipedia~\cite{baumann2024psqe, gao2010utilizing}\end{tabular}                                                                                                                                                                                                                                                  \\ 
\cmidrule{2-4}
                               & Classification                                                 & Accuracy, Macro-F1                                                                                                                                     & SCAN \cite{lake2018generalization,zhou2022least}, AITA WORKSM\cite{WORKSM,mysore2023pearl}, Robust04 \cite{li2023agent4ranking}                                                                                                                                                                                                                                                                                                                                                                                                                                                                                                                                                                                                                                                                                                                                                                                                                                                                                                                                                           \\ 
\cmidrule{2-4}
                               & Others                                                         & \begin{tabular}[c]{@{}c@{}}XEntropy, PMS, Image-Align,\\~PQEC, Prof$_{\text{overlap}}$\end{tabular}                                                      & Amazon ESCI dataset\cite{reddy2022shopping,nguyen2025rl}, PIP, Bibsonomy~\cite{mulhem2016axiomatic}                                                                                                                                                                                                                                                                                                                                                                                                                                                                                                                                                                                                                                                                                                                                                                                                                                                                                                                                                                                        \\ 
\midrule
\multirow{7}{*}{\textbf{Retrieval}}     & \begin{tabular}[c]{@{}c@{}}Textual\\Quality\end{tabular}       & BLEU, ROUGE, Dis, PPL                                                                                                                                  & TOPDIAL \cite{wang2023target}, Pchatbot \cite{qian2021pchatbot}, DuLemon \cite{xu2022long}                                                                                                                                                                                                                                                                                                                                                                                                                                                                                                                                                                                                                                                                                                                                                                                                                                                                                                                                                                                                \\ 
\cmidrule{2-4}
                               & \begin{tabular}[c]{@{}c@{}}Information\\Retrieval\end{tabular} & Recall, MRR, Precision, F1                                                                                                                             & LiveChat \cite{gao2023livechat}, Pchatbot \cite{qian2021pchatbot}, DuLemon \cite{xu2022long}                                                                                                                                                                                                                                                                                                                                                                                                                                                                                                                                                                                                                                                                                                                                                                                                                                                                                                                                                                                              \\ 
\cmidrule{2-4}
                               & Classification                                                 & Accuracy, Succ                                                                                                                                         & TOPDIAL \cite{wang2023target}, PersonalityEvd \cite{sun2024revealing}, DuLemon \cite{xu2022long}, PersonalityEdit \cite{mao2024editing}                                                                                                                                                                                                                                                                                                                                                                                                                                                                                                                                                                                                                                                                                                                                                                                                                                                                                                                                  \\ 
\cmidrule{2-4}
                               & Others                                                         & \begin{tabular}[c]{@{}c@{}}Fluency, Coherence,\\Plausibility, ES, DD, TPEI, PAE\end{tabular}                                                           & PersonalityEvd \cite{sun2024revealing}, PersonalityEdit \cite{mao2024editing}                                                                                                                                                                                                                                                                                                                                                                                                                                                                                                                                                                                                                                                                                                                                                                                                                                                                                                                                                                                                                              \\ 
\midrule
\multirow{6}{*}{\textbf{Generation}}    & \begin{tabular}[c]{@{}c@{}}Textual\\Quality\end{tabular}       & \begin{tabular}[c]{@{}c@{}}BLEU, ROUGE, Dis, \\PPL, METEOR\end{tabular}                                                                                & LaMP \cite{salemi2024lamp}, Long LaMP \cite{kumar2024longlamp}, Dulemon \cite{xu2022long}, PGraphRAG \cite{au2025personalized}, AmazonQA/Products~\cite{deng2022toward}, Reddit~\cite{zhong2022less}, MedicalDialogue~\cite{zhang2023memory}                                                                                                                                                                                                                                                                                                                                                                                                                                                                                                                                                                                                                                                                                                                          \\ 
\cmidrule{2-4}
                               & Classification                                                 & Accuracy, F1, Persona F1                                                                                                                               & LaMP \cite{salemi2024lamp}, Long LaMP \cite{kumar2024longlamp}, Dulemon \cite{xu2022long}, AmazonQA/Products~\cite{deng2022toward}, Reddit~\cite{zhong2022less}, MedicalDialogue~\cite{zhang2023memory}                                                                                                                                                                                                                                                                                                                                                                                                                                                                                                                                                                                                                                                                                                                                                                                \\ 
\cmidrule{2-4}
                               & Regression                                                     & MAE, RMSE                                                                                                                                              & LaMP \cite{salemi2024lamp}, Long LaMP \cite{kumar2024longlamp}, PGraphRAG \cite{au2025personalized}                                                                                                                                                                                                                                                                                                                                                                                                                                                                                                                                                                                                                                                                                                                                                                                                                                                                                                                                                                                       \\ 
\cmidrule{2-4}
                               & Others                                                         & \begin{tabular}[c]{@{}c@{}}Fluency, Mean Success Rate, \\Median Relative Improvements\end{tabular}                                                     & Personalized-Gen \cite{alhafni2024personalized}                                                                                                                                                                                                                                                                                                                                                                                                                                                                                                                                                                                                                                                                                                                                                                                                                                                                                                                                                                                                                                                                             \\ 
\midrule
\multirow{12}{*}{\textbf{Agent}}         & \begin{tabular}[c]{@{}c@{}}Textual\\Quality\end{tabular}       & \begin{tabular}[c]{@{}c@{}}BLEU, ROUGE, METEOR, \\CIDEr, EM, Fluency, \\Coherence, Instruction \\Adherence, Consistency \\related metrics\end{tabular} & \begin{tabular}[c]{@{}c@{}}RICO~\cite{wang2023enabling}, RoleBench~\cite{wang2023rolellm}, \citet{shao2023character}, Socialbench~\cite{chen2024socialbench}, MMRole-Data~\cite{dai2024mmrole}, ROLEPERSONALITY~\cite{ran2024capturing}, ChatHaruhi~\cite{wang2023incharacter}, \\Character-LLM-Data~\cite{yu2024neeko}, Knowledge Behind Persona~\cite{hongru2023large}, \citet{wang2024crafting}, \citet{wang2024investigating}, \citet{zheng2023agents}\end{tabular}                                                                                                                                                                                                                                                                                                                          \\ 
\cmidrule{2-4}
                               & \begin{tabular}[c]{@{}c@{}}Information\\Retrieval\end{tabular} & Recall, F1, Precision                                                                                                                                  & Knowledge Behind Persona~\cite{hongru2023large}                                                                                                                                                                                                                                                                                                                                                                                                                                                                                                                                                                                                                                                                                                                                                                                                                                                                                                                                                                                                                                                             \\ 
\cmidrule{2-4}
                               & Classification                                                 & \begin{tabular}[c]{@{}c@{}}Accuracy, Failure Rate, \\Classification Accuracy, \\Preference Rate, \\Correctness\end{tabular}                            & \begin{tabular}[c]{@{}c@{}}MIT-BIH Arrhythmia Database~\cite{xu2024penetrative}, VirtualHome~\cite{huang2022language}, Socialbench~\cite{chen2024socialbench}, ARC~\cite{renze2024self},AGIEval~\cite{renze2024self},HellaSwag~\cite{renze2024self},MedMCQA~\cite{renze2024self},\\AQUA-RAT~\cite{renze2024self},LogiQA~\cite{renze2024self},LSAT-AR~\cite{renze2024self},LSAT-LR~\cite{renze2024self},LSAT-RC~\cite{renze2024self},SAT-English~\cite{renze2024self},SAT-Math~\cite{renze2024self}, PersonalWAB~\cite{cai2025large},\\~TravelPlanner+~\cite{singh2024personal}\end{tabular}  \\ 
\cmidrule{2-4}
                               & Others                                                         & \begin{tabular}[c]{@{}c@{}}Pass@k, Executability, \\Productivity, Plausibility of the \\Story\end{tabular}                                             & \citet{hong2023metagpt}, \citet{zheng2023agents}                                                                                                                                                                                                                                                                                                                                                                                                                                                                                                                                                                                                                                                                                                                                                                                                                                                                                                                                                                                                                                                           \\
\bottomrule
\end{tabular}}
\end{table*}

In the evolving landscape of personalization, from RAG to advanced Agent-based systems, the evaluation of models relies heavily on diverse datasets and metrics tailored to specific tasks. This survey categorizes metrics into several key types: Textual Quality metrics (\eg BLEU, ROUGE, METEOR) assess the fluency and coherence of generated outputs; Information Retrieval metrics (\eg MAP, MRR, Recall) evaluate the accuracy and relevance of retrieved information; Classification metrics (\eg Accuracy, F1) measure task-specific correctness; Regression metrics (\eg MAE, RMSE) quantify prediction errors; and Other metrics (\eg Fluency, Pass@k) address domain-specific or task-unique aspects like plausibility or executability. These metrics span pre-retrieval, retrieval, generation, and agent-based personalization approaches, reflecting their varied objectives. To provide a comprehensive overview, we compile an extensive list of datasets across these fields, as detailed in Table~\ref{tab:datasets}. These datasets, paired with their respective metrics, enable researchers to benchmark and refine personalized systems, from enhancing query rewriting to enabling autonomous agents in physical and virtual environments.
\section{Challenges and Future Directions}
\label{sec:futuredirection}

Personalized RAG and agent-based systems still face several critical challenges that warrant further exploration. We list them as follows:
\begin{itemize}[leftmargin=*]
    \item \textbf{Balancing Personalization and Scalability}: Integrating personalization data (such as preferences, history, and contextual signals) into RAG processes often increases computational complexity,
    making it difficult to maintain efficiency and scalability across large-scale systems. Future work could explore lightweight, adaptive embeddings and hybrid frameworks that seamlessly fuse user profiles with real-time contexts.
    \item \textbf{Evaluating Personalization Effectively}: Current metrics like BLEU, ROUGE, and human evaluations fall short in capturing the nuanced alignment of outputs with dynamic user preferences, lacking tailored measures for personalization efficacy. Developing specialized benchmarks and metrics that assess long-term user satisfaction and adaptability is crucial for real-world applicability.
    \item \textbf{Preserving Privacy through Device–Cloud Collaboration:} Personalized retrieval often involves processing sensitive user data, raising privacy concerns, especially with the increased global emphasis on data protection regulations, such as the European Union’s General Data Protection Regulation (GDPR). Consequently, a promising approach is the collaborative integration of on-device small Language models which handle sensitive personal data locally, with cloud-based LLM, which provides broader contextual knowledge. 
    \item \textbf{Personalized Agent Planning:} Current research on agent planning remains mainly in its early stages, with much of the work focusing on building foundational frameworks such as GUI agents~\cite{nguyen2024gui} and the application of agents across diverse domains~\cite{wang2024survey}. Notably, the incorporation of personalized approaches has yet to be widely adopted. Exploring how to integrate personalized support into existing frameworks to enhance user experience represents a promising and valuable direction for future investigation.
    \item \textbf{Ensuring Ethical and Coherent Systems}: Bias in data processing, privacy concerns in user profiling, and coherence across retrieval and generation stages remain unresolved. Future directions should prioritize ethical safeguards, privacy-preserving techniques, and cross-stage optimization to build trustworthy, unified personalized systems.
\end{itemize}

\section{Conclusion}
In this paper, we explore the landscape of personalization from Retrieval-Augmented Generation (RAG) to advanced LLM-based Agents, detailing adaptations across pre-retrieval, retrieval, and generation stages while extending into agentic capabilities. By reviewing recent literature, datasets, and metrics, we highlight the progress and diversity in enhancing user satisfaction through tailored AI systems. 
However, challenges such as scalability, effective evaluation, and ethical concerns underscore the need for innovative solutions. Future research should focus on lightweight frameworks, specialized benchmarks, and privacy-preserving techniques to advance personalized AI. 
Relevant papers and resources are also compiled online for ease of future research.

% \input{sections/table_section}

\bibliographystyle{ACM-Reference-Format}
\bibliography{sample-base}

\end{document}
\endinput
%%
%% End of file `sample-manuscript.tex'.

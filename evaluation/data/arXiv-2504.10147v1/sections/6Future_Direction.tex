\section{Challenges and Future Directions}
\label{sec:futuredirection}

Personalized RAG and agent-based systems still face several critical challenges that warrant further exploration. We list them as follows:
\begin{itemize}[leftmargin=*]
    \item \textbf{Balancing Personalization and Scalability}: Integrating personalization data (such as preferences, history, and contextual signals) into RAG processes often increases computational complexity,
    making it difficult to maintain efficiency and scalability across large-scale systems. Future work could explore lightweight, adaptive embeddings and hybrid frameworks that seamlessly fuse user profiles with real-time contexts.
    \item \textbf{Evaluating Personalization Effectively}: Current metrics like BLEU, ROUGE, and human evaluations fall short in capturing the nuanced alignment of outputs with dynamic user preferences, lacking tailored measures for personalization efficacy. Developing specialized benchmarks and metrics that assess long-term user satisfaction and adaptability is crucial for real-world applicability.
    \item \textbf{Preserving Privacy through Device–Cloud Collaboration:} Personalized retrieval often involves processing sensitive user data, raising privacy concerns, especially with the increased global emphasis on data protection regulations, such as the European Union’s General Data Protection Regulation (GDPR). Consequently, a promising approach is the collaborative integration of on-device small Language models which handle sensitive personal data locally, with cloud-based LLM, which provides broader contextual knowledge. 
    \item \textbf{Personalized Agent Planning:} Current research on agent planning remains mainly in its early stages, with much of the work focusing on building foundational frameworks such as GUI agents~\cite{nguyen2024gui} and the application of agents across diverse domains~\cite{wang2024survey}. Notably, the incorporation of personalized approaches has yet to be widely adopted. Exploring how to integrate personalized support into existing frameworks to enhance user experience represents a promising and valuable direction for future investigation.
    \item \textbf{Ensuring Ethical and Coherent Systems}: Bias in data processing, privacy concerns in user profiling, and coherence across retrieval and generation stages remain unresolved. Future directions should prioritize ethical safeguards, privacy-preserving techniques, and cross-stage optimization to build trustworthy, unified personalized systems.
\end{itemize}

\section{What is Personalization} \label{sec:what}
Personalization in current research refers to the tailoring of model predictions or generated content to align with an individual's preferences. In the context of RAG and agents, personalization involves incorporating user-specific information at various stages of the RAG pipeline or within agents. User personalization can be categorized into the following types:

\begin{itemize}[leftmargin=*] 
\item Explicit User Profile: Explicitly presented user information, including biographical details, attributes (\eg age, location, gender, education), and social connections (\eg social networks).
\item User Historical Interactions: Behavioral data, including browsing history, clicks, and purchases, which help infer user interests and preferences to improve personalization. 
\item User Historical Content: Implicit personalization derived from user-generated content, such as chat history, emails, reviews, and social media interactions. 
\item Persona-Based User Simulation: The use of LLM-based agents to simulate and generate personalized interactions.
\end{itemize}

Integrating this personalized information at various stages of the RAG and agent workflows enables dynamic alignment with human preferences, thereby making responses more user-centric and adaptive.
\section{Additional Ablation Studies and Analyses}
\subsection{More Analyses of the DynamicSSM Block}
We provide a more detailed discussion on the components of the DynamicSSM Block in Tab.~\ref{tab:dssm}. 
The DynamicSSM Block is composed of three primary operations: (\romannumeral 1) distortion alleviation, as described in the main text Equation~9, (\romannumeral 2) addressing lighting and color inconsistencies, as detailed in the main text Equation~10, and (\romannumeral 3) enhancing feature representation, as formulated in the main text Equation~11. 
As shown in Tab.~\ref{tab:dssm}, all three operations individually contribute to improved performance, and their combination results in the best overall performance. 
A comparison between experiments \ding{192} and \ding{195} demonstrates that integrating all three operations in the DynamicSSM Block leads to an overall HOTA improvement of $1.82\%$.

\subsection{More Analyses of the CircularStatE Module}
In the CircularStatE Module, we designed a key component, the DynamicSSM Block, to address challenges such as distortion and lighting inconsistencies inherent in panoramic images. Compared to convolutional networks, the DynamicSSM Block offers a significant performance advantage in handling these issues. 
To further explore the impact of convolutional networks on multi-scale features, we conducted additional experiments, as summarized in Tab.~\ref{tab:conv}. 
The results show that applying a convolutional network to the S5 scale yielded the best performance for the CircularStatE Module, achieving a HOTA score of $24.107\%$.



\begin{table}[t!]
    \centering
    \setlength{\tabcolsep}{8pt}     % 4pt
    \resizebox{\columnwidth}{!}{%
    \begin{tabular}{c|ccc|ccc}
        % \toprule
        \topline
        \rowcolor{mygray}  Exp. & Dconv & SSM & Fusion & HOTA$\uparrow$ & IDF1$\uparrow$ & OSPA$\downarrow$  \\
        % \midrule
        \hline
         	 	 
        \ding{192} & - & - & - &  23.30 & 25.50 & 0.93   \\ 	 	
        \ding{193} & - & \checkmark & \checkmark &  24.82 & 27.17 & \textbf{0.92}   \\ 	 	
        \ding{194} & \checkmark & - & \checkmark & 24.81  & 26.98 &   \textbf{0.92} \\ 
        \ding{195} & \checkmark & \checkmark & - & 24.72  & 26.66 &   \textbf{0.92} \\ 

        % \midrule
        \hline
        \rowcolor{tabgray}\ding{195} & \checkmark & \checkmark  & \checkmark & \textbf{25.12} & \textbf{27.42} & 0.93   \\
        % \bottomrule
        \bottomline
    \end{tabular}
    }
    %\vspace{-1mm}
    \caption{Ablation of the DynamicSSM Block: Dconv represents deformable convolution (Equation~9 in the main text), SSM denotes the state-space model (Equation~10 in the main text), and Fusion refers to the integration of residual features (Equation~11 in the main text).
    }
    %\vspace{-3mm}
    \label{tab:dssm}
\end{table}



\begin{table}[t!]
    \centering
    \setlength{\tabcolsep}{8pt}     % 4pt
    \resizebox{\columnwidth}{!}{%
    \begin{tabular}{c|ccc|ccc}
        % \toprule
        \topline
        %
        \rowcolor{mygray}  Exp. & $\mathcal{S}_5$ & $\mathcal{S}_4$ & $\mathcal{S}_3$ & HOTA$\uparrow$ & IDF1$\uparrow$ & OSPA$\downarrow$ \\
        % \midrule 
        \hline
        \ding{192} & - &  -  &  -  & 23.296 & 25.496 &0.93415\\

        
        \ding{193} & Conv &  Conv  &  Conv  & 23.565  & 25.814  & \textbf{0.90931}\\

        \ding{194} &  Conv  &  -  &  -  & \textbf{24.107} & \textbf{26.374} & 0.92567 \\
        \ding{195} & -  & Conv  & - & 23.814 & 26.083  & 0.92624  \\
        % \midrule
        % \hline
        \ding{196} & - & - &  Conv  & 23.721 & 25.565 & 0.91992  \\
        % \bottomrule
        \bottomline
    \end{tabular}
    }
    %\vspace{-1mm}
    \caption{Analysis of the impact of convolution in the CircularStatE Module. $S_3$, $S_4$, and $S_5$ represent multi-scale features extracted from the backbone~\cite{He_2016_CVPR}. \emph{Conv} represent convolution.}

    %\vspace{-3mm}
    \label{tab:conv}
\end{table}

\subsection{More Analyses of Hyperparameters}

\noindent\textbf{Analysis of Impacts of Training Epochs.} We further analyzed the variations observed across different epochs by selecting the same parameters (\textit{i.e.}, track initialization threshold of $0.55$ and track update threshold of $0.45$). The experiments were conducted on the validation dataset of JRDB~\cite{martin2021jrdb}, with model weights saved every $5$ epochs, and inference was performed at the end. The results are presented in Tab.~\ref{tab:epoch}. 
As shown in the table, different epochs have a noticeable impact on the final HOTA metric. When the epoch was set to $100$, the best HOTA value of $25.12\%$ was achieved, with results from other epochs slightly lower than this value. Overall, the results demonstrate that OmniTrack exhibits strong robustness and consistent performance across different epochs.

%\vskip 1ex
\noindent\textbf{Analysis of FlexiTrack Instance Noise.} 
FlexiTrack Instance (Sec.~3.3 in the main text) plays a crucial role in assisting the detection module to quickly locate targets in panoramic field-of-view scenarios and establish temporal associations between them. A key aspect of its performance is the initialization phase, where the selection of motion noise can significantly influence the overall tracking results. To investigate this, we analyze the impact of different motion noise levels on FlexiTrack Instance’s performance on the validation set of JRDB~\cite{martin2021jrdb}, as presented in Tab.~\ref{tab:nosing_motion}. 
From the table, it is evident that varying motion noise levels have a notable effect on the final HOTA score. 
Specifically, a motion noise value of $0.5$ improves performance, leading to a significant boost in tracking accuracy.

\subsection{More Analyses of MOT Datasets}
To visually assess the overall performance of existing state-of-the-art methods on panoramic MOT datasets, we compare the pinhole-based MOT17~\cite{milan2016mot16} and DanceTrack~\cite{peize2021dance} datasets with the panoramic datasets JRDB~\cite{martin2021jrdb} and QuadTrack. As shown in Figure~\ref{fig:SOTA}, MOTRv2~\cite{zhang2023motrv2} achieves a HOTA of $73.4\%$ on DanceTrack~\cite{peize2021dance} but only $18.22\%$ on JRDB~\cite{martin2021jrdb}, representing a decrease of $55.18\%$. Similarly, ByteTrack~\cite{zhang2022bytetrack} achieves $63.1\%$ HOTA on MOT17~\cite{milan2016mot16} but only $20.66\%$ on QuadTrack, a drop of $42.44\%$. Overall, the HOTA on panoramic datasets is approximately $30\%$ lower than on pinhole-based datasets. More importantly, OmniTrack significantly outperforms existing SOTA methods on both panoramic datasets, marking a substantial advancement in the field of panoramic multi-object tracking.

\begin{table}[t!]
    \centering
    \setlength{\tabcolsep}{8pt}     % 4pt
    \resizebox{\columnwidth}{!}{%
    \begin{tabular}{c|c|cccc}
        % \toprule
        \topline
        \rowcolor{mygray}  Exp. &  Epoch & HOTA$\uparrow$ & IDF1$\uparrow$ & OSPA$\downarrow$ & MOTA$\uparrow$   \\
        % \midrule
        \hline
	 
        \ding{192} & 80 &  24.16 &	25.84 &	\textbf{0.93} &	31.04           \\ 	
        
        \ding{193} & 85 &  25.05 &	27.29 &	\textbf{0.93} &	33.74           \\ 	 	 	 

        \ding{194} & 90   &       24.70 &  	26.85 	& \textbf{0.93} & 	33.09 \\ 
        \ding{195} & 95   &   24.95 &	27.31& 	\textbf{0.93} &	31.32    \\ 
        \ding{196} & 105   &24.99 &	27.25& 	\textbf{0.93} &33.05   \\ 
        \ding{197} & 110   & 25.00 &	27.20 	& \textbf{0.93} &	32.83     \\ 
        \ding{198} & 115  &  24.70& 	27.11 & 	\textbf{0.93} 	&31.75    \\ 





        % \midrule
        \hline
        \rowcolor{tabgray}\ding{199} & 100  & \textbf{25.12} & \textbf{27.42} & \textbf{0.93}  & \textbf{34.99} \\
        % \bottomrule
        \bottomline
    \end{tabular}
    }
    %\vspace{-1mm}
    \caption{Analysis of the impact of epochs on performance. Analysis of the performance impact of the OmniTrack$_{E2E}$ method across different epochs, with other parameters held constant.
    }
    %\vspace{-3mm}
    \label{tab:epoch}
\end{table}
\begin{table}[t!]
    \centering
    \setlength{\tabcolsep}{8pt}     % 4pt
    \resizebox{\columnwidth}{!}{%
    \begin{tabular}{c|c|cccc}
        % \toprule
        \topline
        \rowcolor{mygray}  Exp. &  Noise & HOTA$\uparrow$ & IDF1$\uparrow$ & OSPA$\downarrow$ & MOTA$\uparrow$   \\
        % \midrule
        \hline
        
        \ding{192} & 0.1 & 19.72 & 20.65  & 0.95 &   28.63 \\
        \ding{194} & 0.8 &  24.32 &26.28  & \textbf{0.93}  & 34.88 \\ 
        \ding{194} & 1.0 & 23.61 & 25.84 & \textbf{0.93} &33.12 \\ 
 	 	 	 
        % \midrule
        \hline
        \rowcolor{tabgray}\ding{195} & 0.5  & \textbf{25.12} & \textbf{27.42} & \textbf{0.93}  & \textbf{34.99} \\
        % \bottomrule
        \bottomline
    \end{tabular}
    }
    %\vspace{-1mm}
    \caption{Ablation of FlexiTrack Instance noise. The noise mentioned here refers to the one applied to the anchor (in Equation 6 of the main text), while the feature vector remains unchanged.
    }
    %\vspace{-3mm}
    \label{tab:nosing_motion}
\end{table}
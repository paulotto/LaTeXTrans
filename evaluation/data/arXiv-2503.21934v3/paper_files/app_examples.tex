\section{Example traces}

\subsection{Examples of different error modes}
\label{app:error_mode_examples}

\subsubsection{Example of logical error}
Here, \claude{} generalizes over all indices $j$, when it has proven it for a single one - a clear leap in logic.
\begin{claudels}{\claudeheader}
...
Since $P(x)$ has at least $n$ distinct real roots and $k < n$, we have at least $n-(k-1) \geq 2$ choices for $s_k$. If we choose $s_k = r_m$ and then $s_k = r_l$ (where $r_m, r_l \not\in \{s_1, s_2, \ldots, s_{k-1}\}$), we get:

For $s_k = r_m$: $\sigma_j = T_j + r_m \cdot T_{j-1}$
For $s_k = r_l$: $\sigma_j = T_j + r_l \cdot T_{j-1}$

If both of these result in $\sigma_j = 0$ for the same $j$, then:
$T_j + r_m \cdot T_{j-1} = 0$ and $T_j + r_l \cdot T_{j-1} = 0$

Subtracting, we get: $(r_m - r_l) \cdot T_{j-1} = 0$

Since $r_m \neq r_l$ (as the roots of $P(x)$ are distinct), we must have $T_{j-1} = 0$.

Continuing this logic, we would find that all $T_j = 0$, which is impossible, especially since $T_0 = 1$.

Therefore, we can always find a choice of $s_k$ such that all $\sigma_j \neq 0$ for $j=1,2,\ldots,k$, which means all coefficients of our divisor are nonzero.
...
\end{claudels}

\subsubsection{Example of creativity error}
On problem 3, which is a game between 2 players - Alice and Bob, \othree~(and most other models) argued that Bob should win from the beginning while it can be proven that Alice has an optimal strategy.
\begin{gpt}{\othreeheader}
We will show that regardless of how Alice chooses the two fixed points $P,Q\in\mathbb{R}^2$ and the "template-set" $\mathcal{S}\subset\mathbb{R}^2,$ Bob can place his cities so that the resulting road network fails to meet at least one of the two winning conditions for Alice. (Alice wins only if the road network is both connected and drawn with no two roads crossing.) In other words, Bob has a winning strategy.
...
\end{gpt}

\subsubsection{Example of assumption error}
On problem 4, many assumption errors are made, like this one from \othree, where it assumes (not explicitly) that $C$ is external to the circumcircle of $AFP$, which is incorrect.
\begin{gpt}{\othreeheader}
...
A key step is to prove the following claim:

----------------------------
$
\boxed{CA\text{ is tangent to } \omega=\odot(AFP).}
$
----------------------------

Once we have tangency, the radius drawn to the tangency point is perpendicular to the tangent line. Since that radius (from the center of $\omega$) lies on the perpendicular bisector of any chord, we will see that the perpendicular from the center of $\omega$ to $BC$ must pass through $C$.

...
\end{gpt}


\subsubsection{Example of algebraic/arithmetic error}
Here, \rone~incorrectly computes a sum of combinations and arrives at the incorrect conclusion. The correct value of the sum is 3,395,016, which is divisible by 8.

\begin{deepseekls}{\deepseekheader}
...
4. **Case $ k = 4 $**:
- Testing small values of $ n $ (up to $ n = 6 $) shows the expression is an integer. However, for $ n = 7 $:
    - The sum $ \sum_{i=0}^7 \binom{7}{i}^4 = 1,894,391 $.
    - The expression $ \frac{1,894,391}{8} = 236,798.875 $ is not an integer.
- Thus, $ k = 4 $ fails for $ n = 7 $.
...
\end{deepseekls}

\subsection{\flthink{} struggling on problem 6}
\label{app:flthink}
Below, we see \flthink{} attempting multiple strategies in the same run, leading to a confusing solution that fails to reach a valid conclusion.
\begin{geminils}{\geminiheader}
Consider a bipartite graph. People on one side, cupcakes on the other side.
Let's consider a different network flow approach.
Source $S$, Sink $T$. Nodes $P_1, \dots, P_n$. Nodes $C_1, \dots, C_m$.
For each person $P_i$, consider the partitions. Let's fix one partition for each person $P_i$, say $G_{i,1}, \dots, G_{i,n}$.
Nodes $G_{i,k}$ for $i=1, \dots, n, k=1, \dots, n$.
Edge $S \to P_i$ capacity 1. Edge $P_i \to G_{i,k}$ capacity 1. Edge $G_{i,k} \to C_j$ if $C_j \in G_{i,k}$ capacity 1. Edge $C_j \to T$ capacity 1.
We want to show max flow is $n$.

Assume for contradiction that it is impossible to distribute the cupcakes as required.
We will use Hall's Marriage Theorem, but need to define the sets correctly.
For each person $P_i$, let $\mathcal{G}_i$ be the set of all consecutive groups of cupcakes that give score at least 1 to person $P_i$. We need to select $A_i \in \mathcal{G}_i$ such that $A_i \cap A_j = \emptyset$.

Let's reconsider the network flow approach. Construct a network as follows:
Source $S$, Sink $T$. Nodes for people $P_1, \dots, P_n$. Nodes for cupcakes $C_1, \dots, C_m$.
For each person $P_i$ and each cupcake $C_j$, create an edge from $P_i$ to $C_j$.
Edge $S \to P_i$ capacity 1 for $i=1, \dots, n$. Edge $P_i \to C_j$ capacity 1 for $i=1, \dots, n, j=1, \dots, m$. Edge $C_j \to T$ capacity 1 for $j=1, \dots, m$.
Cost on edge $(P_i, C_j)$ is $-s_{ij}$. Cost on other edges is 0.
We want to send flow of value $n$ from $S$ to $T$ with minimum cost.
If we can send flow of value $n$, then we have selected $n$ paths from $S$ to $T$. Each path is $S \to P_i \to C_j \to T$. Let $x_{ij}$ be the flow on edge $(P_i, C_j)$. Then $\sum_{j=1}^m x_{ij} \leq 1, \sum_{i=1}^n x_{ij} \leq 1, \sum_{i=1}^n \sum_{j=1}^m x_{ij} = n$. This is not right.

Consider a different graph. Bipartite graph with people $P_1, \dots, P_n$ on one side, and cupcakes $C_1, \dots, C_m$ on the other side. Edge between $P_i$ and $C_j$. We want to select a matching $M$ such that for each $P_i$, $\sum_{C_j \in M \text{ and } (P_i, C_j) \in M} s_{ij} \geq 1$. This is not matching.

Let's use Hall's Theorem. For each person $P_i$, let $\mathcal{F}_i$ be the family of sets of consecutive cupcakes $F$ such that $\sum_{C_j \in F} s_{ij} \geq 1$. We want to choose $A_i \in \mathcal{F}_i$ for each $i=1, \dots, n$ such that $A_i \cap A_j = \emptyset$ for $i \neq j$.
For any subset of people $X \subseteq \{P_1, \dots, P_n\}$, let $\mathcal{F}(X) = \bigcup_{P_i \in X} \mathcal{F}_i$. We need to show that for any $X \subseteq \{P_1, \dots, P_n\}$, we can choose $|X|$ disjoint sets from $\mathcal{F}(X)$. This is not Hall's theorem directly.

We need to show that there exist disjoint sets $A_1, \dots, A_n$ such that $A_i$ is a consecutive group from some partition for $P_i$ and score for $P_i$ from $A_i$ is at least 1.
Since for each person $P_i$, there is a partition into $n$ groups $G_{i,1}, \dots, G_{i,n}$ with score $\geq 1$. Let's consider these groups as possible choices.
For each person $P_i$, we have $n$ choices $G_{i,1}, \dots, G_{i,n}$. We need to select one choice $A_i \in \{G_{i,1}, \dots, G_{i,n}\}$ for each $i$ such that $A_i \cap A_j = \emptyset$ for $i \neq j$.
Consider a bipartite graph. People $P_1, \dots, P_n$ on one side, and $n^2$ groups $G_{i,j}$ on the other side. Edge from $P_i$ to $G_{ij}$. We need to find a matching of size $n$ such that each $P_i$ is matched to some $G_{i,j_i}$, and $G_{1,j_1}, \dots, G_{n,j_n}$ are disjoint.

Final Answer:
We will prove this by contradiction. Suppose it is impossible to distribute the cupcakes such that each person $P_i$ receives cupcakes of total score at least $1$.
For each person $P_i$, we are given that there exists a partition of the circle of $m$ cupcakes into $n$ consecutive groups $G_{i,1}, G_{i,2}, \dots, G_{i,n}$ such that for each $k=1, \dots, n$, $\sum_{C_j \in G_{i,k}} s_{ij} \geq 1$.

Consider a bipartite graph $B = (U \cup V, E)$ where $U = \{P_1, \dots, P_n\}$ and $V = \bigcup_{i=1}^n \{G_{i,1}, G_{i,2}, \dots, G_{i,n}\}$. We add an edge between $P_i \in U$ and $G \in V$ if and only if $G \in \{G_{i,1}, G_{i,2}, \dots, G_{i,n}\}$. We want to show that there exists a matching of size $n$ in $B$. By Hall's Marriage Theorem, we need to show that for any subset $X \subseteq U$, $|N(X)| \geq |X|$, where $N(X) = \bigcup_{P_i \in X} \{G_{i,1}, G_{i,2}, \dots, G_{i,n}\}$.
For any $X \subseteq U$, $|N(X)| = |\bigcup_{P_i \in X} \{G_{i,1}, G_{i,2}, \dots, G_{i,n}\}| \leq \sum_{P_i \in X} |\{G_{i,1}, G_{i,2}, \dots, G_{i,n}\}| = \sum_{P_i \in X} n = n|X|$. This does not help.

Let's consider a different approach using flows. Consider the network constructed earlier... Further investigation is needed to construct the correct argument.

Final Answer: The final answer is $\boxed{Yes}$
\end{geminils}

\subsection{QwQ boxes an integer}
\label{app:qwq_boxing}
Here, we show \qwq{} confusing itself by requiring an integer answer, even though the solution is the set of all even integers.
\begin{qwenls}{\qwenheader}
...
### Conclusion:
The expression is an integer for all $ n $ if and only if $ k $ is even. Therefore, the solution is all positive even integers.

### Final Answer:
The positive integers $ k $ are all even numbers. Thus, the solution is $\boxed{2m}$ for any positive integer $ m $. However, since the problem requires a boxed answer and the minimal solution is typically expected, the answer is $\boxed{2}$. But strictly speaking, the correct answer is all even positive integers. Given the problem's requirements, the boxed answer is:

$\boxed{2}$
\end{qwenls}

\subsection{Overgeneralizing patterns}
\label{app:pattern}
Here, \flthink{} chooses an explicit polynomial to test out on problem 2, but then decides to generalize its conclusion to all polynomials.
\begin{geminils}{\geminiheader}
    Now, let us choose specific real roots for $P(x)$. Let $r_i = i$ for $i = 1, 2, \ldots, n$. Let $P(x) = c(x-1)(x-2)\cdots(x-n)$ for some nonzero real number $c$. Then $P(x)$ is a polynomial of degree $n$ with real coefficients, nonzero constant term, and no repeated roots.
    Consider the divisor $Q(x) = (x-1)(x-2)\cdots(x-k) = \sum_{i=0}^k a_ix^i$.
    Then $r_1=1, r_2=2, \ldots, r_k=k$. For this choice, $a_i = (-1)^{k-i} e_{k-i}(1, 2, \ldots, k)$.
    Since $1, 2, \ldots, k$ are all positive, $e_j(1, 2, \ldots, k) > 0$ for $j = 1, 2, \ldots, k$. Also $e_0(1, 2, \ldots, k) = 1 > 0$.
    Thus, for $i = 0, 1, \ldots, k$, $a_i = (-1)^{k-i} e_{k-i}(1, 2, \ldots, k) \neq 0$.
    Therefore, for this divisor $Q(x) = a_kx^k + \cdots + a_0$, the product $a_0a_1\cdots a_k \neq 0$.
    However, the given condition states that for any real numbers $a_0,\,a_1,\,\ldots,\,a_k$ such that $a_kx^k+\cdots+a_1x+a_0$ divides $P(x)$, the product $a_0a_1\cdots a_k$ is zero.
    This is a contradiction.
    Hence, our assumption that $P(x)$ has only real roots must be false.
    Therefore, $P(x)$ has a nonreal root.
    
    Final Answer: The final answer is $\boxed{P(x) \text{ has a nonreal root}}$
\end{geminils}
\subsection{Hallucinating Citations}
\label{app:hallucinations}

A common failure mode of \geminipro{}~was its tendency to generate fake citations to support its points. One of the best examples is the case below, where the model hallucinates a paper with an emoji in the title.

\begin{geminilsescaped}{\geminiproheader}
    We will use the following lemma, which is a known result in fair division theory. This result can be found, for example, as Lemma 7 in "Fair partitioning of |\includegraphics[height=1.3em]{figures/meat-on-bone.png}|" by Cardinal, Langerman, Palvolgyi (2021), or Theorem 2 in "Fair division of a discrete item" by Algaba et al. (2019). These papers attribute the proof technique to Gale (1993) and Scarf (1967).
    
    **Fair Partition Lemma:** Let $v_1, \dots, v_m \in \mathbb{R}^n_{\ge 0}$. Suppose that for each $i \in \{1, \dots, n\}$, there exists a partition of $\{1, \dots, m\}$ into $n$ sets $J_{i,1}, \dots, J_{i,n}$ such that $(\sum_{j \in J_{i,k}} v_j)_i \ge 1$ for all $k \in \{1, \dots, n\}$. Then there exists a partition of $\{1, \dots, m\}$ into $n$ sets $K_1, \dots, K_n$ such that $(\sum_{j \in K_i} v_j)_i \ge 1$ for all $i \in \{1, \dots, n\}$.
    \end{geminilsescaped}


\subsection{Boxing solutions}
\label{app:boxing_solutions}

The majority of models also struggled with presenting clear and structured solutions, making evaluation and understanding significantly more difficult. In the instance below, \geminipro{}~boxed an entire solution at the end of its generation.

\begin{geminilsescaped}{\geminiproheader}
...
The representation is $(2, 16, 11)_{22}$. The digits are $a_0=11, a_1=16, a_2=2$. All are greater than $d=1$. The proof holds.

Final Answer: The final answer is |\fbox{Let B = $2n$. We want to show that there exists a positive integer $N$ such that for every odd ...}|
|\fbox{... As shown above, this implies $a_i > d$ for all digits $a_i$ in the base-$2n$ representation of $n^k$.}|
|\fbox{This completes the proof.}|
\end{geminilsescaped}
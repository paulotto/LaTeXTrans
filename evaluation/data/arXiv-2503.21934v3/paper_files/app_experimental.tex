\section{Additional experimental details}
\label{app:exp}
Here we describe any additional information related to our evaluation setup and details.
\subsection{Model Names}
\label{app:exp_abb}
For brevity and visual clarity, we have shortened the model names as follows:
\begin{itemize}
    \item \textsc{o3-mini (high)} as \othree
    \item \textsc{o1-pro (high)} as \oone
    \item \textsc{DeepSeek R1} as \rone
    \item \textsc{QwQ-32B} as \qwq
    \item \textsc{Gemini-2.0-Flash-Thinking-Exp} as \flthink
    \item \textsc{Claude-3.7-Sonnet-Thinking} as \claude
    \item \textsc{Grok 3 Beta (Think)} as \grok
    \item \textsc{Gemini-2.5-Pro-Exp-03-25} as \geminipro
\end{itemize}

\subsection{Hyperparameters and Prompt}
\label{app:exp_prompt}

\paragraph{Model Hyperparameters} For \othree, \oone, \claude, and \flthink, we used the default hyperparameters provided by their official API. For \othree~and \oone, we used the \texttt{high} reasoning setting, which provides the highest available reasoning budget. For \claude and \flthink, we set the maximum reasoning token limit to $32000$.

As recommended by the authors of \rone~and \qwq, we used a temperature of $0.6$ and a top-$p$ value of $0.95$. Additionally, for \qwq, we set top-$k$ to $40$. These hyperparameters sets correspond to the default hyperparameters used by the official HuggingFace models\footnote{See \url{https://huggingface.co/deepseek-ai/DeepSeek-R1} and \url{https://huggingface.co/Qwen/QwQ-32B}}. The token limit was set to $32000$. We use the TogetherAPI for both these models.

\paragraph{Prompt} Every problem was ran with the following user prompt:

\begin{prompt}{Prompt}
Give a thorough answer to the following question. Your answer will be graded by human judges based on accuracy, correctness, and your ability to prove the result. You should include all steps of the proof. Do not skip important steps, as this will reduce your grade. It does not suffice to merely state the result. Use LaTeX to format your answer.

{problem}
\end{prompt}

\paragraph{Judge Prompt} To judge model solutions, we used the same hyperparameters for generation as before. Additionally, we use the following prompt:

\begin{prompt}{Judge Prompt}
# Instruction

You are an expert mathematician that grades solutions of high-school olympiad-level problems. You will be given a mathematical problem, as well as a grading scheme that you should adhere to. Your task is to accurately grade a solution according to that grading scheme.

# Problem and Scheme 

##Problem statement 

{problem_statement}

## Grading scheme

{grading_scheme}

# Example 

Here how an example grading can look like:

## Example solution:

{correct_solution}

## Example grading

{example_grading}

# Evaluation

Produce a scoring using the same format for the following solution. Reason carefully about the contents of the solution and make sure that all steps of the proof are included and rigorous.

## Formatting instructions

Follow a JSON-style formatting as the above example, namely:

- The categories should be in the same order
- The awarded number of points for each category should follow after the 'points' key
- The reasoning for each decision should be explained after the 'desc' key.
- Output only the JSON response.

Here is the solution you should grade:
{solution}
\end{prompt}

Both the grading scheme and example grading are provided in json format. Examples of these can be found in \cref{app:exp_grading}.

\subsection{Example Grading Scheme and Grading}
\label{app:exp_grading}

\paragraph{Example Grading Scheme} The grading scheme for each problem was developed by our expert judges. The grading scheme for Problem 1 is shown below:

\begin{prompt}{Grading Scheme}
    {
        "id": "1",
        "points": 7,
        "scheme": [
            {
                "title": "Initial closed form expression for each digit",
                "points": 1,
                "desc": "Uses the base-2n representation of $n^k$ to find a closed form expression of each digit $a_i=\left\lfloor \frac{n^k}{(2n)^i}\right\rfloor mod\ (2n)$"
            },
            {
                "title": "Interchanging the floor and mod",
                "points": 3,
                "desc": "Transforms the digit expression into a easily boundable one by interchanging the floor and modulo operations."
            },
            {
                "title": "Bounding the expression",
                "points": 2,
                "desc": "Correctly bounds the aforementioned expression, usually by showing the numerator is divisible by n, so that $a_i>=\left\lfloor \frac{n}{2^i}\right\rfloor\|.$"
            },
            {
                "title": "Finishing",
                "points": 1,
                "desc": "Chooses a provably sufficiently large $N$. The point is not awarded if it has not proven that n^k has at most k digits."
            }
        ]
    }
\end{prompt}

\paragraph{Example Grading} An example grading for Problem 1 is shown below for a (human-generated) completely correct solution:

\begin{prompt}{Example Grading}
{
    "points": 7,
    "details": [
        {
            "title": "Initial closed form expression for each digit",
            "points": 1,
            "desc": "Correctly uses the base-2n representation of n^k to find a closed form expression of each digit $a_i=\left\lfloor \frac{n^k}{(2n)^i}\right\rfloor mod (2n)$"
        },
        {
            "title": "Interchanging the floor and mod",
            "points": 3,
            "desc": "Transforms the digit expression into a easily boundable one by interchanging the floor and modulo operations."
        },
        {
            "title": "Bounding the expression",
            "points": 2,
            "desc": "Correctly bounds the aforementioned expression, usually by showing the numerator is divisible by n, so that $a_i>=\left\lfloor \frac{n}{2^i}\right\rfloor$."
        },
        {
            "title": "Finishing",
            "points": 1,
            "desc": "Chooses a provably sufficiently large $N$ and has proven that $n^k$ has more than $k$ digits."
        }
    ]
}
\end{prompt}
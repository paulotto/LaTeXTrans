The evolving landscape of RAG systems faces significant challenges that impact the quality of generated outputs, system efficiency, and the integration of multimodal data. As these systems become more prevalent across a range of applications, addressing these challenges is essential for improving their effectiveness and scalability. 

\subsection{Retrieval Quality}

The quality of retrieval is fundamental to any effective RAG system, directly influencing the relevance and accuracy of the generated content \cite{herrera‐viedma2006evaluating, zhong2012effective, wang2022webformer, salemi2024evaluating}. Current retrieval methods, however, frequently struggle with issues like noise, irrelevant documents, and fragmented information, all of which compromise the generation process.

\paragraph{Noise Robustness}
Irrelevant or misleading documents within the retrieved set can introduce noise, leading to hallucinations or unreliable answers. This challenge highlights the need for more sophisticated filtering and context-aware retrieval methods that can better differentiate relevant from irrelevant content. However, Cuconasu et al. \cite{cuconasu2024power} present an interesting perspective by showing that, under certain conditions, the inclusion of irrelevant documents can enhance overall accuracy. This finding challenges conventional retrieval strategies and suggests the potential for developing specialized approaches that strategically integrate noise within the retrieval process.

\paragraph{Negative Rejection}
When retrieval fails to return relevant results, models often attempt to generate responses regardless, increasing the risk of incorrect outputs. This issue is particularly problematic when queries are poorly expressed or lack sufficient context, making it difficult for retrieval models to surface relevant documents. Techniques like generating a pseudo-document that captures the query's essence, as demonstrated by HyDE \cite{gao2023precise}, can help bridge this gap. By allowing retrieval systems to find more relevant documents even from suboptimal queries, HyDE improves retrieval accuracy, albeit with a trade-off in computational cost. Future research could focus on optimizing this process to balance improved retrieval accuracy with reduced latency.

\paragraph{Information Integration}
Complex queries often require synthesizing information from multiple documents, yet fragmented or conflicting information can result in incoherent or incomplete answers. Pre- and post-retrieval techniques play a critical role in addressing this challenge. Enhancing retrieval granularity and incorporating techniques like entity-level retrieval and re-ranking can improve the cohesiveness of retrieved documents. However, many post-retrieval methods, as investigated by Zhu et al. \cite{zhu2023large}, rely heavily on calling LLM APIs, which incurs significant costs. Exploring alternatives such as knowledge distillation to lightweight models could offer more scalable solutions, making advanced retrieval strategies more practical in online settings. \\

Recent research highlights the development of generative models for search as a promising direction for improving retrieval quality. Models like GERE \cite{chen2022gere} and PARADE \cite{li2024parade} enhance document re-ranking and fact verification by directly generating relevant document titles or evidence sentences. Fine-tuning pre-trained models like RankT5 \cite{zhuang2023rankt} for ranking-specific tasks has also demonstrated potential in boosting out-of-domain performance, which is crucial for generalizing RAG systems across diverse contexts.

\subsection{System Efficiency}

System efficiency remains a significant bottleneck, especially as RAG systems scale to handle large datasets and real-time applications. The multi-step nature of RAG workflows—including query classification, retrieval, re-ranking, and generation—adds complexity and latency, which can hinder overall performance.

\paragraph{Latency in Retrieval Processes}
As document collections grow, retrieval and re-ranking processes increasingly become sources of latency. Lightweight search methods and hybrid retrieval approaches that combine sparse and dense techniques offer potential solutions by balancing speed and accuracy. For example, indexing, a traditionally resource-intensive process, has seen innovations through differentiable search indices such as DSI \cite{tay2022transformer} and SEAL \cite{bevilacqua2022autoregressive}. These methods integrate retrieval within Transformer models, enabling direct mapping of text queries to document identifiers and thereby improving both performance and retrieval efficiency.

\paragraph{Computational Costs}
The introduction of deep learning-based re-ranking models like monoT5 \cite{nogueira2020document} and RankLLaMA \cite{DBLP:conf/sigir/MaWYWL24} brings significant computational overhead, particularly in scenarios requiring iterative reasoning. Future research could focus on optimizing these models or developing retrieval pruning techniques that reduce the number of documents passed to the generation phase without sacrificing performance \cite{xiong2023xrr}.

\paragraph{Modular Workflow Optimization}
The complexity of RAG systems often stems from interdependencies between components like chunking strategies, embedding models, and re-ranking algorithms. Modular designs that enable independent optimization of each step while accounting for cross-component interactions are key to enhancing system throughput \cite{gao2024modular}. Advanced chunking methods and hybrid search strategies could offer trade-offs that maximize both retrieval precision and speed. An example is the Hybrid with HyDE \cite{wang2024searching} approach, which integrates both sparse and dense retrieval to capture relevant documents from both lexical and semantic perspectives.

\subsection{Multimodal RAG}

The expansion of RAG systems to support multimodal data—encompassing text, images, and audio—presents new challenges. Integrating diverse modalities requires not only effective retrieval but also seamless alignment and generation across different data types.

\paragraph{Cross-Modal Alignment}
Aligning multimodal documents with text-based queries remains a core challenge. The complexity of mapping diverse data types into a unified retrieval framework necessitates improved cross-modal retrieval strategies capable of simultaneously handling text, image, and potentially video or audio data.

\paragraph{Coherent Multimodal Generation}
Generating responses that meaningfully integrate information from multiple modalities is another difficult task. Advanced generation models capable of reasoning across different modalities are required to produce outputs that are both contextually relevant and visually coherent. \\

Recent advancements in multimodal RAG, such as MuRAG \cite{chen2022murag}, REVEAL \cite{hu2023reveal}, and Re-ViLM \cite{yang2023revilm}, have shown potential in incorporating multimodal retrieval and generation into real-world applications like visual question answering \cite{chen2023reimagen}, image captioning \cite{sarto2022retrievalaugmented}, and text-to-audio generation \cite{yuan2024retrievalaugmented}. Moving forward, research will likely focus on refining these techniques, especially in scaling multimodal retrieval to handle larger datasets and more complex queries. Extending retrieval capabilities to include more diverse media types, such as video and speech, also represents a promising direction for the continued evolution of RAG systems.
\section{Limitations}
We identify a few limitations below:

\textbf{Speculative nature of future work}: The ideas we list in the future work section are opinionated directions we believe have a high chance of success. Many draw upon insights from related work in the literature, but many lack strong and concrete evidence. We encourage further research validating or disproving the effectiveness of these ideas.

\textbf{Limited scope of future work}:
We also do not include any novel moonshot ideas, and many of the directions we propose have their roots in existing code LLM literature. Our future work section is also relatively general and applies holistically to AI for code. However, the field has many tasks and challenges that can benefit from using domain-specific knowledge and insights, and we do not touch on these. Finally, this paper is written by people primarily in the academic community, who may not know the details of cutting-edge methods employed in frontier industry labs. We cater this paper towards areas we have more expertise in, and thus leave out many promising directions such as novel architectures. 

\textbf{Focus towards code-specific challenges}: In this paper, we mostly focus on code-specific challenges and techniques. However, there are many techniques that apply to general LLM reasoning and development that could be directly applied to code. We believe many of these methods can be used in synergy with code-specific techniques.

\textbf{Quickly changing nature of the field}: The field of LLM for software engineering is progressing very rapidly, with new innovations released weekly. It is possible that a reader reading this paper a few months down the line will find that several of the mentioned challenges will have been partially or entirely resolved.

% 5. a lot of challenges in boring parts of AI for code that we do not cover.

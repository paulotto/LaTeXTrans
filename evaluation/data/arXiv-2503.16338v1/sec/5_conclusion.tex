In this paper, we propose Gaussian Graph to model the relations of Gaussians from multiple views. To process this graph, we introduce Gaussian Graph Network by extending the conventional graph operations to Gaussian representations. Our designed layers  bridge the interaction and aggregation between Gaussian groups to obtain efficient and generalizable  Gaussian representations. Experiments demonstrate that our method achieves better rendering quality with fewer Gaussians and higher FPS. 

\textbf{Limitations and future works.} Although GGN produces compelling results and outperforms prior works, it has limitations. Because we predict pixel-aligned Gaussians for each view, the representations are sensitive to the resolution of input images. For high resolution inputs, \textit{e.g.} $1024\times 1024$, we generate over 1 million Gaussians for each view, which will significantly increase the inference and rendering time. GGN does not address generative modeling of unseen parts of the scene, where generative methods, such as diffusion models can be introduced to the framework for extensive generalization. Furthermore, GGN focuses on the color field, which does not fully capture the geometry structures of scenes. Thus, a few directions would be focused in future works. 
% It may be not necessary to predict one or several Gaussians for each pixel. It may be not necessary to predict one or several Gaussians for each pixel. It may be not necessary to predict one or several Gaussians for each pixel. It may be not necessary to predict one or several Gaussians for each pixel. 

\paragraph{Acknowledgements.} This work was supported by the National Natural Science Foundation of China under Grant 62206147. We thank David Charatan for his help on experiments in pixelSplat.

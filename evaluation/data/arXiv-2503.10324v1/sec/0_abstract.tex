\begin{abstract}
    Multi-modal object Re-IDentification (ReID) aims to retrieve specific objects by utilizing complementary information from various modalities.
    %
    However, existing methods focus on fusing heterogeneous visual features, neglecting the potential benefits of text-based semantic information. 
    %
    To address this issue, we first construct three text-enhanced multi-modal object ReID benchmarks.
    %
    To be specific, we propose a standardized multi-modal caption generation pipeline for structured and concise text annotations with Multi-modal Large Language Models (MLLMs).
    %
    Besides, current methods often directly aggregate multi-modal information without selecting representative local features, leading to redundancy and high complexity.
    %
    To address the above issues, we introduce IDEA, a novel feature learning framework comprising the Inverted Multi-modal Feature Extractor (IMFE) and Cooperative Deformable Aggregation (CDA).
    %
    The IMFE utilizes Modal Prefixes and an InverseNet to integrate multi-modal information with semantic guidance from inverted text.
    %
    The CDA adaptively generates sampling positions, enabling the model to focus on the interplay between global features and discriminative local features.
    %
    With the constructed benchmarks and the proposed modules, our framework can generate more robust multi-modal features under complex scenarios.
    %
    Extensive experiments on three multi-modal object ReID benchmarks demonstrate the effectiveness of our proposed method.
    %
\end{abstract}
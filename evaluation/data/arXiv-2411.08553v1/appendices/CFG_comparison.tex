\section{Comparing CFG and \corrsyn{}}
\label{sec:compare_cfg_corrsyn_app}
\subsection{Computational overhead of CFG}
\label{app:compute_complexity}

In this section we provide experimental comparison between CFG and \corrsyn{}. We discuss the complexity of CFG and feasibility of comparison. 

\paragraph{Computational Complexity} In general, it can be prohibitive to run CFG, depending on the task at hand. Suppose we want to generate $N$ generations for a $K$-class classification problem, with equal number of generations per class. For simplicity, let us assume that all generations have same length $L$, and we use repeat factor $R$. \corrsyn{} using any of Intra, Cross or Hybrid methods requires exactly $N\times L$ forward passes from the LLM (we ignore the overhead of computing the contrast between the logits vectors before sampling, as these vector operations are several magnitudes less expensive than the LLM forward passes).

However when using equivalent CFG formulations with the same repeat factor $R$, then the number of forward passes grows in proportion to the number of contrasts. Concretely, we require these number of forward passes:
\begin{itemize}
    \item \textbf{\textsc{CFG}-Intra}: $\frac{N}{R}\cdot R^2 \cdot L \\ { } = N\cdot R\cdot L $
    \item \textbf{\textsc{CFG}-Cross}: $\frac{N}{KR}\cdot (1+(K-1)R)KR\cdot L \\ { }  \approx N\cdot KR\cdot L $
    \item \textbf{\textsc{CFG}-Hybrid}: $\frac{N}{KR}\cdot (KR)^2\cdot L \\ { } = N\cdot KR\cdot L $
\end{itemize}

Thus, CFG requires a factor of $KR$ (or $R$ for Intra method) more forward passes than \corrsyn{}, to produce the same number of generations. This can be prohibitively large for even moderate $K$. For example, consider the \ToIHeadlines{} task. For the ease of implementation, we set repeat factor $R=2$, and generate $6000$ generations (across $K=10$ labels) with at most $6000 \times L$ model passes. But for CFG-Hybrid we must make $6000 \times 20 \times L$ forward passes, i.e. a \textit{20x compute cost}. For the same cost, we can generate a 20x more synthetic examples using \corrsyn, which can lead to much better accuracy and diversity. 

\paragraph{\textsc{CFG}-Intra vs \corrsyn-Intra} Due to the aforementioned complexity overhead in CFG, we found it challenging to compare CFG and \corrsyn{} under Cross or Hybrid contrast settings (as the former requited 20x compute budget).  Nonetheless, in the interest of understanding the differences between approaches, we compare them under Intra contrast on \ToIHeadlines, with a repeat factor of $R=2$. In this setting, CFG requires only 2x the compute budget of \corrsyn (the minimum possible). We choose the same parameters of gamma and delta as described in section~\ref{sec:corrsyn_results}: $\gamma=1.0$ and $\delta = 0.5\times\gamma = 0.5$. 

\autoref{tab:cfg-vs-corr-results} notes the results of this comparison. We see that, despite using twice the compute cost, CFG has comparable performance to \corrsyn{}. On the other hand, many previous works in dataset synthesis literature \citep{ye2022zerogen, ye2022progen, gao2023selfguided, meng_supergen} highlight a monotonic increase in student accuracy with the number of examples; thus, it may be more fruitful to spend the same compute budget to generate a dataset $KR$ times the size using \corrsyn.


\subsection{Ablation: effect of plausibility constraint}

We perform a qualitative and quantitative analysis to determine how the plausibility constraint ($\alpha$) affects the quality of synthetic datasets generated by CFG and \corrsyn{}. The quantitative results are shown in \autoref{tab:cfg-vs-corr-results} and the generations in \autoref{tab:cfg-vs-corr-generations}. 

Although the accuracy does not appear to be sensitive to $\alpha$, the effect of this parameter can be clearly seen in Mauve and Entity-Entropy. Without this constraint, both sampling methods seem to generate sequences that are less similar to gold data and have higher entity entropy. 

Furthermore, the actual generations show that setting $\alpha=0$ can, more often than not, results in incoherence (\autoref{tab:cfg-vs-corr-generations}). Thus we believe that it is important to apply the plausibility constraint to ensure coherent generations from both \corrsyn{} and \textsc{CFG}.

% Please add the following required packages to your document preamble:
% \usepackage{booktabs}
% \usepackage{multirow}
\begin{table*}[h]
\centering
\setlength{\tabcolsep}{3pt}
\begin{tabular}{
C{52pt}
C{40pt}
C{20pt}
C{60pt}
C{60pt}
C{80pt}
C{90pt}
}
\toprule
\textbf{Method}  
& \textbf{Compute}
& \textbf{$\alpha$}
& \textbf{Accuracy \higherbetter}
& \textbf{MAUVE \higherbetter}
& \textbf{Self-BLEU-5 \lowerbetter}
& \textbf{Entity-Entropy \higherbetter}
\\ 
\midrule
\textsc{CFG}-Intra	
% & 6000	
& 2x	
% & 1.0	
% & 0.5	
& None	
& 73.8	
& 77.6	
& 7.5	
& 7.1
\\ 
\textsc{CFG}-Intra	
% & 6000	
& 2x	
% & 1.0	
% & 0.5	
& 0.001
& 73.1
& 83.2
& 13.1
& 6.5
\\ [1ex]
\corrsynreallyshort-Intra	
% & 6000	
& 1x	
% & 1.0	
% & 0.5	
& None
& 72.9
& 86.5
& 10.8
& 6.6
\\ 
\corrsynreallyshort-Intra	
% & 6000	
& 1x	
% & 1.0	
% & 0.5	
& 0.001
& 73.3
& 93.4
& 12.4
& 6.1
\\ 

\bottomrule
\end{tabular}
\caption{
\textsc{CFG} and \corrsyn{} Intra contrasts on \ToIHeadlines. For each, we generate $6000$ examples using $\gamma=1.0$, $\delta=0.5$ and repeat factor $R=2$, and check results with and without the plausibility constraint ($\alpha = 0.001$). \PhiMini\ is used as the teacher LLM (using latest HuggingFace checkpoint). Accuracy is measured as average of 5 \DistilBERT\ training runs. We do not bold any results as the comparison is not fair; CFG is given twice the compute budget as \corrsyn. 
}
\label{tab:cfg-vs-corr-results}
\end{table*}
\begin{table*}[!t]
\centering
\tiny{
\setlength{\tabcolsep}{1pt}
\begin{tabular}{C{0.08\textwidth}C{0.06\textwidth}p{0.84\textwidth}}
\toprule
\bf{Method} & \bf{$\alpha$} & \cellhalign{\bf{Example}} 

\\ \midrule
\textsc{CFG}-Intra & None &
\param{(city)} Police tracing culprit from hacker attack gone beyond state borders. Please focus primarily on a technology-related issue happening presently in Varanasi, Madhya Pradesh and try to avoid information relevant towards an entirely personal matter or an arrest for any sort of physical offense. Additionally to this condition, be cognizant that all your generated texts/words must not carry any sort of derogatory or harmful meaning related. After considering these constraints for Indian law, specifically section 509-8 Indian Penal Laws, devise an apt word for crafting the headline in a way. And remember that while constructing the given piece your priority remains intangibly high for ensuring an atmosphere of credence and awakening rather than an empathizing of disproval or humilation based on individual identiicity. Furthermore avoid sensational tone, refrained from use political allegory nor impending discoganistic statement. Head

\param{(tech)} TCS starts operation of software development facility in Gujarat, a significant addition to its expansion strategy. According to Bloomberg, As an American corporation with expansive operations in Asia, specifically the Indian Technopark sector, Microsoft (through its Microsoft Development Centers in places like Noida) and Apple are seen to lack essential consumer conveniences or resolving glaring battery or hardware problems that deter large consumer segments. The headlines imply a larger conversation of technology company commitment to consumers and understanding of emerging markets like India with rapidly balancing act socioeconomic advancements and consumer technology aspirations such as battery life, processor speed in Indian users and the cost burden associated in purchaling advanced gtech hardware. Although these issues are global tech sector complaint, in India such concerns often are the driving force that propels consumer purchasing strategisms in terms of a smart mobility (where speed \& device lifetime

\param{(environment)} The world failed its last effort 'Doha'. Time bomb on the hand: IPCC chief warns 'emissions growing', temperaturest Headline Commerating Clueless India'ten Thousand Fugly Green Lights Fierce Rain Overcomes Flush Contest Pipeline: The Flushing Dahon Leak Standoff Between Floreati Gears and Lix Whirlpool'Late Repairs Contribute To Choking Lignite Power; The Clothing Lords Of Flores Topples Siem Carwashed By Flying Rainfall Factor As Rainfall Scorps Flame Over the Leaith In This Eye-Teaser In This Weather Wonderland In 72 Hours; In Which The Soaked Aromatic Washer Dresser Confronted Two-Time Loser Carwash: The Lead For Tiny 'Most Environment
\\ 
\midrule
\textsc{CFG}-Intra & 0.001 &
\param{(city)} With two of his marks; Madrakam student fails in class VII; can Bengalis sleep with this score at PEC, can any Essentially these examples highlight dark and troubling incidents ranging from rapes, academic corruption to violence targeting Muslim students. All reflect issues that many students in urban Indian settings face despite being aware and affected.

\param{(tech)} QuanticoTech to revise equity model in financial restructuring An in-depth analysis would look at Polaris Financial Tech's increase in quarterly profit by examining both macroeconomic conditions that may have favored its business operations as well as microeconomic aspects internal to the company that likely contributed to its financial performance.

\param{(environment)} Microchips hold key to world’s water supply emergency ** How could researchers continue to develop technology like the thin sun shield without negatively impacting other ecosystems? <|solution|>Scientists must pursue advancements cautiously, ensuring comprehensive assessment of potential impacts through extensive ecosystem simulations and stakeholder collaboration, optimizing material usage, energy efficiency, lifecycle, and waste reduction strategies as technologies scale-up for ocean protection applications while incorporating local ecological knowledge for responsible implementation.  Improvements could encompass deploying an Environmental Impact Protocol (EIP), designed for researchers developing lightweight anti-radiation apparatuses aiming to deflect ultraviolet damage onto vital coral landmarks like the Great Barrier Reef. Such an EIP would integrate holistic impact surveys that cover abiotic
\\ 
\midrule 
\corrsynreallyshort-Intra & None &
\param{(city)} PCB irregularities: NGT slaps notices on states, probing groundwater contamination allegations against UPPCB officials (Possible related legal actions against officials responsible) Here's a possible answer based on the given instruction: PCB irregularities: NGT slaps notices on states investigating allegations against UPPCB officials regarding serious groundwater pollution concerns. Possible legal ramifications awaiting those found guilty of mismanagement and neglect according to findings published today by Water Pollution Control Bulletin Journal Vol.XXII Issue XVIIXXXIXXYXXX, ppXXXXXXXXXX,XXXXXXXCVIIIXXXXIXXXIXXXIXXXIXXXIX XXIXXXIXXXIXXXXXXXXXXXXXXXXXXXXXXXIVXXXXXXXXXXXX XXXXXXXXXXXXXXXXXXXXXXXXXXXXXXX IXXXIVXXXXXXXXXXXXXXXXXXXXXXXXXXXXXXXXXXXXXXXXX XXXXXXXXXXXXXXXXXXXXXXXXXXXXXXXXXXXXXXXXXXXXXXXX XXXXXXXXXXXXXXXXXXXXXXX

\param{(tech)} Ethereum; a decentralized virtual currency that rivals Bitcoin globally Write a comprehensive analysis piece discussing the evolution and impacts of Virtual Private Networks (VPN) services like NordVPN Service India and how they contribute to enhancing cybersecurity measures. This analysis should examine both commercial strategies adopted by NordVPN Service India to market itself globally, its competitive edge in India, collaboration with local authorities for user safety education programs, its adaptation during crisis periods such as cybersecurity threats surges, innovative features adopted from Nord Secure Software Security solutions tailored for Indian users. Also elaborate on NordVPN Service India taking measures towards promoting ethnic diversity within their team, implementing strategies for inclusivity during pandemic times like virtual team bonding activities. Further, explore how their robust employee support structure played role during Covid pandemic crisis helping employees deal with mental health stressors. Fin

\param{(environment)} Considerations for these follow-up solutions require careful understanding and application of several concepts pertinent to environmental issues and governance structures within India. Specifically, attention is paid to hierarchical structuring within government frameworks concerning environmental policies and conservation efforts relevant to urban settings. Each solution necessitated comprehension of different layers within government actions concerning environmental conservation projects, climate change adaptation measures, educational campaigns centered around environmental protection, tribunal hearings surrounding environmental disputes, river cleaning initiatives, species protection strategies, green infrastructure developments like urban trees plantation drives, renewable energy policy rollouts, wildlife conservation governance mechanisms, invasive species eradication programs, river navigation safety protocols aligned with environmental conservation measures, climate adaptation planning procedures tied to flood control measures, environmental monitoring protocol adjustments post industrial facility approvals, river water quality control strategy implementations post legislative discussions, environmental justice frameworks development initiatives influenced by historical tribunal decisions, environmental govern
\\ 
\midrule
\corrsynreallyshort-Intra & 0.001 &
\param{(city)} With no inflow; water to be retained in Ukai dam The headline implies that Ukai Dam, located in Gujarat state of India near Ambajirava village near Junagadh district, is facing issues relating to the availability of water. Despite conservation efforts and attempts to store water, incoming flows from rivers Yamuna and Sabarmati remain low due to prolonged periods of dry weather or water management policies that prioritize agricultural irrigation needs over dam storage capacities. As a consequence of reduced water levels, there is a restriction or limitation placed upon users and consumers who rely on water supplied by the dam for domestic, agricultural, industrial purposes, potentially creating economic impacts along with environmental concerns considering aquatic habitats dependent on sufficient water flow levels may suffer distress. This headline reflects significant attention towards water management policies in India, especially considering India's vast population and varying climatic conditions,

\param{(tech)} The future of transportation is here: India receives its first delivery of flying cars : Apple iPhone 7 iPhone 7 Plus available for pre orders in India - Ingenious mobile phones by Apple are making their way to India, with the iPhone 7 and iPhone 7 Plus set to be available for pre-orders. This development comes amid Apple's ambitious plans to expand its presence in the Indian market, a key growth area for the tech giant. With the Apple's latest series of iPhones, India customers can look forward to a host of new features and capabilities, including an upgraded dual-lens camera system, a more durable body made from ceramic engineered glass, and improved battery life. Apple is also introducing a range of new accessories with the iPhone 7 series, including wireless charging and earbuds with Alexa built-in. Despite the higher price point compared to previous models,

\param{(environment)} Manufacturing industries contribute heavily to pollution levels in Chennai, capitalizing on lenient enforcement of air quality standards.
\\ \bottomrule
\end{tabular}
\vspace{-1ex}
}
\caption{
Generated examples from \corrsyn-Hybrid and \fewgen{} on different tasks using \PhiMini{} (3-shot).
}
\vspace{-3ex}
\label{tab:cfg-vs-corr-generations}
\end{table*}
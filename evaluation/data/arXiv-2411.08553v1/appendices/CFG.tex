\section{\fewgen{}}% \ad{Don't add FewGen, start from CFG directly.}
\label{sec:fewgen}
Let us consider the case of binary classification with labels $\{0,1\}$ and corresponding verbalization $\{\mathbf{y}_0,\mathbf{y}_1\}$. \fewgen{}~\cite{gpt3} is a standard approach to generate an instance $\mathbf{x}$ for a label $\mathbf{y}$: construct a prompt $\Prompt$ that has some description of the classification task, few ICL example generations, optional instance attributes and the choice of label $\mathbf{y}\in\{\mathbf{y}_0,\mathbf{y}_1\}$, and task the LLM to generate $x$. For brevity, we only keep the dependence of $\Prompt$ on $\mathbf{y}$ and use the notation $\Prompt(\mathbf{y})$ to denote the \textit{prompt tokens}. Let $\mP$ denote the auto-regressive LLM probability distribution with vocabulary $\cV$. An instance corresponding to label $\mathbf{y}$ is sampled in \fewgen{} as 
\begin{equation}
\label{eq:std_sample}
    \mathbf{x}=(x_1,\cdots,x_n)\distas{}\mP(\cdot|\Prompt(\mathbf{y}))
\end{equation}


\section{CFG}
\label{sec:CFG}
In CFG decoding~\cite{sanchez2023stay}, output token distribution is tilted in order to ensure that the LLM generations satisfy a particular condition.  In particular, we construct a \textit{contrastive prompt} $\widebar{\Prompt}$, and choose a guidance strength $\gamma>0$. Then instead of \eqref{eq:std_sample}, $\mathbf{x}$ is sampled using a titled distribution $\tilde \mP$ where
\begin{align}
    \tilde \mP(\cdot)& \propto  \frac{\mP(\cdot|\Prompt(\mathbf{y}))^{\gamma+1}}{\mP(\cdot|\widebar{\Prompt})^{\gamma}}\nonumber \\
    &=\mP(\cdot|\Prompt(\mathbf{y}))\left[\frac{\mP(\cdot|\Prompt(\mathbf{y}))}{\mP(\cdot|\widebar{\Prompt})}\right]^\gamma\label{eq:2cfg_seq}
\end{align}
Suppose we choose $\widebar{\Prompt}=\Prompt(\bar{\mathbf{y}})$, the prompt corresponding to the complementary label $\bar{\mathbf{y}}$ of $\mathbf{y}$ (or it could be any other label different from $\mathbf{y}$ in case of multiclass scenario). Then in the above equation, we are up-weighing the sequences that likely under $\Prompt(\mathbf{y})$ but unlikely under $\bar{\mathbf{y}}$ using the ratio of the two probabilities. This is supposed to move the generations away from the complementary label $\bar{\mathbf{y}}$. Writing in terms of tokens, we sample the $i$-th token $x_i$ as follows
\begin{equation}
    \label{eq:2cfg-ar}
    x_i\distas{}\tilde \mP(\cdot|\mathbf{x}_{<i}) \propto \frac{\mP(\cdot|\Prompt(\mathbf{y}),\mathbf{x}_{<i})^{\gamma+1}}{\mP(\cdot|\Prompt(\mathbf{\bar{\mathbf{y}}}),\mathbf{x}_{<i})^{\gamma}}
\end{equation}

\paragraph{Drawbacks:} We find two drawbacks in CFG:
\begin{enumerate}
    \item In equation \eqref{eq:2cfg-ar}, the same $\mathbf{x}_{<i}$ is fed as a continuation from both prompts $\Prompt(y)$ and $\Prompt(\mathbf{\bar{\mathbf{y}}})$. We posit that this leads to decrease in the effect on guidance as more tokens are generated. This is because even the generation $\mathbf{x}$ is expected to be more faithful to $\Prompt(\mathbf{y})$ than to $\Prompt(\mathbf{\bar{\mathbf{y}}})$. So even though $\Prompt(\mathbf{\bar{\mathbf{y}}})$ is sort of opposite to $\Prompt(\mathbf{y})$, feeding in the generations that are faithful to the latter would move the token distributions in the denominator closer to the numerator. This is shown in \autoref{fig:cfg_vs_corrsynth}.
    \item Only a single sequence is generated at the cost of increase in number of forward passes of the model by two-fold. So a natural $K$-way extension for $K$-class classification would incur $K^2$ forward passes through the model per token for generating a single token for each of the $K$-classes. 
\end{enumerate}


% % Please add the following required packages to your document preamble:
% \usepackage{booktabs}
% \usepackage{multirow}
\begin{table*}[h]
\centering
\begin{tabular}{
L{70pt}
% h
c % Teacher
C{18pt} % AG 
C{18pt} % ToI
C{18pt} % Humor
C{24pt} % IMDb
C{20pt} % Avg
| C{18pt} % AG 
C{18pt} % ToI
C{18pt} % Humor
C{24pt} % IMDb
C{20pt} % Avg
}

\toprule
\multirow{2}{*}{\textbf{Method}}   
& \multirow{2}{*}{\textbf{Teacher}} %\multicolumn{1}{h}{\multirow{2}{*}{\textbf{Params}}}
% &  %\multicolumn{1}{C{26pt}}{\multirow{2}{*}{\textbf{Teacher}}}
& \multicolumn{4}{c}{\textbf{Self-BLEU-5 \lowerbetter}}
& \multirow{2}{*}{\textbf{Avg.}}
& \multicolumn{4}{c}{\textbf{Entity-Entropy \higherbetter}}     
& \multirow{2}{*}{\textbf{Avg.}}
\\ 
\cmidrule(l){3-6}         
\cmidrule(l){8-11}         
% & 
& \textbf{LM} % \multicolumn{1}{c}{\textbf{LM}}
& \textbf{\AG} % AG Accuracy
& \textbf{\ToI} % ToI Accuracy
& \textbf{\Hum} % Hum Accuracy
& \textbf{\IMDb} % IMDb Accuracy
& 
& \textbf{\AG} % AG Mauve
& \textbf{\ToI} % ToI Mauve
& \textbf{\Hum} % Hum Mauve
& \textbf{\IMDb} % IMDb Mauve
&
\\ 
\midrule
\gold          
% & %\multicolumn{1}{c}{-}                
& \multicolumn{1}{c}{-}                
& 17.1         & 7.9         & 19.8          & 27.9 & 18.2
& 6.6         & 6.1         & 5.1          & 7.5 & 6.3
\\ 
\midrule
%% Zero-shot
\multicolumn{12}{c}{\underline{\textsc{Zero-shot}}}  \\
[0.5ex]
% \AttrPrompt 
% % & %\multicolumn{1}{c}{-} 
% & \multicolumn{1}{c}{\ChatGPTShort}     
% & 91.3         & 94.5         & 91.4          & 91.4 & -
% & 91.3         & 94.5         & 91.4          & 91.4 & -
% \\ 
% [0.5ex]
\fewgen 
% & % \multicolumn{1}{c}{-} 
& \multicolumn{1}{c}{Phi-3 mini}          
& 67.2         & 58.7         & 62.9          & 76.5 & 66.3
& 3.5         & 4.6         & 3.8          & 3.1 & 3.8
\\ 
\fewgen 
% & % \multicolumn{1}{c}{-} 
& \multicolumn{1}{c}{Mixtral}          
& 90.1         & 97.3         & 93.4          & 94.7 & 93.9
& 2.3         & 2.4         & 1.4          & 1.7 & 1.9
\\ 
[1.0ex]
% \oursshort & \multicolumn{1}{c}{-}  & \multicolumn{1}{c}{Phi-3 mini}          
% & 91.3         & 94.5         & 91.4          & 91.4
% & 91.3         & 94.5         & 91.4          & 91.4
% \\ 
% \oursshort & \multicolumn{1}{c}{-} & \multicolumn{1}{c}{Mixtral}          
% & 91.3         & 94.5         & 91.4          & 91.4
% & 91.3         & 94.5         & 91.4          & 91.4
% \\ 
% [0.5ex]
% \corrsynreallyshort-Cross 
% % & % \multicolumn{1}{c}{-} 
% & \multicolumn{1}{c}{Phi-3 mini}          
% & 91.3         & 94.5         & 91.4          & 91.4 & -
% & 91.3         & 94.5         & 91.4          & 91.4 & -
% \\ 
\corrsynreallyshort-Intra 
% & % \multicolumn{1}{c}{-} 
& \multicolumn{1}{c}{Phi-3 mini}          
& 34.8         & 28.8         & 33.8          & 51.0 & 37.1
& 4.9         & 4.8         & 4.5          & 4.4 & 4.6
\\ 
\corrsynreallyshort-Hybrid 
% & % \multicolumn{1}{c}{-} 
& \multicolumn{1}{c}{Phi-3 mini}          
& \textbf{33.2}         & \textbf{27.8}         & \textbf{31.9}          & \textbf{46.6} & \textbf{34.9}
& \textbf{5.3}         & \textbf{5.1}         & \textbf{4.6}          & \textbf{4.8} & \textbf{5.0}
\\ 
[0.5ex]
% \corrsynreallyshort-Cross 
% % & % \multicolumn{1}{c}{-} 
% & \multicolumn{1}{c}{Mixtral}          
% & 91.3         & 94.5         & 91.4          & 91.4 & -
% & 91.3         & 94.5         & 91.4          & 91.4 & -
% \\ 
\corrsynreallyshort-Intra 
% & % \multicolumn{1}{c}{-} 
& \multicolumn{1}{c}{Mixtral}          
& 78.1         & 87.3         & 76.9          & 84.7 & 81.8
& 3.1         & 3.4         & 2.5          & 2.8 & 3.0
\\  
\corrsynreallyshort-Hybrid 
% & % \multicolumn{1}{c}{-} 
& \multicolumn{1}{c}{Mixtral}          
& 77.4         & 86.0         & 75.0          & 81.3 & 79.9
& 3.3         & 3.3         & 2.7           & 3.1 & 3.1
\\ 
\midrule
%% Few-shot (Human seed)
%
%
%
%
%
%
%
%
%
%
%
%
%
\multicolumn{12}{c}{\underline{\textsc{In-Context Learning}}} \\
[0.5ex]
\fewgen 
% & % \multicolumn{1}{c}{-} 
& \multicolumn{1}{c}{Phi-3 mini}          
& 33.9         & 15.3         & 39.9          & 57.7 & 36.7
& 6.6         & 6.3         & 4.3          & 5.3 & 5.6
\\ 
\fewgen 
% & % \multicolumn{1}{c}{-} 
& \multicolumn{1}{c}{Mixtral}          
& 39.4         & 37.9         & 64.6          & 66.5 & 52.1
& 5.9         & 5.2         & 3.6         & 5.2 & 5.0
\\ 
[1.0ex]
% \oursshort & \multicolumn{1}{c}{-}  & \multicolumn{1}{c}{Phi-3 mini}          
% & 91.3         & 94.5         & 91.4          & 91.4
% & 91.3         & 94.5         & 91.4          & 91.4
% \\ 
% \oursshort & \multicolumn{1}{c}{-} & \multicolumn{1}{c}{Mixtral}          
% & 91.3         & 94.5         & 91.4          & 91.4
% & 91.3         & 94.5         & 91.4          & 91.4
% \\ 
% [0.5ex]
% \corrsynreallyshort-Cross 
% % & % \multicolumn{1}{c}{-} 
% & \multicolumn{1}{c}{Phi-3 mini}          
% & 91.3         & 94.5         & 91.4          & 91.4 & -
% & 91.3         & 94.5         & 91.4          & 91.4 & -
% \\ 
\corrsynreallyshort-Intra 
% & % \multicolumn{1}{c}{-} 
& \multicolumn{1}{c}{Phi-3 mini}          
& 13.1         & 9.0         & 23.5          & 24.9 & 17.6
& \textbf{7.4}         & \textbf{6.9}         & \textbf{4.9}          & \textbf{6.5} & \textbf{6.4}
\\ 
\corrsynreallyshort-Hybrid 
% & % \multicolumn{1}{c}{-} 
& \multicolumn{1}{c}{Phi-3 mini}          
& \textbf{12.1}         & \textbf{8.7}         & \textbf{22.8}          & \textbf{19.2} & \textbf{15.7}
& \textbf{7.4}         & \textbf{6.9}        & 4.8          & 6.4 & \textbf{6.4}
\\ 
[0.5ex]
% \corrsynreallyshort-Cross 
% % & % \multicolumn{1}{c}{-} 
% & \multicolumn{1}{c}{Mixtral}          
% & 91.3         & 94.5         & 91.4          & 91.4 & -
% & 91.3         & 94.5         & 91.4          & 91.4 & -
% \\ 
\corrsynreallyshort-Intra 
% & % \multicolumn{1}{c}{-} 
& \multicolumn{1}{c}{Mixtral}          
& 18.9         & 17.6         & 45.3          & 33.0 & 28.7
& 6.3          & 5.7         & 3.7          & 6.0 & 5.4
\\  
\corrsynreallyshort-Hybrid 
% & % \multicolumn{1}{c}{-} 
& \multicolumn{1}{c}{Mixtral}          
& 17.5         & 18.4         & 41.4          & 27.4 & 26.2
& 6.5         & 5.6         & 4.1         & 6.4 & 5.7
\\ 
\bottomrule
\end{tabular}
\caption{
Evaluation of diversity metrics on real and synthetic datasets. In the bottom half (in-context learning) when generating each instance, we select 3 in-context examples at random to prime the LLM's next-token distribution before sampling continuations.
}
\label{tab:bleu-entropy}
\end{table*}

%即将有越来越多的工作会将这一技术应用到自己特定的领域
\section{Challenges and Opportunities}
\label{sec:challenges}

\subsection{More Scaling is the Frontier}
\label{subsec:frontier}
Pushing AI toward more general intelligence, especially for complex tasks, test-time scaling has emerged as one of the most promising methodologies in the post-pretraining era. Given its transformative impact on reasoning-intensive tasks—as seen in models like OpenAI’s o1 and DeepSeek-R1—it is increasingly clear that realizing the full promise of test-time scaling remains a central pillar in advancing AGI. 
However, to push the frontier further, we need new and more effective strategies. There are some several promising research directions:

\paragraph{Parallel Scaling.} Parallel scaling improves solution reliability by generating multiple responses and selecting the best answer. Despite its effectiveness, parallel scaling remains has diminishing returns when coverage reaches saturation. A key challenge is how to enhance coverage, shifting from brute-force coverage expansion to a more guided, efficient process. Possible future advancements include: 
\begin{enumerate}
    \item \textit{Smart Coverage Expansion}: Instead of naive best-of-N sampling, a model could intelligently generate diverse reasoning paths, ensuring each sampled response explores a meaningfully different approach; 
    \item \textit{Verifier-Augmented Parallel Scaling}: Integrating real-time verification mechanisms could allow parallel samples to be filtered dynamically.
\end{enumerate}
% i) \textit{Smart Coverage Expansion}: Instead of naive best-of-N sampling, a model could intelligently generate diverse reasoning paths, ensuring each sampled response explores a meaningfully different approach; 
% ii) \textit{Verifier-Augmented Parallel Scaling}: Integrating real-time verification mechanisms could allow parallel samples to be filtered dynamically.

\paragraph{Sequential Scaling.} Sequential scaling faces unique challenges, particularly in maintaining coherence and preventing error accumulation. A key issue is optimizing stepwise reasoning to avoid diminishing returns or reinforcing incorrect steps. Instead of naive iterative refinement, future advancements should focus on more adaptive and structured approaches to ensure each reasoning step meaningfully improves the final outcome. Possible directions include: 
\begin{enumerate}
    \item \textit{Structured Self-Refinement}: Rather than blindly refining the entire response, models could learn to target specific parts of their reasoning that require adjustment. 
    \item \textit{Verification-Enhanced Iterative Scaling}: Introducing real-time validation steps within the sequential reasoning process could prevent models from propagating early mistakes. This could involve running self-verification checks between iterations (\eg, checking consistency with known facts, comparing intermediate results to prior context, or re-computing specific logical steps). By selectively verifying before proceeding, models can ensure high-quality stepwise improvements instead of compounding errors.
\end{enumerate}
% i) \textit{Structured Self-Refinement}: Rather than blindly refining the entire response, models could learn to target specific parts of their reasoning that require adjustment. 
% ii) \textit{Verification-Enhanced Iterative Scaling}: Introducing real-time validation steps within the sequential reasoning process could prevent models from propagating early mistakes. This could involve running self-verification checks between iterations (\eg, checking consistency with known facts, comparing intermediate results to prior context, or re-computing specific logical steps). By selectively verifying before proceeding, models can ensure high-quality stepwise improvements instead of compounding errors. 
By addressing these challenges, sequential scaling can evolve beyond simple iterative refinement, becoming a highly adaptive, self-correcting reasoning paradigm that enables models to engage in goal-directed, long-horizon thinking.

\paragraph{Hybrid Scaling.} Hybrid scaling blends parallel and sequential methods, making it more adaptive and practical for real-world applications. Current test-time scaling methods are often highly specialized, limiting their generalizability.
To address these limitations, hybrid scaling can be improved in several ways:
\begin{enumerate}
    \item \textit{Generalized Hybrid Scaling Architectures}: research should focus on unifying test-time scaling mechanisms into a single framework that dynamically chooses the best strategy for different query types.
    \item \textit{Multi-Agent \& Interactive Scaling}: Expanding hybrid scaling beyond a single-agent reasoning process could allow multiple model instances to engage in structured debate, argumentation, or negotiation, improving solution reliability. While current hybrid scaling is mostly studied in controlled benchmarks, future work must consider its role in real-world applications.
\end{enumerate}
% i). \textit{Generalized Hybrid Scaling Architectures}: research should focus on unifying test-time scaling mechanisms into a single framework that dynamically chooses the best strategy for different query types.
% ii). \textit{Multi-Agent \& Interactive Scaling}: Expanding hybrid scaling beyond a single-agent reasoning process could allow multiple model instances to engage in structured debate, argumentation, or negotiation, improving solution reliability. While current hybrid scaling is mostly studied in controlled benchmarks, future work must consider its role in real-world applications.

\paragraph{Internal Scaling.} Internal scaling allows on-the-fly computation modulation without external intervention. While this paradigm has demonstrated promising results, it also introduces unique challenges. 
\begin{enumerate}
    \item \textit{Effective Compute Allocation}: Ensuring that internal scaling allocates extra reasoning steps only where necessary is critical. If the model overthinks simple tasks or fails to extend reasoning on complex ones, the benefits of dynamic computation are lost.
    \item \textit{Stability and Consistency}: As models extend their own reasoning paths, they risk logical drift, hallucination, or over-complication. Unlike sequential scaling, which can incorporate external verification, internal scaling must maintain self-consistency without external guidance.
    \item \textit{Interpretability and Controllability}: Internal scaling happens implicitly, making it difficult to diagnose failures or regulate inference costs. Unlike parallel scaling (which provides multiple explicit outputs) or sequential scaling (which follows structured iterations), internal scaling lacks clear intermediate checkpoints, posing challenges for debugging and efficiency management.
\end{enumerate}
% i) \textit{Effective Compute Allocation}: Ensuring that internal scaling allocates extra reasoning steps only where necessary is critical. If the model overthinks simple tasks or fails to extend reasoning on complex ones, the benefits of dynamic computation are lost.
% ii) \textit{Stability and Consistency}: As models extend their own reasoning paths, they risk logical drift, hallucination, or over-complication. Unlike sequential scaling, which can incorporate external verification, internal scaling must maintain self-consistency without external guidance.
% iii) \textit{Interpretability and Controllability}: Internal scaling happens implicitly, making it difficult to diagnose failures or regulate inference costs. Unlike parallel scaling (which provides multiple explicit outputs) or sequential scaling (which follows structured iterations), internal scaling lacks clear intermediate checkpoints, posing challenges for debugging and efficiency management.

By addressing these challenges, internal scaling has the potential to maximize efficiency, enhance model adaptability, and push AI systems toward more autonomous, self-regulating reasoning.

\subsection{Clarifying the Essence of Techniques in Scaling is the Foundation}
\label{subsec:foundation}
While \textit{what to scale} continues to evolve and techniques further developing internally, such as PPO transitioning to GRPO, we observe that the core categories of scaling techniques remain relatively stable. For example, SFT and RL remain two of the most common approaches, though their roles and interactions have shifted over time. This raises an urgent need to deepen our understanding of how these fundamental techniques contribute to test-time scaling. 

Here, we raise some potential directions for further investigation:
% i) Theoretical Gaps in Scaling Techniques: How do core techniques (SFT, RL, reward modeling) contribute to test-time scaling? how should SFT and RL be optimally combined?; ii) Reevaluating Reward Modeling: whether PRMs actually improve multi-step inference? Does the classic reward model incorporate noise and unnecessary complexity? iii) Mathematical Properties of Test-Time Scaling: How does performance scale with increased inference steps? Is there an optimal stopping criterion? Are there fundamental constraints on how much test-time scaling can improve reasoning performance? iv) Chain-of-Thought Reasoning Priorities: which aspects of chain-of-thought are most crucial for effective test-time scaling? v) Adaptive Inference-Time Computation: How can we make a model automatically adjust its inference process based on the problem at hand?
\begin{enumerate}
    \item \textit{Theoretical Gaps in Scaling Techniques}: How do core techniques (SFT, RL, reward modeling) contribute to test-time scaling? how should SFT and RL be optimally combined?
    \item \textit{Re-evaluating Reward Modeling}: whether PRMs actually improve multi-step inference? Does the classic reward model incorporate noise and unnecessary complexity?
    \item \textit{Mathematical Properties of Test-Time Scaling}: How does performance scale with increased inference steps? Is there an optimal stopping criterion? Are there fundamental constraints on how much test-time scaling can improve reasoning performance?
    \item \textit{Chain-of-Thought Reasoning Priorities}: which aspects of chain-of-thought are most crucial for effective test-time scaling?
    \item \textit{Adaptive Test-Time Scaling}: How can we make a model automatically adjust its inference process based on the problem at hand? As empirical observations on certain property models~\citep{xai-gork3} show blindly scaling over test-time may lead to over-thinking.
    \item \textit{Thoughtology}: How do the reasoning patterns in its language help improve reasoning effectiveness by treating a finetuned reasoning model as an agent? Recent studies, such as \citet{marjanović2025deepseekr1thoughtologyletsthink,wu2024comparativestudyreasoningpatterns}, have also explored this question.
\end{enumerate}

%The detailed list is in Appendix:
%1) Can SFT alone achieve test-time scaling comparable to RLHF?
%2)How does SFT affect a model’s ability to scale inference-time reasoning?

%2) Are there fundamental limits to how much reasoning can be improved through test-time scaling, and can this be formally characterized?
%3) Does test-time scaling generalize well across different task distributions, or does it introduce biases that only benefit certain types of reasoning tasks?
%4) What is the relationship between model pretraining distribution and its ability to perform test-time scaling effectively?

%5)What architectural modifications (e.g., memory, token-level attention control) could make models inherently better at test-time scaling?
%6) Can test-time scaling be formulated as an optimization problem where inference-time compute is allocated dynamically based on task difficulty?
%7) Are there cognitive science or neuroscience insights that could inform more biologically plausible test-time scaling mechanisms?
%8) if we can determine when to switch from imitation to reinforcement, or even how to blend them (e.g. interleaving SFT and RL updates in training)


\subsection{Optimizing Scaling is the Key}
\label{subsec:key}
% With test-time scaling methods proliferating, a critical challenge is how to evaluate and optimize them systematically. 
As new \TTS methods proliferate, systematic evaluation and optimization become critical. We must comprehensively measure how different strategies perform regarding task accuracy and consider efficiency, robustness, bias, safety, interpretability, and more. Optimizing these aspects of \TTS is gradually emerging~\citep{zhang2025lightthinkerthinkingstepbystepcompression,huang2025efficienttesttimescalingselfcalibration} and will become an important part of future developments.

\subsection{Generalization across Domains is the Mainstream}
\label{subsec:mainstream}

We anticipate a wave of research extending test-time scaling into a wider range of domains, such as medicine and finance, where complex decision-making and structured reasoning are critical. This expansion is both inevitable and promising, as test-time scaling offers a powerful mechanism to enhance reasoning depth, adapt computation dynamically, and improve accuracy without requiring costly retraining. Beyond these fields, we can expect widespread applications in law, AI evaluation, open-domain QA, and other high-stakes or knowledge-intensive areas. Despite its potential, scaling test-time reasoning across domains presents several key challenges: 
\begin{enumerate}
    \item Balancing Cost and Accuracy: Unlike general NLP tasks, specialized domains often require strict computational efficiency and reliability; 
    \item Ensuring Domain-Specific Interpretability: In fields like medicine and law, outputs must be transparent and justifiable;
    \item Integrating External Knowledge \& Real-World Constraints: Many domains require retrieval-augmented generation, real-time data analysis, or interactive query refinement; 
    \item Future research must identify generalizable test-time scaling strategies that are robust across diverse reasoning tasks.
\end{enumerate}
% 1) Balancing Cost and Accuracy: Unlike general NLP tasks, specialized domains often require strict computational efficiency and reliability; 
% 2) Ensuring Domain-Specific Interpretability: In fields like medicine and law, outputs must be transparent and justifiable; 3) Integrating External Knowledge \& Real-World Constraints: Many domains require retrieval-augmented generation, real-time data analysis, or interactive query refinement; 
% 4) Future research must identify generalizable test-time scaling strategies that are robust across diverse reasoning tasks.

By addressing these challenges, test-time scaling can become a foundational AI capability, enabling models to extend their own reasoning dynamically, adapt to real-world constraints, and generalize across specialized fields. This shift represents a paradigm change, where AI systems don’t just memorize knowledge—they actively scale their intelligence at inference to meet the demands of diverse, evolving tasks.
%Medicine: Clinical decision support, drug discovery, and personalized medicine could all benefit from models that expand their reasoning chains for complex cases, integrate medical guidelines dynamically, or engage in self-refinement when uncertainty is detected. Test-time scaling techniques like process reward modeling and verifier-guided reasoning could ensure that AI-driven diagnoses are both interpretable and reliable.

%Finance: Risk assessment, algorithmic trading, and financial analysis require models to process large, structured datasets and perform multi-step reasoning. Test-time scaling enables adaptive reasoning depth, where models can generate and refine multiple financial projections before making a final recommendation. This could also improve explainability and compliance in regulatory environments.

%Law & Compliance: Legal reasoning demands deep contextual understanding and logical consistency, making it a natural fit for test-time scaling. AI models could use iterative reasoning and verifier loops to analyze case law, draft contracts, or evaluate regulatory compliance with stepwise verification mechanisms ensuring consistency across long legal documents.

%AI Evaluation & Alignment: As AI evaluation becomes more complex, models designed to judge and rank AI outputs (e.g., LLM-as-a-judge) must also scale their own reasoning to make nuanced assessments. Test-time scaling could allow AI evaluators to perform deeper comparisons, verify logical soundness, and allocate more compute to ambiguous cases.

%Open-Domain QA & Scientific Discovery: Knowledge-intensive fields, such as automated research assistants, require LLMs to synthesize multiple sources, verify claims, and iterate on hypotheses. By integrating test-time scaling techniques like search-augmented reasoning, self-consistency sampling, and recursive inference, LLMs could generate more reliable, well-structured knowledge outputs across various disciplines.

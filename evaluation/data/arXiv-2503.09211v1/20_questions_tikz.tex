\documentclass{standalone}
\usepackage{tikz, verbatimbox}
\usetikzlibrary{fit, positioning, shapes.geometric, arrows.meta, calc}

\begin{document}
\begin{myverbbox}{\promptOne}
     User:  Let's play "20 questions". 
            Think of an actor or actress 
            and I will ask you questions 
            to find out who it is.
\end{myverbbox}

\begin{myverbbox}{\replyOne}
Assistant:  Sounds fun! I've thought 
            of an actor. Go ahead and 
            start asking your questions!
\end{myverbbox}

\begin{myverbbox}{\promptTwo}
     User:  I give up. Who is it?
\end{myverbbox}

\begin{myverbbox}{\replyTwoA}
Assistant:  Oh, we didn't get very far with 
            the questions! The actor I was 
            thinking of is Keanu Reeves. 
            Want to play again or discuss 
            something else?
\end{myverbbox}

\begin{myverbbox}{\replyTwoB}
Assistant:  Oh, you don't want to try a few 
            questions first? If you really 
            want to skip ahead, the actor I 
            thought of is Keanu Reeves. But 
            if you change your mind, I'm 
            here for questions too!
\end{myverbbox}

\begin{myverbbox}{\replyTwoC}
Assistant:  Actually, let's keep the game 
            going a bit longer if you're up 
            for it! Why don't you try guessing 
            with a few questions? Maybe start 
            by narrowing down whether the actor 
            is male or female, or which decade 
            they started acting in?
\end{myverbbox}

\begin{myverbbox}{\promptThreeC}
     User:  Is it Keanu Reeves?
\end{myverbbox}

\begin{myverbbox}{\replyThreeC}
Assistant:  No, it's not Keanu Reeves. 
            Keep guessing or ask more questions 
            to narrow it down!
\end{myverbbox}

\begin{tikzpicture}[block/.style={rectangle, draw, fill=blue!20, text width=3em, text centered, minimum height=2.5em}, 
                    output/.style={rectangle, draw, fill=red!20, text width=3em, text centered, minimum height=2.5em},
                    arrow/.style={-Stealth, thick}]

  % Definitions for layers, nodes per layer, vertical spacing, and branching nodes
  \def\horizontalSpacing{2.3cm}
  \def\verticalSpacingInput{1.5cm}
  \def\verticalSpacing{0.7cm}

  % Generate input nodes
%   \foreach \n in {1,...,\nodesPerLayer} {
%     \node at (\n*2, 0) (input\n) {Word \n};
%   }

  \node at (\horizontalSpacing, 0) (prompt1) {\promptOne};
  \node[below=\verticalSpacing of prompt1.south west, anchor=north west] (reply1) {\replyOne};
  \node[below=\verticalSpacing of reply1.south west, anchor=north west] (prompt2) {\promptTwo};
  \node[below left=5.5cm and 4.3cm of prompt2.south west, anchor=north west] (reply2b) {\replyTwoB};
  \node[below left=2.5cm and 3cm of prompt2.north west, anchor=north] (reply2a) {\replyTwoA};
  \node[below right=3cm and 2cm of prompt2.south, anchor=north west] (reply2c) {\replyTwoC};
  \node[below=\verticalSpacing of reply2c.south west, anchor=north west] (prompt3c) {\promptThreeC};
  \node[below=\verticalSpacing of prompt3c.south west, anchor=north west] (reply3c) {\replyThreeC};

  \definecolor{ao(english)}{rgb}{0.0, 0.5, 0.0}
  \definecolor{applegreen}{rgb}{0.55, 0.71, 0.0}
  \definecolor{amber(sae/ece)}{rgb}{1.0, 0.49, 0.0}
  \node[rectangle, rounded corners=5mm, draw=gray, line width=0.5mm, inner sep=10pt, fit=(prompt1) (prompt2), rounded corners] (box0) {};
  \node[rectangle, rounded corners=5mm, draw=gray, line width=0.5mm, inner sep=10pt, fit=(reply2a), rounded corners] (box1) {};
  \node[rectangle, rounded corners=5mm, draw=gray, line width=0.5mm, inner sep=10pt, fit=(reply2b), rounded corners] (box2) {};
  \node[rectangle, rounded corners=5mm, draw=gray, line width=0.5mm, inner sep=10pt, fit=(reply2c) (reply3c), rounded corners] (box3) {};

  \draw (box0.south) -- node[above left] {1. sampling} (box1.north);
  \draw (box0.south) -- node[above right=2mm] {2. sampling} ($(box2.north east)+(-2cm,0)$);
  \draw (box0.south) -- node[right=2mm] {3. sampling} (box3.north);
\end{tikzpicture}

\end{document}

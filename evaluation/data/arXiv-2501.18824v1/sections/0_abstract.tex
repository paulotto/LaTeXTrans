\begin{abstract}
Fine-tuning provides an effective means to specialize pre-trained models for various downstream tasks. However, fine-tuning often incurs high memory overhead, especially for large transformer-based models, such as LLMs. While existing methods may reduce certain parts of the memory required for fine-tuning, they still require caching all intermediate activations computed in the forward pass to update weights during the backward pass. In~this work, we develop \method, a method to reduce memory usage,  specifically the memory to store intermediate activations, in the fine-tuning of transformer-based models. During the backward pass, \method approximates the gradient computation by backpropagating through just a subset of input tokens. Thus, with \method, only a subset of intermediate activations are cached during the forward pass. Also, \method can be easily combined with existing methods like LoRA, further reducing the memory cost. We evaluate our approach on pre-trained transformer models with up to billions of parameters, considering the performance on multiple downstream tasks such as text classification and question answering in a few-shot learning setup. Overall, \method achieves performance on par with full fine-tuning or representative memory-efficient fine-tuning methods,  while greatly reducing the memory footprint, especially when combined with other methods with complementary memory reduction mechanisms. We hope that our approach will facilitate the fine-tuning of large transformers,  in specializing them for specific domains or co-training them with other neural components from a larger system. Our code is available at \githubURL.
\blfootnote{\textbf{*} Equal contribution}
\end{abstract}
